% J\"org Baumgart
% 2012-06-01
%
%
\section{Definition und Erzeugung von Querverweisen}

Die Grundlage f\"ur die Erzeugung eines Querverweises bildet die Definition eines 
\textbf{Labels}, z.\,B. \verb+\label{querverweis1}+\label{querverweis1}.

Mit dem Kommando \verb+\vref+, z.\,B. \verb+\vref{querverweis1}+, wird ein Querverweis mit 
den beiden folgenden Eigenschaften erzeugt:

\begin{seList}
\item 
Falls sich das Label auf eine \textsl{Abbildung}, eine \textsl{Tabelle}, ein \textsl{Listing} oder eine 
\textsl{Gleichung} bezieht, wird zus\"atzlich zur entsprechenden Nummer ein Text mit ausgegeben.
Beispielsweise erzeugt \verb+\vref{noten}+ \vref{noten}. Die zugeh\"origen Labels sind dann innerhalb 
der \verb+figure+-, \verb+table+-, \verb+programm-+ oder \verb+equation+-Umgebung definiert. 
Die auszugebenden Texte k\"onnen in der Datei\newline
\hspace*{\fill}\verb+wa-konfiguration-deutsch.tex+\hspace*{\fill}\newline  
umdefiniert werden.

Bezieht sich ein Label auf eine Textstelle, z.\,B. \verb+\label{querverweis1}+, dann wird die Kapitelnummer 
mit dem Zusatz \textsl{Kapitel} ausgegeben: \vref{querverweis1}\newline
F\"ur die Gliederungsebenen \verb+\chapter+, \verb+\section+, \verb+\subsection+, \verb+\subsubsection+ 
und \verb+\paragraph+ kann dieser \textsl{Zusatz} ebenfalls in der Datei \newline
\hspace*{\fill}\verb+wa-konfiguration-deutsch.tex+\hspace*{\fill}\newline 
umdefiniert werden. 
\item
Wenn sich der Querverweis auf die aktuelle Seite bezieht, dann wird keine Seitennummer ausgegeben.
\end{seList}

Bei der Verwendung des \verb+\vref+-Kommandos ist zu beachten, dass vor dem auszugebenden Text ein Leerzeichen 
eingef\"ugt wird. Im Normalfall hat dieses keine weitere Auswirkung. Wenn allerdings ein Absatz direkt mit einem 
\verb+\vref+-Kommando beginnt, dann wird der entsprechende Text nicht linksb\"undig ausgegeben, d.\,h. es liegt 
eine Verletzung des Blocksatzes vor.%

\vref{noten} stellt einen blocksatzverletzenden Querverweis dar.

Dieses \textsl{Problem} l\"asst sich durch die Anwendung des \verb+\vref*+-Kommandos vermeiden.\label{vrefstern} 

\vref*{noten} stellt einen nicht blocksatzverletzenden Querverweis dar.

Allerdings f\"uhrt die Verwendung des \verb+\vref*+-Kommandos innerhalb eines Satzes auch wieder 
zu einem nicht gew\"unschten Ergebnis: Das in \vref*{noten} dargestellte Klausurergebnis ... .%
\footnote{Der Grund, warum die Kommandos \texttt{\textbackslash{}vref} 
und \texttt{\textbackslash{}vref*}
in dieser Form definiert wurden, erschlie{\ss}t sich dem Autor dieses Dokuments allerdings nicht!}

Mit dem Kommande \verb+\pageref+ wird lediglich die Seitennummer ausgegeben, z.\,B. \verb+\pageref{noten}+ \pageref{noten}
oder \verb+\pageref{querverweis1}+ \pageref{querverweis1}.

