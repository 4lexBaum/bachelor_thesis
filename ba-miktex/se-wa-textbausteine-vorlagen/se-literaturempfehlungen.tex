\chapter{Literaturempfehlungen}

\begin{seList}
\item \textsl{Schlosser: Wissenschaftlich Arbeiten mit \LaTeX -- Leitfaden f\"ur Einsteiger}\seFootcite{}{}{Sch:WAS4}\newline
Dieses Buch bietet eine sehr gute, kompakte Einf\"uhrung in \LaTeX. Es enth\"alt Informationen zu 
aktuellen Paketen. 
\item \textsl{Mittelbach/Goossens: Der LaTeX-Begleiter}\seFootcite{}{}{MG:lat}\newline
Das Buch hat einen Umfag von 1138 Seiten und geht ausf\"uhrlich auf eine Vielzahl von Paketen ein. 
Zus\"atzlich werden grundlegende Konzepte vorgestellt, die einen Einblick in die Struktur des \LaTeX-Systems bieten.
Es ist allerdings kein Buch f\"ur einen schnellen Einstieg in \LaTeX, sondern als Nachschlagewerk f\"ur viele Fragen 
rund um das Thema \LaTeX{} gedacht.
\item \textsl{Kohm/Morawski: KOMA-Script -- Eine Sammlung von Klassen und Paketen f\"ur \LaTeXe}\seFootcite{}{}{KM:KS4}\newline
Das Buch stellt ausf\"uhrlich die KOMA-Script-Klassen f\"ur die Erstellung von B\"uchern, Reports, Artikeln und Briefen vor, 
die explizit die Regeln der europ\"aischen Typographie unterst\"utzen. Diese Klassen enthalten sehr viele Parameter, \"uber die 
sich das grundlegende Layout vergleichsweise komfortabel steuern l\"asst. Die KOMA-Script-Klasse \texttt{scrreprt} bildet die 
Grundlage f\"ur die Vorlagendateien zur Erstellung von Seminar-, Projekt- und Bachelorarbeiten.
\item \textsl{Lingnau: \LaTeX-Hacks -- Tipps \& Techniken f\"ur professionellen Textsatz}\seFootcite{}{}{Lin:lat}\newline
Das Buch enth\"alt eine Sammlung von Tricks, Methoden und Techniken aus den vielf\"altigen Anwendungsbereichen von 
\LaTeX{} und \TeX. Bei der Auswahl der Hacks wurde besonderes Augenmerk auf kreative L\"osungen gelegt, die mit 
\LaTeX{} m\"oglich sind.\seFootcite{Vgl.}{Klappentext}{Lin:lat}
\end{seList}