%%Seitenumbruch
%%\enlargethispage{2\baselineskip} 

% Konfigurationsdatei f\"ur die Pfaddefinitionen einlesen
\input{se-wa-pfade}
%
%
% Festlegung der Sprache: 
\newcommand{\seWaSprache}{deutsch}
%\newcommand{\seWaSprache}{englisch}

%
% Einlesen der .sty-Dateien
%
\input{\seWaPathSty/se-wa-input-styles-v098}

\lstdefinelanguage{JavaScript}{
keywords={typeof, new, true, false, catch, function, return, null, catch, switch, var, if, in, while, do, else, case, break, \$.},
keywordstyle=\color{blue}\bfseries,
ndkeywords={class, export, boolean, throw, implements, import, this},
ndkeywordstyle=\color{darkgray}\bfseries,
identifierstyle=\color{black},
sensitive=false,
comment=[l]{//},
morecomment=[s]{/*}{*/},
commentstyle=\color{purple}\ttfamily,
stringstyle=\color{blue}\ttfamily,
morestring=[b]',
morestring=[b]"
}

\lstset{
language=JavaScript,
backgroundcolor=\color{white},
extendedchars=true,
basicstyle=\footnotesize\ttfamily,
showstringspaces=false,
showspaces=false,
numbers=left,
numberstyle=\footnotesize,
numbersep=9pt,
tabsize=2,
breaklines=true,
showtabs=false,
captionpos=b
}

\usepackage{xcolor}

\colorlet{punct}{red!60!black}
\definecolor{delim}{RGB}{20,105,176}
\colorlet{numb}{magenta!60!black}

\lstdefinelanguage{json}{
    basicstyle=\normalfont\ttfamily,
    numbers=left,
    numberstyle=\scriptsize,
    stepnumber=1,
    numbersep=8pt,
    showstringspaces=false,
    breaklines=true,
    frame=lines,
    backgroundcolor=\color{white},
    literate=
     *{0}{{{\color{numb}0}}}{1}
      {1}{{{\color{numb}1}}}{1}
      {2}{{{\color{numb}2}}}{1}
      {3}{{{\color{numb}3}}}{1}
      {4}{{{\color{numb}4}}}{1}
      {5}{{{\color{numb}5}}}{1}
      {6}{{{\color{numb}6}}}{1}
      {7}{{{\color{numb}7}}}{1}
      {8}{{{\color{numb}8}}}{1}
      {9}{{{\color{numb}9}}}{1}
      {:}{{{\color{punct}{:}}}}{1}
      {,}{{{\color{punct}{,}}}}{1}
      {\{}{{{\color{delim}{\{}}}}{1}
      {\}}{{{\color{delim}{\}}}}}{1}
      {[}{{{\color{delim}{[}}}}{1}
      {]}{{{\color{delim}{]}}}}{1},
}


%
% Individuelle Konfiguration des Dokumentes
%
\input{\seWaPathText/wa-konfiguration}

%
% Definition von Abk\"urzungen, Symbolen und eventuell Glossareintr\"agen
%
% 2012-03-22 Verwendung des optionalen Parameters f\"ur die Pluralform einer Abk\"urzung
%
% 2012-02-06 Umstellung auf die neuen Kommandos
%
%
%
%  J\"org Baumgart
%  Definition einiger Abk\"urzungen
%  


% Definition von Abk\"urzungen
%
% 1. Parameter: Schluessel (key) der Abkuerzung
% 2. Parameter: Abkuerzung
% 3. Parameter: Vollform
% 4. Parameter: Vollform im Plural (optional; falls nicht definiert, wird der Wert des dritten Parameters verwendet)
%

\seNewAcronymEntry{dhbw}{DHBW}{Duale Hochschule Baden-W\"urttemberg}{}{}

\seNewAcronymEntry{usb}{USB}{Universal Serial Bus}{}

\seNewAcronymEntry{ctan}{CTAN}{Comprehensive \TeX{} Archive Network}{}

\seNewAcronymEntry{sap}{SAP}{eigenständiger Markenname - früher: \textit{Systeme, Anwendungen und Produkte in der Datenverarbeitung}}{}

\seNewAcronymEntry{mvc}{MVC}{Model-View-Controller}{}

\seNewAcronymEntry{vba}{VBA}{Visual Basic for Applications}{}

\seNewAcronymEntry{erp}{ERP}{Enterprise Resource Planing}{}

\seNewAcronymEntry{ui}{UI}{User Interface}{}

\seNewAcronymEntry{html}{HTML}{Hypertext Markup Language}{}

\seNewAcronymEntry{svg}{SVG}{Scalable Vector Graphics}{}

\seNewAcronymEntry{dom}{DOM}{Document Object Model}{}

\seNewAcronymEntry{xml}{XML}{Extensible Markup Language}{}

\seNewAcronymEntry{css}{CSS}{Cascading Style Sheets}{}

\seNewAcronymEntry{jpg}{JPEG}{Joint Photographic Expters Group}{}

\seNewAcronymEntry{png}{PNG}{Portable Network Graphics}{}

\seNewAcronymEntry{w3c}{W3C}{World Wide Web Consortium}{}

\seNewAcronymEntry{d3}{D3}{Data Driven Document}{}

\seNewAcronymEntry{kpi}{KPI}{Key Perfomance Indicator}{Key Perfomance Indicators}

\seNewAcronymEntry{oss}{OSS}{Open Source Software}{}

\seNewAcronymEntry{osi}{OSI}{Open Source Initiative}{}

\seNewAcronymEntry{c4a}{C4A}{Cloud for Analytics}{}

\seNewAcronymEntry{mit}{MIT}{Massachusetts Institute of Technology}{}

\seNewAcronymEntry{wtfpl}{WTFPL}{WHAT THE FUCK PUBLIC LICENSE}{}

\seNewAcronymEntry{json}{JSON}{JavaScript Object Notation}{}

\seNewAcronymEntry{gsn}{GSN}{Global Soccer Network}{}

%\seNewAcronymEntry{kdd}{KDD}{Knowledge Discovery in Data}{}

%\seNewAcronymEntry{dm}{DM}{Data Mining}{}

%\seNewAcronymEntry{crisp-dm}{CRISP-DM}{Cross Industry Standard Process for Data Mining}{}

%\seNewAcronymEntry{iot}{IoT}{Internet of Things}{}


%\seNewAcronymEntry{olap}{OLAP}{Online Analytical Processing}{}

%\seNewAcronymEntry{sql}{SQL}{Structured Query Language}{}

%\seNewAcronymEntry{knn}{KNN}{K-Nearest Neighbours}{}
\seNewAcronymEntry{matlab}{MATLAB}{MATrix LABoratory}{}

\seNewAcronymEntry{rss}{RSS}{Residual Sum of Squares}{}


\seNewAcronymEntry{tss}{TSS}{Total Sum of Squares}{}

\seNewAcronymEntry{cft}{CFT}{Curve Fitting Tool}{}

\seNewAcronymEntry{lowess}{Lowess}{Locally Weighted Scatter Plot Smooth}{}



\seNewAcronymEntry{dfl}{DFL}{Deutsche Fußball Liga}{}



\seNewAcronymGlossaryEntry{ml}{ML}{Machine Learning}%
{}%
{Das \textit{Lernen} umfasst viele komplexe Aspekte, welche sich nicht direkt auf einen Computer abbilden lassen. Maschinelles Lernen (\textit{Machine Learning}) verfolgt das Ziel durch den Entwurf von Algorithmen rechnergestützte und automatisierte Lernverfahren verfügbar zu machen, um aus den vorliegenden Daten neues Wissen zu generieren.\seFootcite{Vgl.}{S. 13}{Cleve.2014}}

\seNewAcronymGlossaryEntry{olap}{OLAP}{Online Analytical Processing}%
{}%
{Dieses Prinzip wird innerhalb von Business Warehouses verwendet, um innerhalb eines multidimensionalen Datenmodells, in Form eines Datenwürfels, durch gezielte Operationen (\textit{wie Slicing und Dicing, Roll-up, Drill-Down, usw.}) Daten abzufragen.\seFootcite{Vgl.}{S. 1067}{Elmasri.2011}}

\seNewAcronymGlossaryEntry{sql}{SQL}{Structured Query Language}%
{}%
{Innerhalb relationaler Datenbanken wird die deskriptive Abfragesprache SQL verwendet. Die umfassende Funktionalität vereinigt sowohl die Erstellung und Definition von Datenbankschemata (\textit{Data Definition Language}), wie auch die Manipulierung der eigentlichen Daten (\textit{Data Manipulation Language}).\seFootcite{Vgl.}{S. 87-88}{Elmasri.2011}}

\seNewAcronymEntry{knn}{KNN}{K-Nearest Neighbours}{}

\seNewAcronymGlossaryEntry{dm}{DM}{Data Mining}%
{}%
{Es beschreibt einen Prozess des \glqq Sammelns, Säuberns, Verarbeitens und Analysierens von Daten, zur Gewinnung von nützlichen Informationen.\grqq\seFootcite{}{S.1}{Aggarwal.2015} Mit Hilfe von bereitgestellten Methoden und Werkzeugen, welche auf die Daten angewandt werden, können Muster aus großen Datenmengen erkannt und somit neues Wissen generiert werden.}

\seNewAcronymGlossaryEntry{iot}{IoT}{Internet of Things}%
{}%
{Eines der jüngsten und wichtigsten Forschungsgebiete der Informatik ist das Internet der Dinge (\textit{Internet of Thing}). Es vernetzt alltäglich Gegenstände über das Internet, sodass diese intelligent miteinander kommunizieren können, um den Menschen bei seinen Tätigkeiten zu unterstützen.\seFootcite{Vgl.}{S. 1}{Chaouchi.2013}}

\seNewAcronymGlossaryEntry{crisp-dm}{CRISP-DM}{Cross Industry Standard Process for Data Mining}%
{}%
{Hierbei handelt es sich um eine industrieorientierte Darstellung des Data Mining Prozesses, welcher in die folgenden sechs Phasen eingeteilt ist: Verstehen der Aufgabe, Verständnis der Daten, Datenvorverarbeitung, Modellbildung, Evaluation und Einsatz im Unternehmen.\seFootcite{Vgl.}{S. 6-8}{Cleve.2014}}

\seNewAcronymGlossaryEntry{kdd}{KDD}{Knowledge Discovery in Data}%
{}%
{Hierbei handelt es sich um eine mögliche Gestaltung des Data Mining Prozesses. Dabei sind folgende fünf Phasen vorgesehen: Datenselektion, Datenvorverarbeitung, Datentransformation, Data Mining, sowie die Interpretation und Evaluation.\seFootcite{Vgl.}{S. 3}{Runkler.2015}}

\seNewAcronymEntry{svm}{SVM}{Suppport Vector Machine}{}

\seNewAcronymEntry{nn}{NN}{Neuronale Netze}{}

\seNewAcronymGlossaryEntry{mdkq}{MDKQ}{Methode der kleinsten Quadrate}%
{}%
{Das Ziel der Regression ist die Bestimmung der Parameter der Funktion mit Hilfe der Methode der kleinsten Quadrate. Diese versucht, die quadratischen Abstände vom Funktionsgraphen $\hat{f}$ zu den Daten zu minimieren ($\min RSS = \sum_{\forall i} (y_i - \hat{y}(x_i))^2$. Im einfachsten Falle liegt eine \textit{lineare} Regressionsfunktion der Form $\hat{f}(x) = \hat{\alpha} \cdot x + \hat{\beta}$ vor, wobei die Parameter $\hat{\alpha}$ und $\hat{\beta}$ wie folgt bestimmt werden können:\seFootcite{Vgl.}{S. 36 ff}{Studenmund.2014}\begin{equation}
	\hat{\alpha} = \frac{\sum{\forall i} x_i y_i - m \overline{x} \overline{y}}{\sum{\forall i} x^2_i - m \overline{x}^2}
\end{equation} \begin{equation}
	\hat{\beta} = \overline{y} - \hat{\alpha} \overline{x}^2
\end{equation}   }

\seNewGlossaryEntry{bigdata}{Big Data}{}{Hierbei handelt sich um eine Datenmenge, welche die Leistungsfähigkeit herkömmlicher Analysewerkzeuge übersteigt. Die Definition des Begriffes erfolgt häufig anhand der \glqq vier V\grqq.\seFootcite{Vgl.}{S. 10-12}{Freiknecht.2014} \begin{itemize}\item \textbf{Volume:} Es wird gerechnet, dass der weltweit verfügbare Datenbestand sich alle zwei Jahre verdoppelt.
\item \textbf{Velocity:} Eine große Datenmenge entsteht innerhalb kurzer Zeit und muss demzufolge schnell verarbeitet und analysiert werden.
\item \textbf{Variety:} Daten sind sich ähnlich, besitzen jedoch keine einheitliche Struktur und müssen aufgrund geringer Qualität aufbereitet werden.
\item \textbf{Veracity:} In der heutigen Informationsgesellschaft muss die Vertrauenswürdigkeit der Daten immer wieder überprüft werden.\end{itemize}}

\seNewGlossaryEntry{clustering}{Clustering}{}{Das Clustering ist einer der Grundbausteine des Data Mining, welche das Ziel der Aufteilung der Daten in verschieden Gruppen verfolgt, wobei diese Gruppen, auch \textit{Cluster} genannt, zuvor nicht bekannt sind. Die Elemente eines solchen Clusters sollen nach Innen möglichst homogen sein, sich jedoch nach Außen von anderen Clustern deutlich unterscheiden.\seFootcite{Vgl.}{S. 3}{Anderberg.2014}}

\seNewGlossaryEntry{regression}{Regression}{}{
Um Daten durch eine Funktion approximieren zu können, wird die Regression verwendet. Im Allgemeinen beinhaltet diese die Analyse einer abhängigen Variablen von einer oder mehreren unabhängigen Variablen, die durch eine Regressionsgleichung ausgedrückt werden. Die quadratischen Abstände zur Funktion werden dabei durch die \textit{Methode der kleinsten Quadrate} minimiert, sodass eine möglichst genaue Anpassung der Funktion an die Daten erfolgt. In der Praxis haben sich eine Vielzahl verschiedener Regressionsmodelle etabliert, die je nach Anwendungsfall ihre Verwendung finden: \textit{lineare} Regression, \textit{nichtlineare} Regression, \textit{multiple} Regression, sowie die \textit{nichtparametrische} Regression.\seFootcite{Vgl.}{S. 68-94}{Gunther.2014}}

\seNewGlossaryEntry{outlier}{Outlier Detection}{}{Definition folgt}

\seNewAcronymEntry{ki}{KI}{Künstliche Intelligenz}{Künstlichen Intelligenz}



% 2012-03-24
% \"Uber den optionalen Parameter in eckigen Klammern wird die Pluralform f\"ur das erste 
% Auftreten der Abk\"urzung definiert

\seNewAcronymEntry[URLs]{url}{URL}{Uniform Resource Locator}%
{Uniform Resource Locators}


% Definition von Symbolen
%
% 1. Parameter: Schluessel (key) des Symbols
% 2. Parameter: Symbol
% 3. Parameter: Text, der die Sortierreihenfolge festlegt (optional; falls nicht definiert, wird der Wert des zweiten 
%                        Parameters verwendet)
% 4. Parameter: Beschreibung des Symbols
%

\seNewSymbolEntry{ND}{ND}{a}{Nutzungsdauer einer Maschine}

\seNewSymbolEntry{pi}{$\pi$}{b}{Die Kreiszahl}




% Definition von Glossareintraegen
%
% 1. Parameter: Schluessel (key) des Glossareintrags
% 2. Parameter: Begriff, der im Glossar definiert wird
% 3. Parameter: Pluralform des Begriffes (optional; falls nicht definiert, wird der Wert des zweiten Parameters verwendet)
%                        Achtung: Pluralform gilt nur fuer das erste Auftreten des Begriffes im Text
% 4. Parameter: Beschreibung des Glossareintrags
%
%
%

\seNewGlossaryEntry{glos:AD}{Active Directory}{Active Directories}
{Active Directory ist in einem Windows 2000/Windows
Server 2003-Netzwerk der Verzeichnisdienst, der die zentrale
Organisation und Verwaltung aller Netzwerkressourcen erlaubt. Es
erm\"oglicht den Benutzern \"uber eine einzige zentrale Anmeldung den
Zugriff auf alle Ressourcen und den Administratoren die zentral
organisierte Verwaltung, transparent von der Netzwerktopologie und
den eingesetzten Netzwerkprotokollen. Das daf\"ur ben\"otigte
Betriebssystem ist entweder Windows 2000 Server oder
Windows Server 2003, welches auf dem zentralen
Dom\"anencontroller installiert wird. Dieser h\"alt alle Daten des
Active Directory vor, wie z.\,B. Benutzernamen und
Kennw\"orter.\protect\footnote{Bedauerlicherweise wei{\ss} der Autor dieses Dokumentes nicht mehr, woher diese Information stammt -- das 
geht in einer richtigen wissenschaftlichen Arbeit nat\"urlich \"uberhaupt nicht!!!}}
%\protect\seFootcite{Vgl.}{S. 200}{Dud09}}


\seNewGlossaryEntry{glos:bs}{Betriebssystem}{Betriebssysteme}{Die Begriffsdefinition sollten Sie eigentlich kennen!}



% Definition von Glossareintraegen, die gleichzeitig im Abk�rzungsverzeichnis auftreten
%
% 1. Parameter: Schluessel (key) des Glossareintrags
% 2. Parameter: Abk\"urzung
% 3. Parameter: Vollform
% 4. Parameter: Vollform im Plural (optional; falls nicht definiert, wird der Wert des dritten Parameters verwendet)
% 5. Parameter: Beschreibung des Glossareintrags

\seNewAcronymGlossaryEntry{glos:ma}{MA}{Mobile Applikation}{Mobile Applikationen}
{Eine Applikation, die auf einem mobilen Endger\"at ausgef\"uhrt wird.}

% 2012-03-24
% \"Uber den optionalen Parameter in eckigen Klammern wird die Pluralform f\"ur die Abk\"urzung definiert

\seNewAcronymGlossaryEntry[TAen]{glos:ta}{TA}{Transaktion}%
{Transaktionen}%
{Was eine Transaktion ist, sollten Sie ebenfalls bereits wissen!}





% Alternative Definition von Abk\"urzungen; diese sollten nicht verwendet werden!!!
%
%\newacronym{dhbw}{DHBW}{Duale Hochschule Baden-W\"urttemberg}
%\newacronym{usb}{USB}{Universal Serial Bus}


% Alternative Definition von Symbolen
%
% Achtung: ohne sort wird nach Name sortiert
%\newglossaryentry{pi}{
%name=$\pi$,
%description={Die Kreiszahl},
%type=symbolslist,
%sort=b
%}
%
%\newglossaryentry{ND}{
%name=$\mbox{\textsl{ND}}$,
%description={Nutzungsdauer einer Maschine},
%type=symbolslist,%
%sort=a
%}



% Alternative Definition von Glossareintr\"agen
%
%\newglossaryentry{glos:AD}{
%first=Active Directory\textsuperscript{GL},
%name=Active Directory,
%description={Active Directory ist in einem Windows 2000/Windows
%Server 2003-Netzwerk der Verzeichnisdienst, der die zentrale
%Organisation und Verwaltung aller Netzwerkressourcen erlaubt. Es
%erm\"oglicht den Benutzern \"uber eine einzige zentrale Anmeldung den
%Zugriff auf alle Ressourcen und den Administratoren die zentral
%organisierte Verwaltung, transparent von der Netzwerktopologie und
%den eingesetzten Netzwerkprotokollen. Das daf\"ur ben\"otigte
%Betriebssystem ist entweder Windows 2000 Server oder
%Windows Server 2003, welches auf dem zentralen
%Dom\"anencontroller installiert wird. Dieser h\"alt alle Daten des
%Active Directory vor, wie z.\,B. Benutzernamen und
%Kennw\"orter.\protect\seFootcite{Vgl.}{S. 200}{Dud09}}
%}














 

%\seIstErsteProjektarbeit{}
%\seIstZweiteProjektarbeit{}
\seIstBachelorarbeit{}

\newcommand{\version}{0.98}

% 
% Diese Redefinition ist nur f\"ur den Anhang B der  
% Vorlage (Hinweise zur Installation und \"Ubersetzung)
% notwendig; f\"ur Ihre Bachelorarbeit spielt sie keine Rolle
%
\renewcommand{\seVorlage}{\jobname}
\usepackage{color}
\usepackage{pgfplots}
\usepackage{amssymb} 
\usetikzlibrary{decorations.pathreplacing}
\usepackage{rotating}
\usepackage{pifont}
\usepackage{pdflscape}
\usepackage{pdfpages} 
\usepackage{tabularx}
\usepackage{tabularx}
\usepackage{array}
\newcolumntype{P}[1]{>{\centering\arraybackslash}p{#1}}


\begin{document}

% Erzeugung des Titelblatts
%
%
%
\seTitelblattWissenschaftlicheArbeit[
%hilfslinien=ja,
%dhbwlogoSkalierung=0.5,
%dhbwlogoDeltaX=2.4,
%dhbwlogoDeltaY=-10,
firmenlogo=firmenlogo,
firmenlogoSkalierung=0.175,
firmenlogoDeltaX=0,
firmenlogoDeltaY=-6,
studiengang=\seWirtschaftsinformatik,
%studienrichtung=\seApplicationManagement,
%studienrichtung=\seSalesUndConsulting,
studienrichtung=\seSoftwareEngineering,
thema=Modellierung einer Funktion zur Berechnung der
Wahrscheinlichkeit eines Torerfolges im Fußball,
verfasser=Alexander Baum,
%verfasserin=Melanie Musterfrau,
matrikelnummer=8095497,
kurs=WWI\,14\,SE\,A,
firma=SAP SE,
% Da im Text ein Komma enthalten ist, muss der Text eingeklammert werden
%abteilung={Wirtschaftsinformatik, Sales \& Consulting},
abteilung={SAP Sports},
%studiengangsleiterin=,
studiengangsleiter=Prof. Dr.-Ing. J\"org Baumgart,
%studiengangsleiter=Prof. Dr. Thomas Holey,
wissenschaftlicheBetreuerinName=Susanne Klusmann,
wissenschaftlicheBetreuerinEmail=susanne.klusmann@f-i.de,
wissenschaftlicheBetreuerinTelefon=+49 511 5102-22137,
%wissenschaftlicherBetreuerName=Prof. Dr.-Ing. J\"org Baumgart,
%wissenschaftlicherBetreuerEmail=joerg.baumgart@dhbw-mannheim.de,
%wissenschaftlicherBetreuerTelefon=0621/4105\,1216,
firmenbetreuerName=Dr. Andrew McCormick-Smith,
firmenbetreuerEmail=andrew.mccorcmick-smith@sap.com,
firmenbetreuerTelefon=+49 6227 7-41565,
%firmenbetreuerName=,
%firmenbetreuerEmail=,
%firmenbetreuerTelefon=,
bearbeitungszeitraumVon=21. November 2016,
bearbeitungszeitraumBis=20. Februar 2017,
%
% Datum in englischer Schreibweise
%bearbeitungszeitraumVon=28 November  2016,
%bearbeitungszeitraumBis=19 February 2017,
%sperrvermerk=ja
]

% Erzeugung der Kurzfassung; Verfasser, Firma und Thema werden automatisch \"ubernommen
%
% Der optionale Parameter kann verwendet werden, um f\"ur das Thema der Arbeit eine 
% andere Formatierung vorzunehmen; das sollte in der Regel nicht erforderlich sein;
% ausserdem besteht die Gefahr inkonsistenter Titel auf dem Titelblatt und in der 
% Kurzfassung
%
\seKurzfassung{} % dieses Kommando sollte standardm\"assig verwendet werden

%\seKurzfassung[\LaTeX-Vorlage zur Anfertigung \seThemaWaArbeit{} (Version \version{})]

Die heutige technisierte Massenerhebung an Daten im Fußballsport bietet neueste Möglichkeiten der Datenanalyse, wobei die Leistungsfähigkeit herkömmlicher Analysewerkzeuge bei Weitem überstiegen wird. Das Ziel dieser Bachelorarbeit ist es, eines der momentan angesagtesten Statistikmodelle, das \textit{Expected-Goals-Modell}, durch eine Funktion zur Berechnung der Wahrscheinlichkeit eines Torerfolges im Fußball aus jeder möglichen Position des Spielfeldes zu modellieren. Durch die Extraktion dieses neuen und fundierten Wissens können Trainer, Spielanalysten und Scouts profitieren.

Im Rahmen des systematisch angewendeten \textit{Knowledge Discovery in Data} Prozesses werden dem Leser die Grundlagen wie auch die Methoden des Data Minings näher gebracht sowie die einzelnen Prozessschritte und deren beinhaltenden Methoden zur Datenaufbereitung vorgestellt. Anhand dieser Methodik werden die zugrundeliegenden Daten auf Basis der aufgestellten Anforderungen selektiert, vorverarbeitet und transformiert, um diese durch die Verwendung unterschiedlicher Regressionsmodelle zu einer Funktion zu modellieren. Diese Resultate werden anschließend sowohl interpretiert als auch evaluiert, wobei sich die \textit{nichtparametrische} Regression unter der Betrachtung der Koordinaten des Schusses als am geeignetsten herausstellt hat. Abschließend wird ein Ausblick über mögliche Anwendungsfälle des mathematischen Modelles im Bereich des Datenscoutings im Fußballsport gegeben.

% 2012-02-06 Inhaltsverzeichnis muss vor den weiteren Verzeichnisses kommen
%
%
% Ausgabe des Inhaltsverzeichnisses
%
%
\seInhaltsverzeichnis[%
einrueckung=ja,
gliederungsebenen=4
]


% Ausgabe der verschiedenen Verzeichnisse
% abk: Abk\"urzungsverzeichnis
% sym: Symbolverzeichnis
% abb: Abbildungsverzeichnis
% tab: Tabellenverzeichnis
% prg: Listingverzeichnis
% alg: Algorithmenverzeichnis
%
%
% Achtung: Abk\"urzungs- und Symbolverzeichnis werden nur ausgegeben, wenn mindest ein Symbol bzw. 
%                mindestens eine Abk\"urzung in der Arbeit verwendet wurden
%
%
% gliederungsebene:
% -- section: die Verzeichnisse werden einem Kapitel "Verzeichnisse" untergliedert
% -- chapter: die Verzeichnisse sind jeweils eigene Kapitel
% imInhaltsverzeichnis: ja/nein -- Sollen die Verzeichnisse im Inhaltsverzeichnis enthalten sein?
\seVerzeichnisse[gliederungsebene=section,imInhaltsverzeichnis=ja]{abk}{}{abb}{tab}{prg}{}



%Hauptkapitel der Arbeit
\chapter{Einleitung}\pagenumbering{arabic}

\section{Ziel}

\begin{quote} 
\glqq Expected Goals: Das angesagteste Statistikmodell im Profifußball.\grqq\seFootcite{}{}{NilsNordmann.2016}
\end{quote}

Im heutigen kommerziellen Fußballsport verwenden Experten immer wieder die Worte \glqq Geld schießt Tore\grqq, jedoch können das Daten auch. Modewörter aus der Informatik wie \gls{dm}, \gls{bigdata} oder \gls{ml} haben sich inzwischen auch im Bereich des Fußballs etabliert. Durch die Massenerhebung von Spieldaten, wie beispielsweise die Erfassung der Positionsdaten aller Spieler und des Balles oder die Aufzeichnung aller Aktionen eines Spiels von Pässen, über Schüssen, bis hin zu Fouls, entsteht eine rohe Datenflut an Informationen, die die Leistungsfähigkeit herkömmlicher Analysewerkzeuge bei Weitem übersteigt. Der Mensch ist in seiner Wahrnehmung limitiert, ein Computer jedoch nicht. Mit den richtigen Methoden und Werkzeugen lässt sich Wissen aus dieser überdimensionalen Datenmenge extrahieren, wodurch die Analyse im Fußball revolutioniert wird. Spitzenclubs wie Manchester City oder der FC Bayern München leisten sich ganze Teams von Datenspezialisten, welche die Daten interpretieren und auf
das Spielgeschehen übertragen. Die Datenanalyse kann und soll dabei nicht das fußballerische Expertenwissen der Trainer oder Scouts ersetzen, sondern viel mehr die subjektiven Wahrnehmungen validieren. Konkret behandelt diese Arbeit eines der momentan angesagtesten Stastikmodelle im Fußball, die \textit{Expected Goals}, ein Modell das die Wahrscheinlichkeit eines Schusses für einen Torerfolg aus jeder möglichen Position ermittelt. Dazu existieren bereits zahlreiche unterschiedliche Ansätze, jedoch wurde bislang keine Funktion zur Berechnung der Wahrscheinlichkeit modelliert. An dieser Stelle greift die Intention dieser Arbeit, die das Ziel verfolgt mit der Modellierung einer Funktion zur Berechnung der Wahrscheinlichkeit eines Torerfolges im Fußball, neues und fundiertes Wissen zu schaffen, von dem Trainer, Spielanalysten und Scouts profitieren. 
	
\section{Umgebung}
\paragraph{Unternehmen}
Die SAP\footnote{\gls{sap}} wurde 1972 von fünf ehemaligen IBM Mitarbeitern gegründet und ist seit mehr als 40 Jahren, hinsichtlich des Marktanteils mit über 282.000 Kunden, das weltweit führende Unternehmen für Anwendung- und Analysesoftware. Der im baden-württembergischen Walldorf gegründete Aktienkonzern bietet mit dem bis heute bekanntesten Produkt \textit{SAP ERP} eine Softwarelösung zur Abbildung aller Geschäfts- und Produktionsprozesse in einem Unternehmen von Personal- und Rechnungswesen bis hin zur Logistik. Mit dem heutigen Stand der Entwicklung setzt die SAP ihren Fokus verstärkt auf die Bereiche Cloud, Mobile und Internet of Things, um mit anderen Unternehmen konkurrieren zu können und den Anschluss an den Trend der Zeit nicht zu verlieren. Die SAP beschäftigt in über 180 Ländern mehr als 77.000 Mitarbeiter und erzielte im Jahr 2015 einen Umsatz von 20,8 Mrd Milliarden Euro, sowie ein Betriebsergebnis von 6,3 Milliarden Euro.\footnote{Zahlen vor Abzug. der Steuern\\ Weitere Information zum Geschäftsbericht der SAP SE aus dem Jahr 2015 unter: \\ http://www.sap.com/docs/download/investors/2015/sap-2015-geschaeftsbericht.pdf [10.01.2017]}

\paragraph{Abteilung}
Die Praxisphase erfolgte in der Abteilung \textit{Sports \& Entertainment}, die sich von den klassischen SAP Geschäftsbereichen isoliert hat und alles rund um den Sport betreut. Im Bereich des Fußballs liegt der Fokus einerseits auf der Organisation des gesamten Vereins inklusive Umfeld, sprich Management, Marketing, Mannschaft, Jugend oder auch Fans, andererseits auch auf der Spielanalyse mit Hilfe von erhobenen Daten. Dazu steht die Abteilung in regelmäßigen Kontakt mit dem Bundesligaverein der TSG 1899 Hoffenheim, sowie der deutschen Nationalmannschaft, um ständig neue Anwendungsfälle zu gewinnen. Alle Funktionalitäten sollen in einem Produkt, dem sogenannten \textit{Sports One} vereint werden, welches aus verschiedenen Rollen, wie Spieler, Trainer oder auch Mannschaftsarzt verwendet werden kann. Im Bereich der Spielanalyse und des Scoutings werden Unmengen an Daten gesammelt, die es für den späteren Anwender zu analysieren gilt. Hier findet sich der in dieser Arbeit beschriebene Anwendungsfall wider, mit dessen Unterstützung eine Funktion für die Berechnung der Wahrscheinlichkeit eines Torerfolges modelliert werden soll.
\section{Vorgehen}

Neben dem allgemeinen Ziel der Modellierung der Funktion zur Berechnung der Wahrscheinlichkeit eines Torerfolges im Fußball, ergeben sich weitere wissenschaftliche Teilfragen, die es innerhalb dieser Arbeit zu beantworten gilt. Zunächst muss festgestellt werden, welche Datenmenge zugrunde liegt, welche Daten daraus selektiert werden sollen und ob diese gegebenenfalls aufbereitet oder bereinigt werden müssen. Anschließend muss ein Analysewerkzeug gewählt werden, welches durch die Unterstützung eines Softwaretools eine Funktion aus den Daten modelliert. Letztlich gilt es den Erfolg der resultierenden Modellierung zu messen, um die Validität sicherzustellen. Als grundlegende Methodik wird dazu der allgemeingültige Knowledge Discovery in Data Prozess für Data-Mining-Projekte verwendet, welcher die genannten wissenschaftlichen Teilfragen dieser Arbeit systematisch beantwortet. Zunächst werden dazu die theoretischen Bestandteile des Data Minings studiert (siehe \vref{dm}), die einzelnen Phasen und Methoden des Knowledge Discovery in Data Prozesses eruiert (siehe \vref{kdd}), sowie die mathematischen Grundlagen der Funktionsmodellierung mittels der Regressionsanalyse und des Softwaretools MATLAB erläutert (siehe \vref{fm}). Des Weiteren wird der heutige Stand und die Bedeutung der \textit{Expected Goals} näher beleuchtet (siehe \vref{goals}) und die vorhandenen Spieldaten analysiert, um daraus die Schussversuche selektieren zu können (siehe \vref{opta}). Eine wirtschaftliche Betrachtung des Modelles aus Sicht der SAP als auch aus Sicht einen Vereines runden die Analysephase ab (siehe \vref{wa}). Die im theoretischen Teil vorgestellten Methoden der Prozessphasen werden innerhalb der Umsetzung angewendet (siehe \vref{umsetzung}), um eine fundierte Modellierung der Funktion zu gewährleisten, die abschließend anhand einiger Verfahren evaluiert wird (siehe \vref{iue}). So kann der Leser die Arbeit systematisch nachvollziehen und sich entlang des Prozesses hangeln.

Die Funktion wird dabei einerseits unter der Betrachtung der Koordinaten des Schussversuches, als auch in Bezug der Distanz und des Winkels des Schussversuches zum Tor modelliert. Dabei wurden von den internen Fachexperten der Abteilung folgende Anforderung für die Funktion (siehe \vref{tab:anf}) aufgestellt, die es für die Modellierung zu berücksichtigen gilt. Anforderung \textsf{1} und \textsf{2} beschreiben die In- und Output-Variablen der Funktion. Die Parameter, die die Wahrscheinlichkeit zwischen 0 und 1 bestimmen, sind lediglich die Koordinaten des Schussversuches in Form eines $x$- und $y$-Wertes. Wichtig ist, dass lediglich Schussversuche berücksichtigt werden dürfen, die während des \glqq laufenden\grqq~Spiels abgegeben wurden, da die Wahrscheinlichkeit für einen Torerfolg aus einer Standardsituation, wie die eines Elfmeter oder direkten Freistoßes, separat betrachtet werden muss (Anforderung \textsf{3}).\footnote{Bei einem Elfmeter gibt es beispielsweise keine Einwirkung von gegnerischen Spielern.} Die Anforderung \textsf{4} schließt Eigentore von der Modellierung aus, da diese die Wahrscheinlichkeit eines Schussversuches nicht beeinflussen und damit die Modellierung verzerren würden. Diese Anforderung war jedoch nicht von vornherein vorgegeben und konnte erst innerhalb des iterativen Prozesses als Problematik erkannt werden. Anlässlich der Tatsache, dass keine konkrete Aussage getroffen werden kann, ob ein geblockter Schuss sonst in einem Torerfolg resultiert hätte, bleiben solche Schussversuche ebenfalls unberücksichtigt (Anforderung \textsf{5}). Durch die Anforderung \textsf{6} und \textsf{7} wird die Form der Funktion bestimmt, um eine eventuelle zufällige Verteilung der Schussversuche auszugleichen und die Berechnung des Winkels und der Distanz des Schusses zum Tor zu vereinfachen.

%%%%%%%%%%%%%%%%%% Anforderungen %%%%%%%%%%%%%%%%%%
\tablefirsthead{\hline\multicolumn{3}{|c|}{\textbf{Anforderungen}}\\\hline\hline \textbf{Nr.} & \textbf{Beschreibung} & \textbf{vorgegeben}\\\hline}
\tablelasttail{}
\bottomcaption{Auflistung der Anforderungen\label{tab:anf}}
\begin{center}%
\begin{supertabular}{ | P{1.5cm} | P{8cm} | P{2.5cm} |}
\vspace*{1mm}\textbf{1.} 	& \vspace*{1mm}Input-Variablen der Funktion sind die Koordinaten eines Schusses& \vspace*{1mm}	\textit{ja}\\
\hline
\vspace*{1mm}\textbf{2.}	& \vspace*{1mm}Output der Funktion ist die Wahrscheinlichkeit zwischen 0 und 1	& \vspace*{1mm}\textit{ja} 	\\
\hline
\vspace*{1mm}\textbf{3.}	& \vspace*{1mm}Es dürfen nur Schüsse berücksichtigt werden, die während des \glqq laufenden\grqq~Spiels abgegeben wurden  	& \vspace*{1mm}\textit{ja}  	\\
\hline
\vspace*{1mm}\textbf{4.}	& \vspace*{1mm}Eigentore müssen ausgeschlossen werden  	& \vspace*{1mm}\textit{nein}  	\\
\hline
\vspace*{1mm}\textbf{5.}	& \vspace*{1mm}Geblockte Schüsse müssen ausgeschlossen werden\	& \vspace*{1mm}\textit{ja}  	\\
\hline
\vspace*{1mm}\textbf{6.}	& \vspace*{1mm}Der Ursprung der Funktion liegt in der Mitte der gegnerischen Torlinie & \vspace*{1mm}\textit{ja}  	\\
\hline
\vspace*{1mm}\textbf{7.}	& \vspace*{1mm}Die Funktion muss symmetrisch zu beiden Spielhälften sein (Spiegelung an der gedachten Linie zwischen der generischen und der eigenen Tormitte) & \vspace*{1mm}\textit{ja}  	\\
\hline
\end{supertabular}
\end{center}



\chapter{Theoretische Grundlagen}

\section{Data Mining}
\label{dm}
Die vorliegende wissenschaftliche Fragestellung bewegt sich im Bereich des Data Minings. Das folgende Kapitel soll dem Leser dazu eine Einführung in die Thematik geben, um ein Verständnis der grundlegenden Begrifflichkeiten und Ziele des Data Minings zu erlangen (vgl. \vref{defdm}). Darüber hinaus werden die Prozesse des Data Minings (vgl. \vref{prozdm}) beleuchtet, wobei der \textit{Knowledge Discovery in Data} Prozess -- methodischer Aufbau der späteren Umsetzung -- in \vref{kdd} nochmal ausführlich beleuchtet wird.

\subsection{Definition des Data Minings}
\label{defdm}

Der Begriff des Data Minings reicht zurück bis in die 80er Jahre des letzten Jahrhunderts und verfolgt das Ziel, Wissen aus riesigen Datenmengen zu extrahieren.\seFootcite{Vgl.}{S. 2}{Runkler.2015} Es ist ein Prozess des \glqq  \textit{Sammelns, Säuberns, Verarbeitens und Analysierens von Daten, zur Gewinnung von nützlichen Informationen.}\grqq\seFootcite{}{S. 1}{Aggarwal.2015} Der weltweit gesammelte Datenbestand steigt immer stärker an und stellt Analysten vor die Herausforderung, aus dieser Datenflut wertvolle Informationen und organisiertes Wissen zu extrahieren. Erst das heutige \glqq Informationszeitalter\grqq~führte zum Beginn des renommierten Wissenschaftsbereich des Data Minings, welcher in der Literatur auch als natürliche Evolution der Informationstechnologie bezeichnet wird.\seFootcite{Vgl.}{S. 1}{Garcia.2015}\seFootcite{Vgl.}{S. 2}{Han.2012} Grundlegende interdisziplinäre, wissenschaftliche Teilgebiete des Data Minings sind, z.B. die Statistik, das maschinelles Lernen (\gls{ml}), die Mustererkennung, die Systemtheorie oder die \gls{ki}.\seFootcite{Vgl.}{S. 2}{Runkler.2015}\seFootcite{Vgl.}{S. 1}{Shi.2015}


Cleve und Han vergleichen die Suche nach Mustern und Zusammenhängen in den Daten mit dem Abbau von Rohstoffen.\footnote{Die englische Übersetzung lautet \textit{\glqq Mining\grqq}} Sowie im Bergbau nach Schätzen wie Gold und Silber im Gestein gesucht wird, so strebt das \gls{dm} nach dem Ableiten von Wissen aus den (Roh-)Daten.\seFootcite{Vgl.}{S. 1}{Cleve.2014}\seFootcite{Vgl.}{S. 5-6}{Han.2012} Han geht sogar einen Schritt weiter und präferiert den Begriff des \textit{Knowledge Mining from Data} -- bezogen auf den verwendenden Terminus des \textit{Gold Mining}, statt des \textit{Rock or Sand Mining} -- da diese Bezeichnung das eigentliche Ziel der Gewinnung von Wissen beinhaltet.\seFootcite{Vgl.}{S. 5-6}{Han.2012}\footnote{Weitere Termini nach Han: \textit{knowledge mining from data, knowledge extraction, data/pattern analysis, data archaeology, and data dredging.}}

\glqq Unter Wissen verstehen wir interessante Muster, die allgemein gültig sind, nicht trivial, neu, nützlich und verständlich.\grqq\seFootcite{}{S. 2}{Runkler.2015} Insofern wird das Ziel verfolgt, komplexe Paradigmen zu erkennen, die durch bloße Betrachtung der Daten nicht aufgedeckt werden können. Oftmals fehlt dem Datenanalyst das spezifische Fachwissen zur Erkennung von Mustern, sodass durch die Einbeziehung von Experten ein iterativer Prozess entsteht, bis ein gewünschtes Ergebnis erzielt wurde. Zunächst werden aus den Daten, Informationen gewonnen, aus welchen wiederum das Wissen abgeleitet werden kann, wobei in diesem Prozess der Wissensextraktion die Datenmenge sukzessive abnimmt und sich verdichtet, wie in \vref{wissenprozess} verdeutlicht.

\begin{figure}[H]
\centering
\includegraphics[scale=1.3]{se-wa-jpg/wissenprozess}
\caption[Wissensextraktion aus Daten]{Wissensextraktion aus Daten\protect\footnotemark}
\label{wissenprozess}
\end{figure}
\footnotetext{Vgl. Abbildung \textit{Shi} et al., Intelligent Knowledge, 2015, S. 5.}

\textit{Daten} stellen dabei nur eine Reihe von Zeichen da, wobei die Bedeutung zunächst unklar ist. Erst wenn bekannt wird, in welchem Kontext die Daten stehen und welche Beziehungen zwischen diesen existiert, können diese interpretiert und zu einer (relevanten) \textit{Information} werden. Das \textit{Wissen} entsteht letztlich durch die Verknüpfung von vielen Daten und Informationen mit zusätzlichen Erfahrungen.\seFootcite{Vgl.}{S. 16-18}{Shi.2015}

Durch den Einsatz von modernster Computerhard- als auch software ist es möglich, immens große Datenmengen zu erheben, zu verarbeiten und zu analysieren, wodurch in diesem Kontext der Begriff \gls{bigdata} entstanden ist.\seFootcite{Vgl.}{S. 3}{Witten.2011} \gls{bigdata} bezeichnet Datenmengen, die mit herkömmlichen Analysemethoden nicht mehr zu verarbeiten wären und deshalb die Anwendung von Data Mining benötigen.\seFootcite{Vgl.}{S. 5}{Fasel.2016}\seFootcite{Vgl.}{S. 1}{Shi.2015} Dazu ein paar ausgewählte Bespiele aus verschiedenen Datenbereichsquellen:\seFootcite{Vgl.}{S. 2}{Aggarwal.2015}

\begin{itemize}
\item \textbf{Word Wide Web:} Die Anzahl der Dokumente im Internet hat schon lange die Milliarden Marke geknackt, wobei die des unsichtbaren \glqq Webs\grqq noch viel größer ist. Durch Nutzerzugriffe auf Inhalte, werden auf Serverseite Log-Dateien kreiert, um beispielsweise die Auslastung und Zugangszeiten zu protokollieren. Andererseits wird das Kundenverhalten auf kommerziellen Seiten aufgezeichnet, um personalisierte Werbung schalten zu können.

\item \textbf{Benutzerinteraktion:} Festnetzanbieter nutzen die durch Telefonate entstandenen Daten wie Gesprächslänge und Ort, um relevante Muster über die Netzwerkauslastung, zielgerichtete Werbung oder auch anzusetzende Preise durch Datenanalyse zu extrahieren.
\item \textbf{Internet of Things:} Durch kostengünstige (tragbare) Sensoren und deren kommunikative Vernetzung, entstand das \gls{iot}. Einer der Trends der heutigen Informationstechnologie, welcher durch die Erhebung von Massendaten eine signifikante Rolle für das Data Mining einnimmt.

\item \textbf{Weitere Beispiele:}  Social Media Plattformen (allen voran Facebook, Twitter und Co.), Finanzmärkte (z.B. der Aktienmarkt), Sport (z.B. Baseball, Basketball, Football oder wie in dieser Arbeit Fußball), uvm.\seFootcite{Vgl.}{S. 39}{Fayyad.1996}\seFootcite{Vgl.}{S. 1-2}{Han.2012}\seFootcite{Vgl.}{S. 85 ff}{Chu.2014} 
\end{itemize}

\glqq Wir befinden uns in einer Welt, in der wir reich an Daten sind, jedoch arm an Informationen und Wissen.\grqq\seFootcite{}{S. 5}{Han.2012} Der unglaublich rasante und gigantische Datenzuwachs hat bei weitem unsere menschliche Vorstellungskraft und Möglichkeiten übertroffen, sodass wir auf effiziente Werkzeuge angewiesen sind (siehe \vref{dmmethoden}). Die sich immer weiter ausbreitende Spalte zwischen Daten und Informationen, führt nur noch durch die Nutzung von Methoden des Data Minings zu den begehrten \glqq \textit{Golden Nuggets of Knowledge}\grqq.\seFootcite{Vgl.}{S. 5}{Han.2012} Dazu müssen die (Roh-)Daten gezielt ausgewählt und umstrukturiert werden, um diese anschließend durch Algorithmen analysieren zu können. Folglich entstanden Data Mining Prozesse, die dieses Problem mit Hilfe systematischer Abläufe lösen sollen (vgl. \vref{prozdm}). Zudem wird \glqq Data Mining [...] heute durch eine zunehmende Anzahl von Software-Tools unterstützt, z. B. KNIME, MATLAB, SPSS, SAS, STATISTICA, TIBCO Spotfire, R, Rapid Miner, Tableau, QlikView, oder WEKA.\grqq\seFootcite{Vgl.}{S. 3}{Runkler.2015} Das Software-Tool \textit{MatLab} wird innerhalb der Funktionsmodellierung in \vref{fm} vorgestellt und anschließend als Werkzeug zur Nutzung von Data Mining Methoden in der Umsetzungsphase genutzt (vgl. \vref{umsetzung}).


\subsection{Data Mining Prozesse}
\label{prozdm}

In der Literatur grenzen viele Wissenschaftler den Begriff des eigentlichen Data Minings, vom Gesamtprozess der Extraktion von Wissen ab. Andere wiederum behandeln beide Termini synonym zu einander.\seFootcite{Vgl.}{S. 39}{Fayyad.1996}\seFootcite{Vgl.}{S. 2}{Mariscal.2010}\seFootcite{Vgl.}{S. 1}{Garcia.2015} Schlechte Qualität der Daten mindert die Leistungsfähigkeit des Data Minings. Um die Aussagekraft der Daten nicht zu gefährden, sind vorab Prozessschritte notwendig, die Daten in adaptierter Form für die Methoden des Data Minings bereitstellen.\seFootcite{Vgl.}{S. 10}{Garcia.2015} Hierzu werden im Folgenden kurz die zwei bekanntesten Prozessmodelle vorgestellt:

\begin{itemize}
\item \gls{kdd}
\item \gls{crisp-dm}
\end{itemize}

\subsubsection{Knowledge Discovery in Data}
\label{dmkdd}

Der Begriff des \textit{Knowledge-Discovery-in-Data}-Prozesses wurde in den frühen 90-er Jahren geprägt und wird als \glqq nicht trivialer Prozess zur Identifizierung von gültigen, neuartigen, potentiell sinnvolle und letztlich verständlichen Muster in Daten\grqq\seFootcite{}{S. 41}{Fayyad.1996} definiert.\seFootcite{Vgl.}{S. 2}{Mariscal.2010} Erstmals wurde der Terminus von Gregory Piatetsky-Shapiro auf der \textit{International Joint Conference on Artificial Intelligence}, 1989 in Detroit (USA), der Öffentlichkeit präsentiert.\seFootcite{Vgl.}{S. 1}{Adhikari.2015} Der in \vref{kddpic} dargestellte iterative \gls{kdd}-Prozess nach Fayyad, beinhaltet folgende Schritte, wobei das \gls{dm} als ein eigener Prozessschritt ausgewiesen wird:\seFootcite{Vgl.}{S. 5}{Cleve.2014}

\begin{enumerate}

\item \textbf{Datenselektion}: Auswahl der geeigneten Datenmengen.
\item \textbf{Datenvorverarbeitung}: Behandlung fehlender oder problembehafteter Daten.
\item \textbf{Datentransformation}: Umwandlung in adäquate Datenformate.
\item \textbf{Data Mining}: Suche nach Muster.
\item \textbf{Interpretation und Evaluation}: Interpretation der Ergebnisse und Auswertung.

\end{enumerate}

Auf die einzelnen Prozessschritte und deren Methoden wird genauer in \vref{kdd} eingegangen. Die Abkürzung \textit{KDD} steht in der Literatur für unterschiedliche Bezeichnungen, wie zum Beispiel \textit{Knowledge Discovery in Databases}, \textit{Knowledge Discovery in Data Mining} oder \textit{Knowledge Discovery in Data Warehouses}.\seFootcite{Vgl.}{S. 26 ff}{OseiBryson.2015} Alle zielen dabei auf die Erforschung von Wissen aus Datenmengen ab, wodurch in dieser Arbeit die allgemeingültige Bezeichnung \textit{Knowledge Discovery in Data} verwendet wird.

\begin{figure}[H]
\centering
\includegraphics[scale=0.85]{se-wa-jpg/kdd}
\caption[Der Knowledge Discovery in Data Prozess]{Der Knowledge Discovery in Data Prozess\protect\footnotemark}
\label{kddpic}
\end{figure}
\footnotetext{Vgl. Abbildung \textit{Fayyad} et al., From Data Mining to Knowledge, 1996, S. 41.}


\subsubsection{CRISP-DM}
Das \gls{crisp-dm}-Modell wurde im Jahr 2000 durch ein Konsortium, bestehend aus mehreren Firmen, entwickelt. Beteiligt daran waren:\seFootcite{Vgl.}{S. 6}{Cleve.2014}\seFootcite{Vgl.}{S. 3}{Mariscal.2010}

\begin{itemize}
\item NRC Corporation,
\item Daimler AG,
\item SPSS,
\item Teradata und
\item OHRA.
\end{itemize}

Dieses Modell verfolgt das Ziel, einen standardisierten und branchenübergreifenden Data-Mining-Prozess zu definieren und das dadurch berechnete Modell zu validieren. Hierbei wird von einem Lebenszyklus mit sechs beinhaltenden Etappen ausgegangen, der in \vref{crisp} dargestellt werden.\seFootcite{Vgl.}{S. 6-8}{Cleve.2014} Im Folgenden werden dazu die einzelnen Schritte des Prozesses aus der Abbildung (nummeriert von Schritt 1 bis 6) kurz beschrieben.

\begin{figure}[H]
\centering
\includegraphics[scale=0.9]{se-wa-jpg/crisp}
\caption[CRISP-DM Prozess]{CRISP-DM Prozess\protect\footnotemark}
\label{crisp}
\end{figure}
\footnotetext{Vgl. Abbildung \textit{Mariscal} et al., A survey of data mining, 2010, S. 13}

\paragraph{1. Verstehen der Aufgabe}(\textit{Business understanding}): 
Hier steht das grundsätzliche Verständnis des Fachgebietes und der Aufgabe im Vordergrund. Die Ziele werden definiert, Ressourcen des Unternehmens ermittelt und die Ausgangssituation bestimmt. Weiterhin müssen Erfolgskriterien quantifiziert und Risiken eruiert werden, um eine Kostenplanung aufstellen zu können. 
\paragraph{2. Verständnis der Daten}(\textit{Data understanding}): 
Diese Phase beschäftigt sich mit den benötigten Daten zur Durchführung der Analyse. Daten werden gesammelt und beschrieben, um deren betriebliche Bedeutung zu verstehen.
\paragraph{3. Datenvorbereitung}(\textit{Data preparation}): 
Es gilt den Data-Mining-Prozess-Schritt vorzubereiten, wobei fehlerhafte und inkonsistente Daten korrigiert werden müssen, um diese schließlich in eine Datenstruktur transformieren zu können, die für die Methoden des Data Minings nutzbar sind.
\paragraph{4. Data Mining - Modellbildung}(\textit{Modeling}): 
In dieser Phase wird ein Modell mit Hilfe des Data Minings erstellt, welches durch ein iterativen Aufbau immer wieder verfeinert und verbessert wird. 
\paragraph{5. Evalution:}
Die erzielten Ergebnisse werden an den aus Phase 1 definierten Erfolgskriterien gemessen, um beispielsweise festzustellen, ob der wirtschaftliche Nutzen erzielt wurde.
\paragraph{6. Einsatz im Unternehmen}(\textit{Deployment}): 
Zuletzt gilt es den Einsatz der Resultate in das Unternehmen vorzubereiten und in das operative Geschäft zu integrieren.

Das Modell bezieht und orientiert sich, wie schon am Namen zu erkennen ist, stark an wirtschaftlichen Projekten und beschreibt \textit{Was} zu tun ist, jedoch nicht genau \textit{Wie}, sodass Projektteams innerhalb dieses Rahmens beginnen ihre eigenen Methoden zu verwenden.\seFootcite{Vgl.}{S. 4}{Mariscal.2010}

Im Vergleich zum \gls{kdd}-Modell nach Fayyad, sind die Phase 1 und 2 des \gls{crisp-dm}-Modells sehr stark projektabhängig und spiegeln die Sicht der Industrie auf das Projekt wider.\seFootcite{Vgl.}{S. 8}{Cleve.2014} Im Gegensatz dazu konzentriert sich der \gls{kdd}-Prozess auf die Datenbereitstellung und Analyse, sodass dieser als grundlegende Methodik für die spätere Umsetzung der wissenschaftlichen Aufgabenstellung herangezogen wird und genauer in \vref{kdd} beleuchtet wird.

Mariscal et al. diskutieren in ihrer Studie weitere zahlreiche Prozessmodelle zur Extraktion von Wissen aus riesigen Datenmengen, wobei die Kernelemente der Datenselektion, -vorverarbeitung und -tranformation, sowie der anschließende Schritt des eigentlichen Data Minings immer wieder aufzufinden sind.\seFootcite{Vgl. vorgestellte Modelle aus}{}{Mariscal.2010} Nicht zuletzt ist zu erwähnen, dass in der Literatur unterschiedliche Auffassungen zu dem Begriff des Data Minings existieren und dieser oftmals mit den Data Mining Prozessen synonym verwendet wird. Ein Hinweis darauf sind auch die weit über 500 wissenschaftliche Artikel zu dem Journal \textit{Data Mining and Knowledge Discovery} auf \textit{Springer Link}.



\section{Knowledge Discovery in Data}
\label{kdd}

Das folgende Kapitel beschreibt den \textit{Knowledge Discovery in Data} Prozess, der im vorherigen Kapitel (vgl. \vref{dmkdd}) als grundlegende Methodik der Arbeit ausgewählt wurde. Hierzu werden die einzelnen Prozessschritte der Datenselektion, der Datenvorverarbeitung, der Datentransformation, der Data-Mining-Methoden, sowie der Interpretation der Ergebnisse konkretisiert, um diese in der späteren Umsetzung der wissenschaftlichen Aufgabe anwenden zu können.

\glqq Experten [...] haben realisiert, dass eine große Anzahl an Datenquellen der Schlüssel zu bedeutsamen Wissen sein kann und das dieses Wissen in dem Entscheidungsfindungsprozess genutzt werden sollte. Eine einfache \gls{sql}-Abfrage oder \gls{olap} reichen für eine komplexe Datenanalyse oft nicht aus.\grqq\seFootcite{}{S. 1}{Adhikari.2015} Hier greift der in \vref{kddpic} dargestellte \gls{kdd}-Prozess, ein multiples iteratives Modell, in dem die einzelnen Schritte solange wiederholt und aufeinander abgestimmt werden müssen, bis aus den zugrundeliegenden Daten, Wissen abgeleitet werden kann.\seFootcite{Vgl.}{S. 7}{Mariscal.2010} Das Data Mining selbst kommt erst nach ausführlicher Datenvorbereitung zum Einsatz und kann so zu einer automatischen und explorativen Anpassung eines Modells -- wie bei der Funktionsmodellierung (vgl. \vref{fm}) -- an riesige Datenmengen genutzt werden.\seFootcite{Vgl.}{S. 1}{Adhikari.2015}\seFootcite{Vgl.}{S. 7}{Mariscal.2010}

In der Literatur existieren unterschiedliche Vorstellungen der einzelnen Prozessschritte, wodurch es oftmals zu Überschneidungen zwischen den einzelnen Gebieten kommt. So findet sich die Methode der \textit{Data Integration} einerseits in der Datenselektion wieder, andererseits auch in der Datenvorverarbeitung.\seFootcite{Vgl.}{S. 1}{Garcia.2015}\seFootcite{Vgl.}{S. 198}{Cleve.2014} Im Folgenden wird versucht, diese Schritte klar voneinander abzutrennen. Hierbei wird sich größtenteils an den Ausarbeitungen von Han et al. und Cleve et al. orientiert.



\section{Datenselektion}
\label{ds}

Opta Sports erfasst in einem Fußballspiel zwischen 1600 und 2000 verschiedene Events (z.B. Pässe, Schüsse, Fouls, uvm.), wovon eine geeignete Datenmenge für die Funktionsmodellierung ausgewählt werden muss. Zunächst werden die im \gls{xml}-Format vorliegenden Daten eines Spieles für eine bessere Weiterverarbeitung in ein \gls{json}-Format geparst. Über die in \vref{opta} beschriebenen \textsf{Type-IDs} und \textsf{Qualifiers} können die relevanten Schüsse in dieser großen Datenmenge selektiert werden. Aus der \vref{tab:events} lassen sich alle Schussversuche innerhalb eines Spiels identifizieren, wobei die Zieldaten nur Schüsse beinhalten dürfen, die während des \glqq freien\grqq~Spiels abgeben wurden, nicht aus Eigentoren resultierten und welche die nicht geblockt wurden (vgl. \vref{tab:anf}). Der Ausschluss solcher Schüsse erfolgt über die in \vref{tab:quali} gelisteten Qualifier. Die Eigentore wurden nicht von vornherein ausgeschlossen und konnte erst durch die Visualisierung aller Torerfolge in \vref{owngoals} als irrelevant erkannt werden, da diese die Wahrscheinlichkeit ein Tor aus dieser Position zu erzielen total verzerren. \enlargethispage{2\baselineskip} Das \glqq Eigentor des Jahres\grqq\seFootcite{Vgl.}{}{FrankfurterAllgemeineZeitung.2014} von Christopher Kramer aus 45-Meter kann beispielsweise ohne Angabe des Qualifiers auch als sehr weiten Fernschuss interpretiert werden.


\begin{sidewaysfigure}[H]
\centering
\includegraphics[scale=0.4]{se-wa-jpg/owngoals}
\caption[Ausschluss der Eigentore]{Ausschluss der Eigentore (rote Punkte) }
\label{owngoals}
\end{sidewaysfigure}

%\begin{figure}
%\centering
%\includegraphics[scale=0.28]{se-wa-jpg/owngoals}
%\caption[Ausschluss der Eigentore]{Ausschluss der Eigentore}
%\label{owngoals}
%\end{figure}

\subsection{Datenvorverarbeitung}
\label{aufbereitung}
\glqq Da die Zieldaten aus den Datenquellen lediglich extrahiert werden, ist im Rahmen der Datenvorverarbeitung die Qualität des Zieldatenbestandes zu untersuchen und – sofern nötig – dieser durch den Einsatz geeigneter Verfahren zu verbessern.\grqq\seFootcite{}{S. 9}{Cleve.2014}

Diese essentielle Phase verfolgt das Ziel, die unstrukturierten und zunächst nutzlos scheinenden, selektierten Rohdaten, in qualitativ hochwertigere Daten umzuwandeln, um diese der passenden \gls{dm}-Methode in einem geeigneten Format bereitstellen zu können. Die Struktur und das Format müssen perfekt auf die vorliegende Aufgabe passen, ansonsten führt die geringe Qualität der Daten zu schlechten bzw. falschen Resultaten, bis hin zu Laufzeitfehlern.\seFootcite{Vgl.}{S. 10-11}{Garcia.2015} Es gilt auch hier das Prinzip: GIGO – garbage in, garbage out.\seFootcite{Vgl.}{S. 197}{Cleve.2014} Die oftmals schlechte Qualität der (Roh-)Daten ist durch \textit{fehlende, ungenaue, inkonsistente bzw. widersprüchliche} Daten zu begründen.\seFootcite{Vgl.}{S. 84}{Han.2012}\seFootcite{Vgl.}{S. 196}{Cleve.2014} Im Folgenden werden dazu einige Ursachen beispielhaft aufgeführt.

Ungenaue bzw. falsche Daten können schon bei der Erhebung entstehen, wenn ein falsches Datenerhebungsinstrument ausgewählt wird. Bei Stichproben sollte die Gesamtmenge so präzise wie möglich widergespiegelt werden, um die Datenakkuratesse nicht zu gefährden.\seFootcite{Vgl.}{S. 25}{Fahrmeir.2007} Weiterhin können technische und menschliche Fehler zu ungenauen Daten führen, indem Personen beispielsweise ihre persönlichen Informationen bei einer Befragung absichtlich verschleiern (z.B. Standardwert für Geburtsdatum 1. Januar), wobei man diese Problematik auch als \textit{\glqq disguised missing data\grqq}~bezeichnet.\seFootcite{Vgl.}{S. 24}{Fahrmeir.2007}\seFootcite{Vgl.}{S. 196}{Cleve.2014} Neben der falschen subjektiven Einschätzung des Menschen bei der Erhebung, können auch von einem technischem Blickwinkel ungenaue Daten ermittelt werden, wie z.B. durch (teils-)defekte Sensoren.\enlargethispage{\baselineskip}  Nicht zuletzt können Daten bei einem Transfer verfälscht werden bzw. sogar verloren gehen.\seFootcite{Vgl.}{S. 84}{Han.2012}

Fehlende Daten lassen sich einerseits durch technische Mängel begründen, andererseits auch durch die Tatsache, dass bestimmte Attribute schlichtweg von Beginn an bei der Erhebung nicht beachtet wurden oder durch bestimmte Restriktionen nicht verfügbar waren.\seFootcite{Vgl.}{S. 84-85}{Han.2012}

Die aufgezeigten Beispiele spiegeln nur einen kleinen Teil möglicher Ursachen wider und sollen die Bedeutsamkeit dieser Phase für den Data-Mining-Prozess aufzeigen. Die Datenvorbereitung stellt dabei einige leistungsstarke Werkzeuge zur Verfügung, um die Datenqualität nachhaltig zu verbessern:\seFootcite{Vgl.}{S. 11 ff}{Garcia.2015}\seFootcite{Vgl.}{S. 196 ff}{Cleve.2014}\seFootcite{Vgl.}{S. 84 ff}{Han.2012}

\begin{itemize}
\item \textbf{Data Cleaning}: In diesem Schritt werden die Daten bereinigt, indem beispielsweise \textit{fehlerhafte} oder \textit{störende} Daten korrigiert werden (siehe \vref{dc}).

\item \textbf{Data Integration}: Diese Phase beschäftigt sich mit der fehlerfreien Zusammenführung von Daten, da diese oftmals aus mehreren unterschiedlichen Quellen stammen (siehe \vref{di}).

\item \textbf{Data Reduction}: Um die Algorithmen der Data-Mining-Methoden nutzen zu können, muss die exorbitante Datenmenge reduziert bzw. komprimiert werden, sodass lange Laufzeiten verhindert beziehungsweise reduziert werden können (siehe \vref{dr}).
\end{itemize}

Auf die in \vref{werkzeuge} vereinfacht dargestellten Werkzeuge und ihre Konzepte, wird in den folgenden Unterkapiteln näher eingegangen.

\begin{figure}[H]
\centering
\includegraphics[scale=1.2]{se-wa-jpg/preprocessing}
\caption[Werkzeuge der Datenvorverarbeitung]{Werkzeuge der Datenvorverarbeitung\protect\footnotemark}
\label{werkzeuge}
\end{figure}
\footnotetext{Vgl. Abbildung \textit{Han}, Data Mining: Concepts and techniques, 2012, S. 87}


\subsubsection{Data Cleaning}
\label{dc}
In der realen Welt sind Daten häufig \glqq unvollständig, mit Fehlern oder Ausreißern behaftet oder sogar inkonsistent.\grqq\seFootcite{}{S. 199}{Cleve.2014} Um Fehler oder gar falsche Resultate im Data-Mining-Prozess frühzeitig zu vermeiden, ist es von großer Bedeutung, die Datenmengen zu bereinigen. Der Fokus sollte hierbei auf der Informationsneutralität liegen. Das bedeutet, es sollen möglichst keine neuen Informationen hinzugefügt werden, die das reale Abbild verzerren oder verfälschen können.\seFootcite{Vgl.}{S. 199-200}{Cleve.2014} Folgende Problemarten gilt es zu behandeln:

\paragraph{Fehlende Daten} 
Dem Datenanalyst stehen einige Möglichkeiten zur Verfügung, um auf fehlende Daten zu reagieren:\seFootcite{Vgl.}{S. 88-90}{Han.2012}\seFootcite{Vgl.}{S. 200-201}{Cleve.2014}

\begin{itemize}
\item \textit{Attribut ignorieren}
\\ Der Datensatz mit dem fehlenden Attribut wird gänzlich ignoriert oder gelöscht. Jedoch können dadurch wichtige Informationen für die Datenanalyse verloren gehen, wodurch dieses Verfahren nur bei Datensätzen mit mehreren Lücken angewandt werden sollte.

\item \textit{Manuelles Einfügen}
\\ Besitzt der Datenanalyst das nötige Wissen, kann dieser einzelne Datensätze nachträglich manuell einfügen. Dieser Vorgang entwickelt sich schnell zu einem sehr zeitaufwändigen und schwer zu realisierenden Vorgang, der aufgrund des Mangels an Ressourcen (personeller wie auch zeitlicher) nicht zu realisieren ist, sobald die Datenmenge wächst (z.B. Nachtrag von 500 Kundendaten per Hand).

\item \textit{Globale Konstante}
\\ Den fehlenden Wert durch eine globale Konstante (auch als Standardwert bezeichnet) zu ersetzen, ist sinnvoll, wenn auch ein leeres Feld als Information angesehen wird. Beispiele für Konstanten wären \textit{unbekannt} oder \textit{minus unendlich}.

\item \textit{Durchschnittswert}
\\ Handelt es sich bei dem fehlenden Attribut um einen metrischen Wert, so kann der Durchschnittswert aller Einträge als Ersatz verwendet werden. Der Durchschnittswert bietet sich als äußerst einfache Möglichkeit, wenn die Daten klassifiziert werden können und die Berechnung nur auf Datensätzen derselben Klasse angewandt wird. Die Methode der \gls{knn}\seFootcite{Vgl.}{S. 76}{Garcia.2015} steht zur Verfügung, wenn keine Klassen vorhanden sind. Hierbei wird der Durchschnitt, der dem aktuellen Datensatz ähnlichsten Werte benutzt.

\item \textit{Wahrscheinlichster oder häufigster Wert}
\\ Durch statistische Methoden kann der wahrscheinlichste Wert für das fehlende Attribut ermittelt werden, jedoch sollte diese Angleichung begründet sein. Bei nicht numerischen Werten kann als weitere Möglichkeit auch der häufigste Wert als Ersatz für das fehlende Attribut verwendet werden.
\end{itemize}

\paragraph{Verrauschte Daten und Ausreißer}
\label{outlierchapter}
Durch ungenaue Messwerte oder falsche Schätzungen entstehen die sogenannten \textit{verrauschten Daten}.\footnote{Im englischen Sprachgebrauch als \textit{noisy data} bekannt.} Um diese bereinigen zu können, stehen dem Datenanalyst einige Verfahren zur Verfügung, durch welche diese fehlerbehafteten Daten angeglichen werden können.\footnote{Auch als \textit{smoothing} bekannt.} Als \textit{Ausreißer} bezeichnet man dabei Daten, die erheblich von den anderen Daten abweichen oder außerhalb eines Wertebereiches (z.B. $0<x<100$) liegen.\seFootcite{Vgl.}{S. 89-90}{Han.2012} Beispielsweise liegen bei einer Befragung Daten von 30- bis 50-Jährigen vor, darunter jedoch auch eine 90-jährige Person. Hierbei könnte es sich um einen Ausreißer handeln, aber auch um einen fehlerhaften Datensatz.\seFootcite{Vgl.}{S. 196}{Cleve.2014} \glqq Ob solche Ausreißer für das Data Mining ausgeblendet oder adaptiert werden sollten oder besser doch im Originalzustand zu verwenden sind, hängt vom konkreten Kontext ab.\grqq\seFootcite{}{S. 196}{Cleve.2014}


\begin{itemize}

\item \textit{Klasseneinteilung (bining)}
\\ Durch die Gruppierung verrauschter Daten in Klassen, können diese beispielsweise durch den Mittelwert oder die naheliegenden Grenzwerte ersetzt werden. 

\item \gls{regression}
\\ Die Darstellung der Daten in Form einer mathematischen Funktion, bietet die Möglichkeit, fehlerbehaftete Daten  durch die berechneten Funktionswerte zu ersetzen. Für zwei Abhängigkeiten zwischen zwei Attributen steht hierbei neben der \textit{linearen Regression}, auch die \textit{multiple lineare Regression} für mehrere Attribute als Werkzeuge zur Verfügung (weiterführende Ausarbeitung zur Regressionsanalyse siehe \vref{ra}). 

\item \textit{Verbundbildung (\gls{clustering})}
\\ Eine der einfachsten Möglichkeiten um Ausreißer zu erkennen, bietet die Verbundbildung, auch \gls{clustering} genannt. Hierbei werden ähnliche Daten, wie in \vref{outlier} dargestellt, zu \textit{Clustern} zusammengeführt, wodurch sich die Ausreißer direkt identifizieren lassen.

\begin{figure}[H]
\centering
\includegraphics[scale=1.2]{se-wa-jpg/outlier}
\caption[Outlierdetection mittels Clustering]{Outlierdetection mittels Clustering\protect\footnotemark}
\label{outlier}
\end{figure}
\footnotetext{Vgl. Abbildung \textit{Han}, Data Mining: Concepts and techniques, 2012, S. 91}
\end{itemize}

\paragraph{Falsche und inkonsistente Daten}
Bei falschen bzw. inkonsistenten Daten ergeben sich prinzipiell zwei Möglichkeiten zur Korrekturbehandlung. Einerseits können der Datensatz oder bestimmte Attribute durch \textit{Löschen} entfernt werden, wobei jedoch die Gefahr einer zu großen Reduktion des Datenbestandes entsteht und relevante Informationen für das Data Mining verloren gehen könnten. Die zweite Korrekturvariante versucht den inkonsistenten Datensatz, durch die \textit{Zuhilfenahme anderer Datensätze}, sinnvoll zu ersetzen. Sollte eine Unterscheidung zwischen \textit{falsch} und \textit{richtig} nicht möglich sein, wären beim Löschen immer mindestens zwei Datensätze betroffen.\seFootcite{Vgl.}{S. 203-204}{Cleve.2014}

\subsubsection{Data Integration}
\label{di}
Bei Data-Mining-Projekten ist oftmals die Integration mehrerer Datenbestände aus unterschiedlichen Quellen erforderlich. Diese Phase sollte mit äußerster Sorgfalt durchgeführt werden, um frühzeitig redundante und inkonsistente Datensätze zu vermeiden, wodurch die Genauigkeit und Geschwindigkeit der nachfolgenden Data Mining Algorithmen nicht gefährdet wird.\seFootcite{Vgl.}{S. 93-94}{Han.2012}\enlargethispage{\baselineskip} Folgende Punkte gilt es bei der Datenintegration zu beachten:

\begin{itemize}
\item \textbf{Identifikationsproblem von Entitäten}:
\\ Bei der Datenintegration aus multiplen Datenquellen, wie beispielsweise Datenbanken oder Dokumenten, stellt die Schema-Integration wie auch die Objektanpassungen eine schwierige Herausforderung dar. Der Datenanalyst muss sicherstellen, dass zum Beispiel das Attribut \textit{kunden\_nummer} aus der einen Datenquelle, dieselbe Referenz besitzt, wie das Attribut \textit{kunden\_id} aus einer anderen und es sich folglich um dasselbe Attribut handelt. Dies wird allgemein als \textit{Entity Identification Problem} bezeichnet.\seFootcite{Vgl.}{S. 199}{Cleve.2014}\seFootcite{Vgl.}{S. 94}{Han.2012} Die Metadaten der Attribute beinhalten Informationen, wie \textit{Name, Bedeutung, Datentyp, Wertebereich,} uvm. und können durch Abgleich derer zu einer Vermeidung von Fehlern bei der Integration beitragen. Weiterhin muss gesondert auf die \textit{Datenstruktur} geachtet werden, um keine referentiellen Abhängigkeiten bzw. Beziehungen zwischen den Daten zu zerstören.\seFootcite{Vgl.}{S. 94}{Han.2012} 

\item \textbf{Redundanzen bei Attributen}:
\\ Ein Attribut, welches durch ein anderes Attribut ableitbar ist -- wie zum Beispiel das Alter vom Geburtsjahr berechnet werden kann -- wird als redundant bezeichnet. Die Vielzahl von Redundanzen führt zu unnötig großen Datenmengen, die wiederum die Performanz sowie die Resultate eines Data Mining Algorithmus negativ beeinträchtigen können.\seFootcite{Vgl.}{S.41}{Garcia.2015} Folglich sollte diese Problematik durch die Anwendung von statistischen Verfahren, in Form der Korrelationsanalyse, dezidiert behandelt werden. Für numerische Werte ist dabei der Einsatz von Korrelationskoeffizienten und Kovarianzen hilfreich. Um die Implikation zweier Attribute einer nominalen Datenmenge\footnote{Rein qualitative Merkmalsausprägungen ohne natürliche Rangordnung (wie z.B. das Geschlecht).} bestimmen zu können, verwendet man in der Regel den $\chi^2$(\textit{Chi-Quadrate})-Test.\seFootcite{Vgl.}{S. 95.}{Han.2012}\seFootcite{Vgl.}{S. 41}{Garcia.2015}\seFootcite{Vgl.}{S. 64}{Cleve.2014} 

\item \textbf{Duplikatserkennung}:
\\ Duplikate verkörpern Redundanzen auf Datensatzebene und führen einerseits zu unnötig großen Datenmengen, die sich wiederum auf die Performanz der Algorithmen auswirken. Andererseits führt jedoch auch die verfälschte Gewichtung der mehrfach vorkommenden Datensätze, zu falschen Analyseergebnissen.\enlargethispage{\baselineskip} Ein häufiger Grund stellt dabei die Verwendung von denormalisierten Datenbanktabellen dar.\seFootcite{Vgl.}{S. 98}{Han.2012}\seFootcite{Vgl.}{S. 43}{Garcia.2015}

\item \textbf{Konflikte bei Attributswerten}:
\\ Hierbei handelt es sich um die unterschiedliche Darstellung, Skalierung und Kodierung von Attributswerten. Beispielsweise kann das Attribut \textit{Gewicht} durch das metrische System oder das britische Maßsystem repräsentiert werden, woraus bei der Zusammenführung von Daten in eine einheitliche Quelle immer wieder Konflikte resultieren.\seFootcite{Vgl.}{S. 99}{Han.2012}\seFootcite{Vgl.}{S. 199}{Cleve.2014} 
\end{itemize}

\subsubsection{Data Reduction}
\label{dr}
Die bereits mehrfach angesprochene Problematik der riesigen Datenmengen bei Data-Mining-Projekten, steigert die Komplexität und vermindert die Effizienz der Algorithmen. Daher strebt die Datenreduktion -- wie die Bezeichnung erkennen lässt -- nach einer reduzierten repräsentativen Teilmenge, welche die Integrität des Originals nicht verliert. Dazu können folgende drei Techniken angewandt werden:\seFootcite{Vgl.}{S. 147 ff}{Garcia.2015}\seFootcite{Vgl.}{S. 99-100}{Han.2012}\seFootcite{Vgl.}{S. 206-208}{Cleve.2014} 

\begin{enumerate}

\item Dimensionsreduktion 

\item Datenkompression

\item Numerische Datenreduktion

\end{enumerate}

\paragraph{Dimensionsreduktion}
Hierbei bleiben irrelevante Attribute des Datensatzes unberücksichtigt und nur für die Analyse relevante Daten werden miteinbezogen. Allgemein empfehlen sich dafür zwei Verfahren: Bei der schrittweisen \textit{Vorwärtsauswahl} werden wesentliche Attribute einer sukzessiv wachsenden Zielmenge zugeordnet. Im Gegensatz dazu werden bei der \textit{Rückwärtseliminierung} die uninteressanten Daten schrittweise aus der Zielmenge eliminiert.\seFootcite{Vgl.}{S. 206}{Cleve.2014} 

\paragraph{Datenkompression}
\label{datenkompression}
\enlargethispage{2\baselineskip}Bei dieser Technik wird durch Transformation oder Codierung versucht, eine Reduktion der Datenmenge zu erreichen. Fasst man beispielsweise die einzelnen Attribute \textit{Tag, Monat und Jahr} zu einem neuen Attribut \textit{Datum} zusammen, können Datensätze komprimiert werden.\seFootcite{Vgl.}{S. 207}{Cleve.2014}
 
\paragraph{Numerische Datenreduktion}
Anstatt die gesamte Datenmenge für die Analyse heranzuziehen, wird innerhalb der numerischen Datenreduktion eine repräsentative Teilmenge -- in Form einer Stichprobe -- für das \gls{dm} genutzt. Im Vordergrund steht hierbei die passende Auswahl unterschiedlicher Stichprobeverfahren, wie der \textit{zufälligen Stichprobe} oder der \textit{repräsentativen Stichprobe}, woraus jedoch kein verzerrtes Abbild der Daten resultieren darf.\seFootcite{Vgl.}{S. 25-27}{Fahrmeir.2007}\seFootcite{Vgl.}{S. 207}{Cleve.2014}
\subsection{Datentransformation}
\label{dt}
Nachdem die (Roh-)Daten selektiert, bereinigt und auf eine relevante Zielmenge reduziert wurden, müssen diese noch in eine adaptierte Form für die Algorithmen des Data Minings transformiert werden.\seFootcite{Vgl.}{S. 112}{Han.2012} Oftmals müssen sogar neue Attribute aus einem Datensatz kreiert werden, da dieser nicht in geeigneter Struktur für das Data-Mining-Verfahren vorliegt.\seFootcite{Vgl.}{S. 48}{Garcia.2015} Dazu es gibt eine Reihe an unterschiedlichen Transformationsmöglichkeiten, wobei in dieser Arbeit ein Auszug der relevanten Methoden vorgestellt werden soll:

\paragraph{Codierung}
Liegen beispielsweise Attribute mit einer ordinalen Ausprägung vor (wie \textit{sehr groß, groß, mittel und klein}), müssen diese bei einer Verwendung des \gls{knn}-Algorithmus in numerische Werte transformiert werden (Werte zwischen 0 und 1). Hierbei würde sich folgende Codierung für das Attribut \textit{Körpergröße} anbieten:\seFootcite{Vgl.}{S. 210}{Cleve.2014}

\begin{itemize}
\item \textit{sehr groß} $\rightarrow$ 1
\item \textit{groß} $\rightarrow$ 0,66
\item \textit{mittel} $\rightarrow$ 0,33
\item \textit{klein} $\rightarrow$ 0
\end{itemize}

Die Ordnungsrelation, hier \textit{sehr groß > groß > ...}, darf dabei jedoch nicht verloren gehen. In Abhängigkeit zu dem jeweiligen Verfahren, müssen Daten, so wie dies bei Maßeinheiten immer wieder der Fall ist, oftmals kodiert werden.\seFootcite{Vgl.}{S. 211}{Cleve.2014}

\paragraph{Normalisierung und Skalierung}
Unterschiedliche Maßeinheiten -- wie \textit{Körpergröße} und \textit{Körpergewicht} -- können die Datenanalyse negativ beeinflussen und müssen daher in eine einheitliche Skalierung transformiert werden, um eine gleiche Gewichtung aller Attribute zu erreichen. Man bedient sich hierbei in der Regel an der \textit{Min-Max-Normalisierung} (siehe \vref{minmax}) oder der \textit{Z-Transformation}, um numerische Werte auf ein [0,1] Intervall zu normieren.\seFootcite{Vgl.}{S. 114}{Han.2012}\seFootcite{Vgl.}{S. 212}{Cleve.2014}

\begin{figure}[H]
\begin{equation}
x_{neu} = \frac{x - min(x_i)}{max(x_i) - min(x_i)}
\end{equation}
\caption{Min-Max-Normalisierung}
\label{minmax}
\end{figure}

\paragraph{Datenaggregation}
Nicht nur aus Sicht der Datenkompression (vgl. \vref{datenkompression}) ist die Datenaggregation erforderlich. Vielmehr \glqq kann die Aggregation aus inhaltlichen Gründen sinnvoll sein.\grqq\seFootcite{}{S. 214}{Cleve.2014} Wenn Daten auf einer zu detaillierten Ebene vorliegen -- wie beispielsweise Einwohnerzahlen von Stadtteilen -- müssen diese für einen Städtevergleich erst summiert werden, um bundesweite Aussagen treffen zu können. Je nach Kontext können verschiedene Aggregationsmethoden (wie z.B. Summenbildung, Durchschnitt, usw.) für die Transformation zu einem einzigen Wert angewendet werden.\seFootcite{Vgl.}{S. 112}{Han.2012}

\paragraph{Datenglättung}
Die bereits in \vref{outlierchapter} vorgestellten Techniken zur Bereinigung von verrauschten Daten und Ausreißern, finden auch bei der Transformation ihre Verwendung. Die Datenglättung strebt nach einer reduzierten Datenmenge, in welcher jeder numerische Wert durch einen idealisierten Wert, wie beispielsweise der \textit{Regression}, ersetzt wird.\seFootcite{Vgl.}{S. 214-215}{Cleve.2014}
\section{Data Mining}
\subsection{Funktionsmodellierung}
\subsection{Regressionsanalyse}
\subsection{MatLab}
\section{Interpretation und Evaluation der Modelle}
Die Interpretation sowie die Evaluation der Resultate des Data Minings stehen am Ende jedes KDD-Prozesses. Um eine fundierte Entscheidung für ein Modell treffen zu können, werden im Folgenden die vier vorgestellten Modellvarianten interpretiert und miteinander verglichen (siehe \vref{inter}), sowie die ausgewählte Funktion anschließend in \vref{eva} evaluiert.

\subsection{Interpretation}
\label{inter}
Anhand der in \vref{int} vorgestellten Kriterien der \textit{Validität}, der \textit{Neuartigkeit}, der \textit{Nützlichkeit}, sowie der \textit{Verständlichkeit} des Modelles, werden die vier vorgestellten und realisierten Regressionsmodelle aus \vref{umsetzung} miteinander verglichen. Für eine vereinfachte Darstellung der Erfüllung der Anforderungen werden den Modellen folgende Abkürzungen zu geordnet:

\begin{itemize}
\item Multiple Regression der Winkel- und Distanzbetrachtung $\rightarrow$ \textbf{A}
\item Nichtparam. Regression der Winkel- und Distanzbetrachtung $\rightarrow$ \textbf{B}
\item Multiple Regression der Koordinatenbetrachtung $\rightarrow$ \textbf{C}
\item Nichtparam. Regression der Koordinatenbetrachtung $\rightarrow$ \textbf{D}
\end{itemize}

Die \vref{tab:verg} bietet in Form einer Matrix einen übersichtlichen Vergleich, inwieweit jedes Modell die obengenannten Kriterien erfüllen konnte. Die \textit{Validität} setzt sich dabei einerseits aus der Bewertung der objektiven Bestimmtheitsmaßen der einzelnen Modellierungen zusammen, andererseits auch aus subjektiven Wahrnehmungen, wie die vorgestellte Problematik der Wahrscheinlichkeitsbestimmung von Schussversuchen aus spitzem Winkel (vgl. \vref{heatmap}). Dabei überzeugt einzig allein das Modell \textbf{D} mit Bestimmtheitsmaßen um die \textsf{80}\% und einer korrekten Modellierung der Schussversuche aus spitzem Winkel. Die anderen drei Modelle haben mit ca. \textsf{50}\% eine zu geringe Anpassung der Funktion an die Daten und werden demzufolge auch für zukünftige Daten nicht valide sein. Das Kriterium der \textit{Neuartigkeit} erfüllen alle Modelle, da bislang keine Modellierungen von Funktionen zur Berechnung der Wahrscheinlichkeit eines Torerfolges im Fußball existieren. Die \textit{Nützlichkeit} ergibt sich größtenteils aus der \textit{Validität}, da nur ein Modell für einen Anwender nützlich sein kann, wenn valide Werte zurückgegeben werden. Modell \textbf{B} und \textbf{D} können diese Anforderung teilweise erfüllen und ermöglichen trotz abstrakter Modellierung erste Ansätze zur Mustererkennung. Letztlich beschreibt die \textit{Verständlichkeit} inwiefern die Ergebnisse des Modells von einem anderen Anwender erfasst werden. 

%%%%%%%%%%%%%%%%%% Vergleich %%%%%%%%%%%%%%%%%%
\tablefirsthead{\hline\multicolumn{5}{|c|}{\textbf{Vergleich der Modelle}}\\\hline\hline \textbf{Kriterium} & \textbf{A} & \textbf{B} & \textbf{C} & \textbf{D}\\\hline}
\tablehead{}
\tabletail{}
\tablelasttail { \multicolumn {5}{| c |}
{\textsl { \textcolor{green}{\ding{52}} erfüllt~~~~~~~\textcolor{orange}{\ding{108}} teilweise erfüllt~~~~~~~ \textcolor{red}{\ding{54}} nicht erfüllt  }}\\\hline }
\bottomcaption{Vergleich der Modelle\label{tab:verg}}
\begin{center}%
\begin{supertabular}{ | P{4cm} | P{2cm}  | P{2cm} | P{2cm}  |  P{2cm}  |}
&&&&\\
\textit{Validität}	& \textcolor{red}{\ding{54}}	 & \textcolor{orange}{\ding{108}} & \textcolor{orange}{\ding{108}}	&\textcolor{green}{\ding{52}}	\\
&&&&\\
\hline
&&&&\\
\textit{Neuartigkeit}	& \textcolor{green}{\ding{52}}	 & \textcolor{green}{\ding{52}}	& \textcolor{green}{\ding{52}}	& \textcolor{green}{\ding{52}}	\\
&&&&\\
\hline
&&&&\\
\textit{Nützlichkeit}	&\textcolor{red}{\ding{54}}	 & \textcolor{orange}{\ding{108}} & \textcolor{orange}{\ding{108}}	& \textcolor{green}{\ding{52}}	\\
&&&&\\
\hline
&&&&\\
\textit{Verständlichkeit}	&\textcolor{red}{\ding{54}}	 & \textcolor{red}{\ding{54}}	& \textcolor{orange}{\ding{108}}	& \textcolor{green}{\ding{52}}	\\
&&&&\\
\hline
\end{supertabular}
\end{center}

\subsection{Evaluation}
\label{eva}


\section{Funktionsmodellierung}
\label{fm}
Nachdem die Regressionsanalyse in \vref{dmmethoden} als Data-Mining-Methode für diese Arbeit festgelegt wurde, wird der Leser im folgenden Kapitel mit den grundlegenden Bestandteilen der Funktionsmodellierung mit Hilfe der \gls{regression} vertraut gemacht (siehe \vref{ra}). Dazu werden die unterschiedlichen Modelle der Regression vorgestellt und in Bezug auf die vorliegende Problemstellung bewertet. Anschließend wird in \vref{matlab} das Software-Tool \gls{matlab} zur Lösung und graphischen Darstellungen von mathematischen Problemen in Bezug auf die Regressionsanalyse beschrieben, um dessen Konzepte und Funktionsweise für die spätere Umsetzung nachvollziehen zu können.


\subsection{Regressionsanalyse}
\label{ra}
\subsection{Allgemein}
Die Regression (lat. \textit{regredi} für umkehren, zurückkehren) beinhaltet im Allgemeinen die Analyse einer abhängigen Variablen von einer oder mehreren unabhängigen Variable.\seFootcite{Vgl.}{S. 5}{Studenmund.2014} Dabei drücken die unabhängigen Variablen die abhängige Variable mittels einer \textit{Regressionsgleichung} aus.\seFootcite{Vgl.}{S. 475}{Fahrmeir.2007} Die in der mathematischen Gleichung beinhaltenden Parameter (auch \textit{Regressoren} genannt) müssen so gewählt und justiert werden, dass diese bestmöglich zu den vorhanden Daten passen.\seFootcite{Vgl.}{S. 68}{Gunther.2014}\seFootcite{Vgl.}{S. 1-2}{Schimek.2000} Diese rein datenbasierte mathematische Beschreibung hat ihren Ursprung in einer Studie von Francis Galton\footnote{britischer Naturforscher im 19. Jahrhundert - prägte erstmals den Begriff der \textit{Regression}}, in der die Körpergröße von Kindern in Bezug zu denen ihrer Eltern analysiert wurde.\seFootcite{Vgl.}{S. 68}{Gunther.2014} Anhand der vorliegenden Problemstellung dieser Arbeit, lässt sich das Vorgehen der Regressionsanalyse beispielhaft demonstrieren:

Es wird versucht die Wahrscheinlichkeit eines Torerfolges (\textit{=abhängige Variable}), durch mehrere Faktoren, wie der Distanz oder des Winkel zum Tor bzw. der Koordinaten des Schusses (\textit{=unabhängige Variablen}), mittels einer Funktion (\textit{=Regressionsgleichung}), zu ermitteln. 

Anhand der \textit{linearen Regression} sollen die grundlegende Bestandteile aller Regressionsmodelle erläutert werden. Um eine Punktewolke durch eine Funktion $\hat{f}$ zu approximieren, bedient man sich der quadratischen Abstände der Punkte zur Funktion und versucht diese durch die \gls{mdkq} zu minimieren.\seFootcite{Vgl.}{S. 44}{Hastie.2016} Die lineare Regressionsfunktion dabei liegt in der Form 

\begin{equation}
	\hat{f}(x) = \hat{\alpha} \cdot x + \hat{\beta}
	\label{lrf}
\end{equation}

vor.\seFootcite{Vgl.}{S. 476}{Fahrmeir.2007} Wie in \vref{lr} exemplarisch dargestellt, werden die Abstände zwischen den Punkten und der Funktion ($y_i -  \hat{y}(x_i) \rightarrow i= 1,...,m$) summiert, woraus sich die \textit{Summe der Fehlerquadrate} (=\gls{rss}) ergibt:\seFootcite{Vgl.}{S. 37}{Studenmund.2014}

\begin{equation}
	RSS = \sum\limits_{i=1}^n (y_i - \hat{y}(x_i))^2
\end{equation}

Durch die Berechnung der \gls{rss} wird die Distanz zwischen den Daten und dem Modell berechnet. Um eine möglichst optimale Anpassung des Modells an die Daten dabei zu erreichen, müssen die Parameter $\hat{\alpha}$ und $\hat{\beta}$ so gewählt werden, dass die Summe der Fehlerquadrate minimal ist.\seFootcite{Vgl.}{S. 69}{Gunther.2014} Es gilt:

\begin{equation}
	\min\limits_{a,b\in\mathbb{R}} RSS
\end{equation}

Das übliche Vorgehen für die Auffindung eines Minimums einer Funktion mit mehreren Variablen -- hier RSS($\hat{\alpha},\hat{\beta}$) -- wird auch hier angewendet, um mit Hilfe der partiellen Ableitung die Parameter ausfindig zu machen:\seFootcite{Vgl.}{S. 39}{Studenmund.2014}

\begin{equation}
	\hat{\alpha} = \frac{\sum\limits_{i=1}^m x_i y_i - m \overline{x} \overline{y}}{\sum\limits_{i=1}^m x^2_i - m \overline{x}^2}
\end{equation}

\begin{equation}
	\hat{\beta} = \overline{y} - \hat{\alpha} \overline{x}^2
\end{equation}

Diese grundlegenden Bestandteile finden sich in allen Regressionsmodellen wieder, die in \vref{rm} kurz vorgestellt werden. 

%%%%%%%%%%%%%%%%%%%%%%%%%%%%%%%%%%%%%%%%%%%%%%%%%%

\subsubsection{Regressionsmodelle}\label{rm}
In der Praxis haben sich durch den allgemein Ansatz der Regression eine Vielzahl von Modellen etabliert, die je nach Anwendungsfall ihre Verwendungen finden:


\begin{itemize}

%%%%%%%%%%%%% LINEARE REGRESSION %%%%%%%%%%%%%%%%%%

\item \textbf{Lineare Regression}
\\ Am bekanntesten ist die lineare Regression, die auch oft als \glqq Ausgleichsgerade\grqq~bezeichnet wird und zur Prognose einer interessierenden Größe $y$, in Abhängigkeit \textit{einer} bekannten Größe $x$, angewendet wird.\seFootcite{Vgl.}{S. 475}{Fahrmeir.2007} Eine Funktion $\hat{f}(x)$ wie \vref{lrf} ist von ihren Regressoren $\alpha$ und $\beta$ \textit{linear} abhängig und heißt deshalb \textit{lineare Regressionsfunktion}.\seFootcite{Vgl.}{S. 68}{Gunther.2014} In \vref{lr} ist eine solche Funktion beispielhaft dargestellt.
\input{\seWaPathText/BA/Theorie/figure_lr}


%%%%%%%%%%%%% NICHT LINEARE REGRESSION %%%%%%%%%%%%%%%%%%

\item \textbf{Nichtlineare Regression}

In vielen realen Anwendungen kann die Regressionsfunktion nicht durch die Linearkombination der Regressionskoeffizienten berechnet werden, da diese in \textit{linearer} Weise von den Regressoren abhängt. Allgemein lässt sich dieses Modell mit $\boldsymbol{x} = (x_1,...,x_n)$ und $\boldsymbol{a} = (a_1,...,x_s)$ wie folgt ausdrücken:\seFootcite{Vgl.}{S. 85}{Gunther.2014}

\begin{equation}
	\hat{f}(\boldsymbol{x}) = f(\boldsymbol{x},\boldsymbol{a})
\end{equation}

In \vref{nlr} liegen die Daten in einem oszillierenden, sinusförmigen Muster vor und können folglich mit der allgemeinen Sinusfunktion beschrieben werden ($\hat{f} = \alpha_0 \cdot \sin(\alpha_1 \cdot (x-\alpha_2))$). Bei dieser nichtlinearen Regressionsfunktionen können die Regressoren dazu verwendet werden, die Funktion möglichst genau an die Daten anzupassen, wobei $\alpha_0$ die Amplitude bestimmt, $\alpha_1$ die Periode und $\alpha_2$ die Sinus-Funktion entlang der $x$-Achse verschiebt.\seFootcite{Vgl.}{S. 85-86}{Gunther.2014} Auch hier lässt sich mit Hilfe des Standardmodells, der Minimierung der kleinsten Quadrate, die Parameter bestimmen.\seFootcite{Vgl.}{S. 509}{Fahrmeir.2007}

\input{\seWaPathText/BA/Theorie/nlr}


%%%%%%%%%%%%% MULTIPLE REGRESSION %%%%%%%%%%%%%%%%%%

\item \textbf{Multiple Regression}\enlargethispage{2\baselineskip} 
In den meisten technischen und wirtschaftlichen Anwendungsfällen ist die Zielvariable $y$ von mehr als einer unabhängigen Variable abhängig. Dieser Fall kann durch die \textit{multiple} oder auch \textit{multivariate Regression} behandelt werden, wobei sich die Regressionsfunktion mit den unabhängigen Variablen ($\boldsymbol{x} = (x_1,...,x_n)$) und $f_i$ beliebigen reellen Funktionen im Allgemeinen wie folgt ausdrücken lässt:\seFootcite{Vgl.}{S. 41-42}{Studenmund.2014}

\begin{equation}
	\hat{f}(\boldsymbol{x}) = \alpha_0 + \alpha_1 f_1(\boldsymbol{x}) + \alpha_2 f_1(\boldsymbol{x}) + ... + \alpha_s f_s(\boldsymbol{x}) 
\end{equation}

Die Vorgehensweise zur Ermittlung der Parameter ist dabei dieselbe wie bei der linearen Regression - auch hier wird die Methode der Minimierung der kleinsten Quadrate angewendet. Liegt beispielsweise eine Punktewolke wie in \vref{mr} vor, kann diese durch die Funktion

\begin{equation}
	\hat{f}(x_1,x_2) = \alpha_0 + \alpha_1 x_1 + \alpha_2 x_2
\end{equation}

beschrieben werden, wobei der Regressor $\alpha_0$ die Verschiebung der Fläche entlang der $y$-Achse angibt, $\alpha_1$ die Steigung der Variable $x_1$ sowie $\alpha_2$ die Steigung von $x_2$. 

\input{\seWaPathText/BA/Theorie/mr}

%%%%%%%%%%%%% NICHT PARAMETRISCHE REGRESSION %%%%%%%%%%%%%%%%%%

\item \textbf{Nichtparametrische Regression}
In vielen Anwendungen lässt sich nicht von vornherein eine parametrische Spezifikation der Regressionsfunktion angeben. In den Beispielen zuvor -- ob linear oder nichtlinear -- wurde jeweils ein konkreter Ausdruck vorgegeben, um mittels Minimierung die Funktion an die Daten anzupassen. Betrachtet man \vref{pr}, so lässt sich schnell erkennen, dass es dazu keine passende mathematische Funktion geben wird.\seFootcite{Vgl.}{S. 91}{Gunther.2014} Die nichtparametrischen Regression verfolgt das Ziel, die Funktion $\hat{f}$ möglichst genau zu schätzen. Etabliert haben sich Methoden wie \textit{Spline-Regressionen} und \textit{lokale Regressionsschätzer}, die jedoch aufgrund ihres numerischen Aufwand nicht hier nicht detailliert beschrieben werden und selbst nur durch die Verwendung von statistischen Programmpaketen (siehe \vref{matlab}) benutzt werden können.\seFootcite{Vgl.}{S. 510}{Fahrmeir.2007}

\input{\seWaPathText/BA/Theorie/pr}
\textit{\glqq Splinefunktionen gehören zu den wichtigsten und verbreitetsten Regressionsmethoden und werden quer durch alle Disziplinen z.B. in Betriebswirtschaft, Informatik, Bildverarbeitung, Medizin, Maschinen.\grqq}\seFootcite{}{S.93}{Gunther.2014} 

Der Gedanke der \textit{smoothing splines}\footnote{Die englische Übersetzung bedeutet so viel wie \textit{glättende Verzahnung}} ist, den Bereich der $x$-Werte durch ein feines Gitter so zu unterteilen, das sich die angrenzenden Intervalle durch glatt miteinander verbundene Polynomfunktionen niedrigen Grades (oftmals kubisches Polynom) approximieren lassen.\seFootcite{Vgl.}{S. 510}{Fahrmeir.2007} In \vref{splineZoom} ist dazu die Punktewolke aus \vref{pr} im Wertebereich zwischen $x_1$ und $x_2$ genauer dargestellt, um das Resultat dieses Verfahren besser betrachten zu können. Legt man den Augenmerk nun nur noch auf den Intervallbereich $I_i$, lässt sich dieser Bereich durch ein kubisches Polynom ausdrücken. Werden all diese Intervalle als eigene Funktionen definiert und stückweise an den \glqq Knotenpunkten\grqq~ stetig und differenziert aneinander gesetzt, erhält man die gesuchte Regressionsfunktion $\hat{f}(x)$.\seFootcite{Vgl.}{S. 510}{Fahrmeir.2007}

\input{\seWaPathText/BA/Theorie/splineZoom}

\end{itemize}

\paragraph{Bewertung} Wie in der Einleitung dieses Kapitels beschrieben, wird die Wahrscheinlichkeit für einen Torerfolg durch mehrere Faktoren, wie beispielsweise die Koordinaten des Schusses, beeinflusst, wodurch die \textit{lineare Regression} für die Modellierung ausgeschlossen werden kann, da die Zielvariable in der vorliegenden Problemstellung von mindestens zwei unabhängigen Variablen abhängt. Im multiplen Regressionsmodell können mehrere unabhängige Variablen behandelt werden, jedoch müsste auch hier von vornherein eine parametrische Spezifikation der Funktion angegeben werden. Eine erste mögliche Vorstellung in Bezug wäre dazu die Funktion $\hat{f}(x_1,x_2) = \alpha_0 e^{-x_1} \cdot \alpha_1\sin(\alpha_2 \cdot (\pi x_2 - \alpha_3))$, die in \vref{vermutung} abgebildet ist. 
\input{\seWaPathText/BA/Theorie/vermutung}

Die Parameter $x_1$ (=Breite des Spielfeldes) und $x_2$ (=Länge des Spielfeldes) geben hierbei die Koordinaten des Schusses an, wobei das gegnerische Tor auf der $x_1$-Achse -- also bei $x_2=0$ -- mittig platziert ist. Die Vermutung ist, dass sich die Wahrscheinlichkeit $y$ mit zunehmender Nähe zum Tor steigt, wodurch die Fläche in Richtung Tor stetig erhöht und eine Art \glqq Gipfel\grqq~ entsteht. Bereits bei der Betrachtung dieser Vermutung, ist zu erkennen, das eine \glqq herkömmliche\grqq mathematische Funktion zu einer zu sehr groben Glättung führt. Beispielsweise findet ein Schuss seitlich neben dem Tor auf der Grundlinie statt, dann ist allein aufgrund des Winkels ein Torerfolg fast unmöglich und die Wahrscheinlichkeit somit falsch repräsentiert. Folglich ist es sinnvoll, \textit{nichtparametrische Regressionsfunktionen} in Form von \textit{Splines} zu verwenden, um eine exakte Anpassung der Funktion an die vorliegenden Daten zu erreichen. Eine detaillierte Gegenüberstellung der multiplen und nichtparametrischen Regression wird innerhalb der Umsetzung in \vref{mdf} aufgezeigt.

%%%%%%%%%%%%% BESTIMMTHEITSMASS %%%%%%%%%%%%%%%%%%
\subsubsection{Bestimmtheitsmaß}
Um die Qualität der Anpassung eines Modells an die Daten zu überprüfen, stellt der graphische Vergleich zwischen Modell und Daten die einfachste Möglichkeit da. Betrachten wir nochmals die Sinus-Funktion aus \vref{nlr}, so wird schnell klar, dass man kein Regressionsexperte sein muss, um feststellen zu können, dass das Modell sehr gut an die Daten angepasst wurde.\seFootcite{Vgl.}{S.71}{Gunther.2014}  Für vernünftige Prognosen wird jedoch das Bestimmtheitsmaß $R^2$ verwendet, welches den \textit{goodness of fit}(\textit{dt. Anpassungsgüte}) misst.\seFootcite{Vgl.}{S. 51}{Studenmund.2014} Die Qualität wird dabei auf einer Skala zwischen 0 und 1 dargestellt, wobei ein sehr hoher Wert für eine gute Anpassung und ein niedriger Wert für eine schlechte Anpassung des Modells an die Daten spricht. Diese Skala ist nützlich um verschiedene Regressionsmodelle miteinander vergleichen zu können.\seFootcite{Vgl.}{S.71}{Gunther.2014} Das Bestimmtheitsmaß $R^2$ misst dabei zu welchem prozentualen Anteil die Abweichung der gemessenen abhängigen Variablen durch die unabhängigen Variablen des Modells erklärt wird und ist formal wie folgt definiert:\seFootcite{Vgl.}{S.118}{Daroczi.2015}\seFootcite{Vgl.}{S.72}{Gunther.2014}\seFootcite{Vgl.}{S. 51}{Studenmund.2014}

\begin{equation}
R^2 = 1 - \frac{\sum\limits_{i=1}^n (y_i - \hat{y}_i)^2}{\sum\limits_{i=1}^n (y_i - \overline{y})^2}= \frac{RSS}{TSS}
\label{r2}
\end{equation}

Wie in \vref{r2} zu erkennen, kann das $R^2$ auch als Verhältnis von \gls{rss} zu \gls{tss}, also der erklärten Variation zur gesamten Abweichungsqudratsumme, dargestellt werden.\seFootcite{Vgl.}{S. 48}{Studenmund.2014}\seFootcite{Vgl.}{S.119}{Daroczi.2015} Die Hinzunahme von weiteren erklärenden $x$-Variablen führt im schlechtesten Fall dazu, dass das Bestimmtheitsmaß gleich bleibt, unabhängig davon ob die zusätzlichen Variablen die Qualität des Modells verbessern. Man spricht dabei von einer \textit{Überparametrisierung} des Modells.\seFootcite{Vgl.}{S. 160}{Cleff.2008} Um diesen Fall zu vermeiden, verwendet man in der Praxis das \textit{korrigierte Bestimmtheitsmaß} $\overline{R}^2$ (\textit{engl.: adjusted $R^2$}), dass die Hinzunahme von Variablen mit geringer Erklärungskraft bestraft.\seFootcite{Vgl.}{S. 161}{Cleff.2008} Eine zusätzliche Variable sollte also nur dann aufgenommen werden, wenn der dadurch gewonnene Erklärungswert für das Modell größer als der \textit{Bestrafungsabschlag} des korrigierten Bestimmtheitsmaßes ist. Das $\overline{R}^2$ kann daher zum Vergleich von Regressionsmodellen mit unterschiedlicher Anzahl von unabhängigen Variablen herangezogen werden, um die Anpassungsgüte des Modells an die Daten zu messen.\seFootcite{Vgl.}{S. 56}{Studenmund.2014} Die ursprüngliche Interpretation von $R^2$ geht jedoch durch Bestrafung der Hinzunahme weiterer Parameter weites gehend verloren, sodass beide Bestimmtsheitsmaße für eine Bewertung herangezogen werden sollten.\seFootcite{Vgl.}{S. 161}{Cleff.2008} 

In diesem Kontext spricht man in der Fachsprache auch von \textbf{Overfitting}, einer Überanpassung des Modells durch Hinzunahme irrelevanter Variablen, die zu einer unnötigen Steigerung der Komplexität führen. Das Modell scheint dabei für die vorliegenden Daten exakt zu passen, scheitert jedoch bei der Prognose von noch ungesehenen Daten. \textbf{Underfitting} ist die gegenteilige Bezeichnung und beschreibt ein zu simpel gewähltes Modell, welches zu wenig relevante Regressoren enthält und relativ schlecht an die Daten angepasst ist.\seFootcite{Vgl.}{S. 470}{Cios.2007} 







\chapter{Analysephase}

\section{Expected Goals}
Diese Passage soll dem Leser den aktuellen Forschungsstand der \textit{Expected Goals} vermitteln, deren Bedeutsamkeit für den Fußballsport dabei explizit aufzeigen, sowie den Einfluss von Data-Mining-Methoden hinsichtlich der Wissensgewinnung darstellen.

\begin{quote}
\textit{\glqq Expected Goals - Das angesagteste Statistikmodell im Profifußball\grqq}
\end{quote}

So betitelt Nils Nordmann seinen Online-Artikel im Interview mit Dustin Böttger, Geschäftsführer von \gls{gsn}, einem der gefragtesten Datenanalysten aus Deutschland, der mit mehreren Bundesligavereinen in Kooperation steht.\seFootcite{}{}{NilsNordmann.2016} Statistische Analysen sind im Bereich des Fußballs keine Neuheit mehr, jedoch liegt der Ursprung der sportlichen Datenanalyse in einer anderen Sportart. Der amerikanische Historiker und Statistiker Bill James veröffentlichte 1977 erste Analysen zwischen geschlagenen und gefangenen Bällen im Baseball, um eine objektive Bewertung der Gesamtleistung eines Spielers aufstellen zu können. Schumaker, Solieman und Chen bezeichnen diese Entwicklung als eine Art \glqq Revolution\grqq-- einen Wandel von traditionellen Statistiken hin zum Wissensmanagement.\seFootcite{Vgl.}{S.36}{Schumaker.2010} Diese löste eine Welle der Kreation neuer Maßzahlen aus, wovon einige im Jahr 2002 von der amerikanischen Baseball Profimannschaft \textit{Oakland A’s Billy Bean} als Grundlage zur Zusammenstellung eines neuen Teams dienten. Die \textit{Boston Red Sox} ließen sich von dieser Idee inspirieren und  gewannen anschließend sogar 2004 und 2007 die Meisterschaft.\seFootcite{Vgl.}{S.36}{Schumaker.2010} Auch aus anderen Sportarten gibt es vergleichbare Beispiele, wie die Revolution im Basketball im Jahr 1980 durch den Statistiker Dean Oliver, der neue Messwerte zur Beurteilung von Spielern veröffentlichte.\footnote{Dean Oliver beratete 2005 die \textit{Seatlle Supersonics} und verhalf zur amerikanischen Meisterschaft}

Waren im Fußball in der Vergangenheit noch rein quantitative \glspl{kpi} wie der Ballbesitz, die Passquote oder die Anzahl der Torschüsse von Bedeutung, wird das Spiel heutzutage bis in das kleinste Detail (z.B. die Anzahl der vertikal \glqq überspielten\grqq~Gegenspieler durch einen Pass) analysiert. Durch den Fortschritt der Videotechnik können alle Aktionen eines Spieles aufgezeichnet werden, wodurch sich neue stichhaltige Bewertungsmethoden heraus kristallisiert haben. Sumpter, Anderson und weitere Fachexperten untersuchen mit Hilfe von Mathematik und Statistik das Spiel und stellen in ihren Ausführungen einige Thesen und Modelle auf.\seFootcite{Vgl.}{}{Sumpter.2016}\seFootcite{Vgl.}{}{Anderson.2014}\seFootcite{Vgl.}{}{Heuer.2010} Eines der momentan angesagten Modelle sind die \glqq \textbf{Expected Goals}\grqq (\textit{dt. die zu erwartenden Tore}), welche die Qualität von Torschüssen vielseitig, objektiv und plausibel misst.\seFootcite{}{}{NilsNordmann.2016} Dazu wird jedem Schuss, unter der Berücksichtigung von Parametern (wie beispielsweise der Position oder des Körperteils mit dem geschossen wurde), eine bestimmte Erfolgswahrscheinlichkeit zugewiesen. Die Bestimmung der Wahrscheinlichkeit, die Auswahl der einbezogenen Schüsse wie auch Parameter, als auch das gesamte Modell wird öffentlich von den Analytikern (meist aus Unternehmen der Sportanalyse/-beratung) kurz ausgeführt oder gar komplett geheimgehalten. Einblicke in ihre Modellierungen bieten unter anderem Opta Sports\seFootcite{Vgl.}{}{PhilippObloch.2015}, der TV-Sender Sky Sports,\seFootcite{Vgl.}{}{PhilippErtl.2016} oder Experten, wie Michael Caley, in ihren Internetpublikationen.\seFootcite{Vgl.}{}{MichaelCaley.2017} Ein \textit{Expected-Goal-Modell} offeriert eine statistisch belegte und damit objektive Bewertung von Schüssen und bildet einen neuen \gls{kpi} über die Qualität einer Torchance. Anhand dieser Grundlage ist es möglich, weitere Bewertungsmethoden für Spieler und Mannschaften zu ermitteln, die vor allem im Scouting-Bereich ihre Anwendung finden. Durch die qualitative Bewertung der Schüsse eines Stürmers mittels des Expected-Goal-Modells, kann eine objektive Aussage über dessen Erfolgsquote getroffen werden (beispielsweise ob diese über den erwarteten Tore liegt), welche dann zur Spielersuche herangezogen werden kann. Eine Gefahr in der Modellierung der Expected Goals stellt die \textit{Überparametrisierung} (vgl. \vref{bhm}) dar. Werden zu viele Parameter, z.B. welcher Spieler geschossen hat und ob mit seinem starken oder schwachen Fuß geschossen wurde, seine Tagesform, die Leistung des generischen Torhüters, usw. für die Modellierung herangezogen, verliert das Modell durch zu viele Details seine Abstraktion und folglich die allgemeine Aussagekraft für alle Schüsse. Die Kunst liegt in der \vref{bhm} beschriebenen Balance von \textit{Underfitting} und \textit{Overfitting} des Modells.

Durch die Professionalisierung der Datenaufnahme im Fußball werden stetig mehr Daten während eines Spieles erhoben\footnote{beispielsweise durch Videobildverarbeitung oder Sensordaten}, woraus im Laufe einer Saison eine Datenmenge resultiert, die die Leistungsfähigkeit herkömmlicher Analysewerkzeuge übersteigt. Um wertvolle Informationen aus den umfangreich Daten zu extrahieren, greifen auch Datenanalysten im Bereich des Fußball auf die Prozesse und Methoden des Data Minings zurück. Ausführliche Einblicke in die Komplexität der Datenanalyse im Sport stellen unter anderem Schumaker et al. in ihrer Ausarbeitung vor.\seFootcite{Vgl.}{}{Schumaker.2010} Data-Mining-Methode wie das Clustering zur Einteilung von Spielertypen, die Regressionanalyse zur Ermittlung von Erfolgsfaktoren einer Saison, Entscheidungsbäume zur Bestimmung des perfekten Ein- und Auswechslungszeitpunktes, als auch Neuronale Netze zur Prognose von Spielausgängen, werden zur Wissensgewinnung verwendet.\seFootcite{Vgl.}{}{GunjanKumar.2013} Darüber hinaus werden einige dieser Techniken zur komplexen Erkennung von Taktiken und Spielphilosophien eingesetzt, welche in der Ausführung von Rein konkretisiert werden.\seFootcite{Vgl.}{}{Rein.2016}




\section{Opta-Spieldaten}
test\seFootcite{Vgl.}{S.1}{OptaSports.2017a}
test\seFootcite{Vgl.}{S.1}{OptaSports.2017b}

\section{Wirtschaftliche Betrachtung}
Im Kontext dieser Arbeit sollen innerhalb des folgenden Kapitels unter wirtschaftlichen Betrachtung einige Ansätze in Grundzügen aufgezeigt werden, welche Konsequenzen aus der Sichtweise der SAP als Unternehmen und der eines Fußballvereins stecken.\enlargethispage{2\baselineskip}  Aufgrund vertraulicher Daten werden für die Beispielkalkulationen fiktive Daten herangezogen.

\paragraph{Sichtweise des Unternehmens}

\begin{figure}[H]
\centering
\includegraphics[scale=0.575]{se-wa-jpg/sportsone}
\caption{Überblick über Sports One}
\label{sportsone}
\end{figure}

Sports One ist eine von SAP, speziell für die Sportbranche entwickelte Software-Lösung und bietet derzeit Funktionen in den Bereichen Team Management, Spiel- und Trainingsanalysen, Spielerfitness, Leistungsdiagnostik und Scouting, welche im Gesamtkontext in \vref{sportsone} dargestellt sind. Das Modell der \textit{Expected Goals} würde, wie in \vref{goals} hingewiesen, innerhalb des Scoutingsbereich integriert werden, wodurch Offensivspieler in einem neuen Maßstab bewertet werden können. Für die Entwicklung dieses zusätzlichen Features für den Kunden entstehen unterschiedliche Kosten, die nachfolgend exemplarisch aufgezeigt werden. Die Entwicklung des grundlegenden mathematischen Modelles zur Berechnung der Wahrscheinlichkeit eines Torerfolges, ohne Integration in das bestehende Produkt, nahm in etwa 50 Werktage und eine Mitarbeiterkapazität in Anspruch. Bei einem fiktiven internen Tagessatz von \textsf{1.500\euro} ergeben sich dadurch \textit{Erstellungkosten des Grundmodelles} in Höhe von \textsf{75.000\euro}. Die Spieldaten werden dazu vom Datenprovider Opta Sports (vgl. \vref{opta}) über einen Lizenzvertrag jährlich eingekauft, wobei diese Verträge jedoch direkt zwischen Verein und Datenprovider geschlossen werden. Sports One stellt dazu die entsprechenden Schnittstellen für den Datenimport in das System bereit. Die für das Modell zugrundeliegenden Daten wurden von Opta Sports zu Forschungszwecken kostenfrei bereitgestellt, sodass letztlich keine Kosten für den Einkauf von den Daten aus Sicht des Unternehmens entstehen. Um das Modell letztlich im Scoutingbereich von Sports One nutzen zu können, fallen weitere Entwicklungskosten an. So müsste ein entsprechende Benutzerfläche mit angebundener Anwendungslogik erstellt, sowie die Schnittstelle für den Datenimport für das Modell bereitgestellt werden. Ein Team von drei Mitarbeitern würde bis zur Auslieferung des Features schätzungsweise drei Monate (ca. 70 Werktage) benötigen, wodurch zusätzliche \textit{Erstellungskosten der Anwendung} in Höhe von \textsf{315.000\euro} ergeben. Nach der Auslieferung dieses Features an den Kunden entstehen geschätzte jährliche \textit{Support- und Wartungskosten} von \textsf{100.000\euro}. Des Weiteren muss innerhalb der Beratung und des Verkaufs der Gesamtlösung das neue Feature propagiert werden, wodurch weitere jährliche \textit{Kosten der Beratung und des Verkaufs} in Höhe von \textsf{100.000\euro} anfallen. \vref{calc} fasst dazu alle anfallenden Kosten zusammen und gibt einen Ausblick auf die nächsten Jahre.

\begin{table}[H]
\centering
\caption{Exemplarische Kalkulation}
\label{calc}
\begin{tabular}{|l|r|r|r|r|}
\hline
\multicolumn{1}{|c|}{\textbf{Kosten}} & \multicolumn{1}{c|}{\textbf{Jahr 1}} & \multicolumn{1}{c|}{\textbf{Jahr 2}} & \multicolumn{1}{c|}{\textbf{Jahr 3}} & \multicolumn{1}{c|}{\textbf{Jahr 4}} \\ \hline
Erstellungskst. d. Grundmodelles      & 75.000\euro                              & 0\euro                                   & 0\euro                                   & ...                                  \\ \hline
Erstellungskst. d. Anwendung          & 315.000\euro                             & 0\euro                                   & 0\euro                                   & ...                                   \\ \hline
Support- und Wartungskosten           & 100.000\euro                             & 100.000\euro                             & 100.000\euro                             & ...                                  \\ \hline
Kosten der Beratung \& Verkauf        & 100.000\euro                             & 100.000\euro                             & 100.000\euro                             & ...                                  \\ \hline
Summe                                 & \underline{590.000\euro}                             & \underline{200.000\euro} & \underline{200.000\euro}                            & ...                                  \\ \hline
\end{tabular}
\end{table}

In Jahr 1 fallen durch die einmalige Entwicklung des Modelles und der Anwendung, sowie der Integration in das Gesamtprodukt hohe Kosten an, die durch die Investition geschuldet sind. In den darauf folgenden Jahren entstehen lediglich Kosten durch der Beratung und des Verkaufs sowie durch den Support und die Wartung des Modelles. Dieses Feature verlangt keine Vereins-spezifischen Konfigurationen, sodass beispielsweise bei einer Anzahl von 40 Kunden, die Kosten des Modelles ab Jahr 2 pro Kunde sich jährlich auf \textsf{5.000\euro} belaufen, die wiederum durch den Verkauf des Gesamtproduktes (als jährliche Lizenz) ausreichend gedeckt werden. 


\paragraph{Sichtweise des Vereins}
Mit Hilfe eines solcher integrierten Anwendung im Scoutingbereich ergeben sich aus der Sicht eines Fußballvereines einige wirtschaftlichen Konsequenzen. Neben den Linzenzkosten des Sport One Produktes müssen auch Spieldaten über einen Datenprovide, wie Opta Sports, zu eingekauft werden, wodurch sich zunächst Ausgaben ergeben. Durch die Nutzung des Features der \textit{Expected Goals} und der dadurch resultierenden objektiven Bewertung eines Spielers, können im Gegenzug einige Kosten eingespart werden. Beispielsweise wird für die Suche nach einem neuen Mittelstürmer einige Scouts beauftragt, die auf der einen Seite dafür bezahlt werden müssen, auf der anderen durch die Fahrtkosten auf die entsprechenden Spiele auch weitere Kosten verursachen. Durch die Vorselektion der Spieler anhand dieses neuen KPIs, wird die Auswahl der in Frage kommenden Spieler deutlich resultiert und somit folgich auch die Anzahl der benötigten Scouts, sodass Kosten eingespart werden. Letztlich muss ein Spieler jedoch immer vor einer Verpflichtung subjektiv eingeschätzt werden, da dieser auch beispielsweise von der Persönlichkeit in die Teamstruktur passen muss, was durch objektive Kennzahlen nicht gemessen werden kann. Des Weiteren können dadurch verborgene Talente entdeckt werden, die in einem frühen Stadium noch zu einem geringen Preis verpflichtet werden und einige Jahre später durch einen Verkauf an einen internationalen Top-Verein Millionen einbringen können, wodurch sich die Investition in diese Software-Lösung ausreichend gelohnt hätte. Wie in den genannten Beispielen aus dem Bereich des Baseballs und des Basketballs (vgl. \vref{goals}) ist es im Fußball nicht möglich eine Mannschaft, bestehend aus elf Spielern, anhand dieser einzelnen Kennzahl vollständig zusammen zu stellen, da in einem solchen Mannschaftssport weitere Faktoren, wie taktisches Verständnis oder Teamfähigkeit, eine wichtige Rolle einnehmen. 
\chapter{Umsetzung}
\label{umsetzung}

Die Umsetzung der vorliegenden Problemstellung wird mit Hilfe des in \vref{kdd} vorgestellten KDD-Prozesses durchgeführt. Anhand der erlernten Grundlagen und Methoden der einzelnen Prozessschritte, werden diese sukzessive durchlaufen, um eine möglichst exakte Modellierung der Funktion zu realisieren. Zunächst werden die relevanten Daten in \vref{ds} selektiert, anschließend aufbereitet (vgl. \vref{dv}) und in das für die Regressionsanalyse passende Format transformiert (vgl. \vref{dt}). Im darauf folgenden Schritt des Data Minings wird die Funktion unter der Betrachtung des Winkels und der Distanz, als auch in Bezug auf die Koordinaten des Schusses anhand der Regression (vgl. \vref{fm}) modelliert. Abschließend werden die aus \gls{matlab} gewonnenen Ergebnisse interpretiert und evaluiert, um eine fundierte Entscheidung über die Auswahl eines Modells treffen zu können.

\section{Datenselektion}
\label{ds}

Opta Sports erfasst in einem Fußballspiel zwischen 1600 und 2000 verschiedene Events (z.B. Pässe, Schüsse, Fouls, uvm.), wovon eine geeignete Datenmenge für die Funktionsmodellierung ausgewählt werden muss. Zunächst werden die im \gls{xml}-Format vorliegenden Daten eines Spieles für eine bessere Weiterverarbeitung in ein \gls{json}-Format geparst. Über die in \vref{opta} beschriebenen \textsf{Type-IDs} und \textsf{Qualifiers} können die relevanten Schüsse in dieser großen Datenmenge selektiert werden. Aus der \vref{tab:events} lassen sich alle Schussversuche innerhalb eines Spiels identifizieren, wobei die Zieldaten nur Schüsse beinhalten dürfen, die während des \glqq freien\grqq~Spiels abgeben wurden, nicht aus Eigentoren resultierten und welche die nicht geblockt wurden (vgl. \vref{tab:anf}). Der Ausschluss solcher Schüsse erfolgt über die in \vref{tab:quali} gelisteten Qualifier. Die Eigentore wurden nicht von vornherein ausgeschlossen und konnte erst durch die Visualisierung aller Torerfolge in \vref{owngoals} als irrelevant erkannt werden, da diese die Wahrscheinlichkeit ein Tor aus dieser Position zu erzielen total verzerren. \enlargethispage{2\baselineskip} Das \glqq Eigentor des Jahres\grqq\seFootcite{Vgl.}{}{FrankfurterAllgemeineZeitung.2014} von Christopher Kramer aus 45-Meter kann beispielsweise ohne Angabe des Qualifiers auch als sehr weiten Fernschuss interpretiert werden.


\begin{sidewaysfigure}[H]
\centering
\includegraphics[scale=0.4]{se-wa-jpg/owngoals}
\caption[Ausschluss der Eigentore]{Ausschluss der Eigentore (rote Punkte) }
\label{owngoals}
\end{sidewaysfigure}

%\begin{figure}
%\centering
%\includegraphics[scale=0.28]{se-wa-jpg/owngoals}
%\caption[Ausschluss der Eigentore]{Ausschluss der Eigentore}
%\label{owngoals}
%\end{figure}

\subsection{Datenvorverarbeitung}
\label{aufbereitung}
\glqq Da die Zieldaten aus den Datenquellen lediglich extrahiert werden, ist im Rahmen der Datenvorverarbeitung die Qualität des Zieldatenbestandes zu untersuchen und – sofern nötig – dieser durch den Einsatz geeigneter Verfahren zu verbessern.\grqq\seFootcite{}{S. 9}{Cleve.2014}

Diese essentielle Phase verfolgt das Ziel, die unstrukturierten und zunächst nutzlos scheinenden, selektierten Rohdaten, in qualitativ hochwertigere Daten umzuwandeln, um diese der passenden \gls{dm}-Methode in einem geeigneten Format bereitstellen zu können. Die Struktur und das Format müssen perfekt auf die vorliegende Aufgabe passen, ansonsten führt die geringe Qualität der Daten zu schlechten bzw. falschen Resultaten, bis hin zu Laufzeitfehlern.\seFootcite{Vgl.}{S. 10-11}{Garcia.2015} Es gilt auch hier das Prinzip: GIGO – garbage in, garbage out.\seFootcite{Vgl.}{S. 197}{Cleve.2014} Die oftmals schlechte Qualität der (Roh-)Daten ist durch \textit{fehlende, ungenaue, inkonsistente bzw. widersprüchliche} Daten zu begründen.\seFootcite{Vgl.}{S. 84}{Han.2012}\seFootcite{Vgl.}{S. 196}{Cleve.2014} Im Folgenden werden dazu einige Ursachen beispielhaft aufgeführt.

Ungenaue bzw. falsche Daten können schon bei der Erhebung entstehen, wenn ein falsches Datenerhebungsinstrument ausgewählt wird. Bei Stichproben sollte die Gesamtmenge so präzise wie möglich widergespiegelt werden, um die Datenakkuratesse nicht zu gefährden.\seFootcite{Vgl.}{S. 25}{Fahrmeir.2007} Weiterhin können technische und menschliche Fehler zu ungenauen Daten führen, indem Personen beispielsweise ihre persönlichen Informationen bei einer Befragung absichtlich verschleiern (z.B. Standardwert für Geburtsdatum 1. Januar), wobei man diese Problematik auch als \textit{\glqq disguised missing data\grqq}~bezeichnet.\seFootcite{Vgl.}{S. 24}{Fahrmeir.2007}\seFootcite{Vgl.}{S. 196}{Cleve.2014} Neben der falschen subjektiven Einschätzung des Menschen bei der Erhebung, können auch von einem technischem Blickwinkel ungenaue Daten ermittelt werden, wie z.B. durch (teils-)defekte Sensoren.\enlargethispage{\baselineskip}  Nicht zuletzt können Daten bei einem Transfer verfälscht werden bzw. sogar verloren gehen.\seFootcite{Vgl.}{S. 84}{Han.2012}

Fehlende Daten lassen sich einerseits durch technische Mängel begründen, andererseits auch durch die Tatsache, dass bestimmte Attribute schlichtweg von Beginn an bei der Erhebung nicht beachtet wurden oder durch bestimmte Restriktionen nicht verfügbar waren.\seFootcite{Vgl.}{S. 84-85}{Han.2012}

Die aufgezeigten Beispiele spiegeln nur einen kleinen Teil möglicher Ursachen wider und sollen die Bedeutsamkeit dieser Phase für den Data-Mining-Prozess aufzeigen. Die Datenvorbereitung stellt dabei einige leistungsstarke Werkzeuge zur Verfügung, um die Datenqualität nachhaltig zu verbessern:\seFootcite{Vgl.}{S. 11 ff}{Garcia.2015}\seFootcite{Vgl.}{S. 196 ff}{Cleve.2014}\seFootcite{Vgl.}{S. 84 ff}{Han.2012}

\begin{itemize}
\item \textbf{Data Cleaning}: In diesem Schritt werden die Daten bereinigt, indem beispielsweise \textit{fehlerhafte} oder \textit{störende} Daten korrigiert werden (siehe \vref{dc}).

\item \textbf{Data Integration}: Diese Phase beschäftigt sich mit der fehlerfreien Zusammenführung von Daten, da diese oftmals aus mehreren unterschiedlichen Quellen stammen (siehe \vref{di}).

\item \textbf{Data Reduction}: Um die Algorithmen der Data-Mining-Methoden nutzen zu können, muss die exorbitante Datenmenge reduziert bzw. komprimiert werden, sodass lange Laufzeiten verhindert beziehungsweise reduziert werden können (siehe \vref{dr}).
\end{itemize}

Auf die in \vref{werkzeuge} vereinfacht dargestellten Werkzeuge und ihre Konzepte, wird in den folgenden Unterkapiteln näher eingegangen.

\begin{figure}[H]
\centering
\includegraphics[scale=1.2]{se-wa-jpg/preprocessing}
\caption[Werkzeuge der Datenvorverarbeitung]{Werkzeuge der Datenvorverarbeitung\protect\footnotemark}
\label{werkzeuge}
\end{figure}
\footnotetext{Vgl. Abbildung \textit{Han}, Data Mining: Concepts and techniques, 2012, S. 87}


\subsubsection{Data Cleaning}
\label{dc}
In der realen Welt sind Daten häufig \glqq unvollständig, mit Fehlern oder Ausreißern behaftet oder sogar inkonsistent.\grqq\seFootcite{}{S. 199}{Cleve.2014} Um Fehler oder gar falsche Resultate im Data-Mining-Prozess frühzeitig zu vermeiden, ist es von großer Bedeutung, die Datenmengen zu bereinigen. Der Fokus sollte hierbei auf der Informationsneutralität liegen. Das bedeutet, es sollen möglichst keine neuen Informationen hinzugefügt werden, die das reale Abbild verzerren oder verfälschen können.\seFootcite{Vgl.}{S. 199-200}{Cleve.2014} Folgende Problemarten gilt es zu behandeln:

\paragraph{Fehlende Daten} 
Dem Datenanalyst stehen einige Möglichkeiten zur Verfügung, um auf fehlende Daten zu reagieren:\seFootcite{Vgl.}{S. 88-90}{Han.2012}\seFootcite{Vgl.}{S. 200-201}{Cleve.2014}

\begin{itemize}
\item \textit{Attribut ignorieren}
\\ Der Datensatz mit dem fehlenden Attribut wird gänzlich ignoriert oder gelöscht. Jedoch können dadurch wichtige Informationen für die Datenanalyse verloren gehen, wodurch dieses Verfahren nur bei Datensätzen mit mehreren Lücken angewandt werden sollte.

\item \textit{Manuelles Einfügen}
\\ Besitzt der Datenanalyst das nötige Wissen, kann dieser einzelne Datensätze nachträglich manuell einfügen. Dieser Vorgang entwickelt sich schnell zu einem sehr zeitaufwändigen und schwer zu realisierenden Vorgang, der aufgrund des Mangels an Ressourcen (personeller wie auch zeitlicher) nicht zu realisieren ist, sobald die Datenmenge wächst (z.B. Nachtrag von 500 Kundendaten per Hand).

\item \textit{Globale Konstante}
\\ Den fehlenden Wert durch eine globale Konstante (auch als Standardwert bezeichnet) zu ersetzen, ist sinnvoll, wenn auch ein leeres Feld als Information angesehen wird. Beispiele für Konstanten wären \textit{unbekannt} oder \textit{minus unendlich}.

\item \textit{Durchschnittswert}
\\ Handelt es sich bei dem fehlenden Attribut um einen metrischen Wert, so kann der Durchschnittswert aller Einträge als Ersatz verwendet werden. Der Durchschnittswert bietet sich als äußerst einfache Möglichkeit, wenn die Daten klassifiziert werden können und die Berechnung nur auf Datensätzen derselben Klasse angewandt wird. Die Methode der \gls{knn}\seFootcite{Vgl.}{S. 76}{Garcia.2015} steht zur Verfügung, wenn keine Klassen vorhanden sind. Hierbei wird der Durchschnitt, der dem aktuellen Datensatz ähnlichsten Werte benutzt.

\item \textit{Wahrscheinlichster oder häufigster Wert}
\\ Durch statistische Methoden kann der wahrscheinlichste Wert für das fehlende Attribut ermittelt werden, jedoch sollte diese Angleichung begründet sein. Bei nicht numerischen Werten kann als weitere Möglichkeit auch der häufigste Wert als Ersatz für das fehlende Attribut verwendet werden.
\end{itemize}

\paragraph{Verrauschte Daten und Ausreißer}
\label{outlierchapter}
Durch ungenaue Messwerte oder falsche Schätzungen entstehen die sogenannten \textit{verrauschten Daten}.\footnote{Im englischen Sprachgebrauch als \textit{noisy data} bekannt.} Um diese bereinigen zu können, stehen dem Datenanalyst einige Verfahren zur Verfügung, durch welche diese fehlerbehafteten Daten angeglichen werden können.\footnote{Auch als \textit{smoothing} bekannt.} Als \textit{Ausreißer} bezeichnet man dabei Daten, die erheblich von den anderen Daten abweichen oder außerhalb eines Wertebereiches (z.B. $0<x<100$) liegen.\seFootcite{Vgl.}{S. 89-90}{Han.2012} Beispielsweise liegen bei einer Befragung Daten von 30- bis 50-Jährigen vor, darunter jedoch auch eine 90-jährige Person. Hierbei könnte es sich um einen Ausreißer handeln, aber auch um einen fehlerhaften Datensatz.\seFootcite{Vgl.}{S. 196}{Cleve.2014} \glqq Ob solche Ausreißer für das Data Mining ausgeblendet oder adaptiert werden sollten oder besser doch im Originalzustand zu verwenden sind, hängt vom konkreten Kontext ab.\grqq\seFootcite{}{S. 196}{Cleve.2014}


\begin{itemize}

\item \textit{Klasseneinteilung (bining)}
\\ Durch die Gruppierung verrauschter Daten in Klassen, können diese beispielsweise durch den Mittelwert oder die naheliegenden Grenzwerte ersetzt werden. 

\item \gls{regression}
\\ Die Darstellung der Daten in Form einer mathematischen Funktion, bietet die Möglichkeit, fehlerbehaftete Daten  durch die berechneten Funktionswerte zu ersetzen. Für zwei Abhängigkeiten zwischen zwei Attributen steht hierbei neben der \textit{linearen Regression}, auch die \textit{multiple lineare Regression} für mehrere Attribute als Werkzeuge zur Verfügung (weiterführende Ausarbeitung zur Regressionsanalyse siehe \vref{ra}). 

\item \textit{Verbundbildung (\gls{clustering})}
\\ Eine der einfachsten Möglichkeiten um Ausreißer zu erkennen, bietet die Verbundbildung, auch \gls{clustering} genannt. Hierbei werden ähnliche Daten, wie in \vref{outlier} dargestellt, zu \textit{Clustern} zusammengeführt, wodurch sich die Ausreißer direkt identifizieren lassen.

\begin{figure}[H]
\centering
\includegraphics[scale=1.2]{se-wa-jpg/outlier}
\caption[Outlierdetection mittels Clustering]{Outlierdetection mittels Clustering\protect\footnotemark}
\label{outlier}
\end{figure}
\footnotetext{Vgl. Abbildung \textit{Han}, Data Mining: Concepts and techniques, 2012, S. 91}
\end{itemize}

\paragraph{Falsche und inkonsistente Daten}
Bei falschen bzw. inkonsistenten Daten ergeben sich prinzipiell zwei Möglichkeiten zur Korrekturbehandlung. Einerseits können der Datensatz oder bestimmte Attribute durch \textit{Löschen} entfernt werden, wobei jedoch die Gefahr einer zu großen Reduktion des Datenbestandes entsteht und relevante Informationen für das Data Mining verloren gehen könnten. Die zweite Korrekturvariante versucht den inkonsistenten Datensatz, durch die \textit{Zuhilfenahme anderer Datensätze}, sinnvoll zu ersetzen. Sollte eine Unterscheidung zwischen \textit{falsch} und \textit{richtig} nicht möglich sein, wären beim Löschen immer mindestens zwei Datensätze betroffen.\seFootcite{Vgl.}{S. 203-204}{Cleve.2014}

\subsubsection{Data Integration}
\label{di}
Bei Data-Mining-Projekten ist oftmals die Integration mehrerer Datenbestände aus unterschiedlichen Quellen erforderlich. Diese Phase sollte mit äußerster Sorgfalt durchgeführt werden, um frühzeitig redundante und inkonsistente Datensätze zu vermeiden, wodurch die Genauigkeit und Geschwindigkeit der nachfolgenden Data Mining Algorithmen nicht gefährdet wird.\seFootcite{Vgl.}{S. 93-94}{Han.2012}\enlargethispage{\baselineskip} Folgende Punkte gilt es bei der Datenintegration zu beachten:

\begin{itemize}
\item \textbf{Identifikationsproblem von Entitäten}:
\\ Bei der Datenintegration aus multiplen Datenquellen, wie beispielsweise Datenbanken oder Dokumenten, stellt die Schema-Integration wie auch die Objektanpassungen eine schwierige Herausforderung dar. Der Datenanalyst muss sicherstellen, dass zum Beispiel das Attribut \textit{kunden\_nummer} aus der einen Datenquelle, dieselbe Referenz besitzt, wie das Attribut \textit{kunden\_id} aus einer anderen und es sich folglich um dasselbe Attribut handelt. Dies wird allgemein als \textit{Entity Identification Problem} bezeichnet.\seFootcite{Vgl.}{S. 199}{Cleve.2014}\seFootcite{Vgl.}{S. 94}{Han.2012} Die Metadaten der Attribute beinhalten Informationen, wie \textit{Name, Bedeutung, Datentyp, Wertebereich,} uvm. und können durch Abgleich derer zu einer Vermeidung von Fehlern bei der Integration beitragen. Weiterhin muss gesondert auf die \textit{Datenstruktur} geachtet werden, um keine referentiellen Abhängigkeiten bzw. Beziehungen zwischen den Daten zu zerstören.\seFootcite{Vgl.}{S. 94}{Han.2012} 

\item \textbf{Redundanzen bei Attributen}:
\\ Ein Attribut, welches durch ein anderes Attribut ableitbar ist -- wie zum Beispiel das Alter vom Geburtsjahr berechnet werden kann -- wird als redundant bezeichnet. Die Vielzahl von Redundanzen führt zu unnötig großen Datenmengen, die wiederum die Performanz sowie die Resultate eines Data Mining Algorithmus negativ beeinträchtigen können.\seFootcite{Vgl.}{S.41}{Garcia.2015} Folglich sollte diese Problematik durch die Anwendung von statistischen Verfahren, in Form der Korrelationsanalyse, dezidiert behandelt werden. Für numerische Werte ist dabei der Einsatz von Korrelationskoeffizienten und Kovarianzen hilfreich. Um die Implikation zweier Attribute einer nominalen Datenmenge\footnote{Rein qualitative Merkmalsausprägungen ohne natürliche Rangordnung (wie z.B. das Geschlecht).} bestimmen zu können, verwendet man in der Regel den $\chi^2$(\textit{Chi-Quadrate})-Test.\seFootcite{Vgl.}{S. 95.}{Han.2012}\seFootcite{Vgl.}{S. 41}{Garcia.2015}\seFootcite{Vgl.}{S. 64}{Cleve.2014} 

\item \textbf{Duplikatserkennung}:
\\ Duplikate verkörpern Redundanzen auf Datensatzebene und führen einerseits zu unnötig großen Datenmengen, die sich wiederum auf die Performanz der Algorithmen auswirken. Andererseits führt jedoch auch die verfälschte Gewichtung der mehrfach vorkommenden Datensätze, zu falschen Analyseergebnissen.\enlargethispage{\baselineskip} Ein häufiger Grund stellt dabei die Verwendung von denormalisierten Datenbanktabellen dar.\seFootcite{Vgl.}{S. 98}{Han.2012}\seFootcite{Vgl.}{S. 43}{Garcia.2015}

\item \textbf{Konflikte bei Attributswerten}:
\\ Hierbei handelt es sich um die unterschiedliche Darstellung, Skalierung und Kodierung von Attributswerten. Beispielsweise kann das Attribut \textit{Gewicht} durch das metrische System oder das britische Maßsystem repräsentiert werden, woraus bei der Zusammenführung von Daten in eine einheitliche Quelle immer wieder Konflikte resultieren.\seFootcite{Vgl.}{S. 99}{Han.2012}\seFootcite{Vgl.}{S. 199}{Cleve.2014} 
\end{itemize}

\subsubsection{Data Reduction}
\label{dr}
Die bereits mehrfach angesprochene Problematik der riesigen Datenmengen bei Data-Mining-Projekten, steigert die Komplexität und vermindert die Effizienz der Algorithmen. Daher strebt die Datenreduktion -- wie die Bezeichnung erkennen lässt -- nach einer reduzierten repräsentativen Teilmenge, welche die Integrität des Originals nicht verliert. Dazu können folgende drei Techniken angewandt werden:\seFootcite{Vgl.}{S. 147 ff}{Garcia.2015}\seFootcite{Vgl.}{S. 99-100}{Han.2012}\seFootcite{Vgl.}{S. 206-208}{Cleve.2014} 

\begin{enumerate}

\item Dimensionsreduktion 

\item Datenkompression

\item Numerische Datenreduktion

\end{enumerate}

\paragraph{Dimensionsreduktion}
Hierbei bleiben irrelevante Attribute des Datensatzes unberücksichtigt und nur für die Analyse relevante Daten werden miteinbezogen. Allgemein empfehlen sich dafür zwei Verfahren: Bei der schrittweisen \textit{Vorwärtsauswahl} werden wesentliche Attribute einer sukzessiv wachsenden Zielmenge zugeordnet. Im Gegensatz dazu werden bei der \textit{Rückwärtseliminierung} die uninteressanten Daten schrittweise aus der Zielmenge eliminiert.\seFootcite{Vgl.}{S. 206}{Cleve.2014} 

\paragraph{Datenkompression}
\label{datenkompression}
\enlargethispage{2\baselineskip}Bei dieser Technik wird durch Transformation oder Codierung versucht, eine Reduktion der Datenmenge zu erreichen. Fasst man beispielsweise die einzelnen Attribute \textit{Tag, Monat und Jahr} zu einem neuen Attribut \textit{Datum} zusammen, können Datensätze komprimiert werden.\seFootcite{Vgl.}{S. 207}{Cleve.2014}
 
\paragraph{Numerische Datenreduktion}
Anstatt die gesamte Datenmenge für die Analyse heranzuziehen, wird innerhalb der numerischen Datenreduktion eine repräsentative Teilmenge -- in Form einer Stichprobe -- für das \gls{dm} genutzt. Im Vordergrund steht hierbei die passende Auswahl unterschiedlicher Stichprobeverfahren, wie der \textit{zufälligen Stichprobe} oder der \textit{repräsentativen Stichprobe}, woraus jedoch kein verzerrtes Abbild der Daten resultieren darf.\seFootcite{Vgl.}{S. 25-27}{Fahrmeir.2007}\seFootcite{Vgl.}{S. 207}{Cleve.2014}
\subsection{Datentransformation}
\label{dt}
Nachdem die (Roh-)Daten selektiert, bereinigt und auf eine relevante Zielmenge reduziert wurden, müssen diese noch in eine adaptierte Form für die Algorithmen des Data Minings transformiert werden.\seFootcite{Vgl.}{S. 112}{Han.2012} Oftmals müssen sogar neue Attribute aus einem Datensatz kreiert werden, da dieser nicht in geeigneter Struktur für das Data-Mining-Verfahren vorliegt.\seFootcite{Vgl.}{S. 48}{Garcia.2015} Dazu es gibt eine Reihe an unterschiedlichen Transformationsmöglichkeiten, wobei in dieser Arbeit ein Auszug der relevanten Methoden vorgestellt werden soll:

\paragraph{Codierung}
Liegen beispielsweise Attribute mit einer ordinalen Ausprägung vor (wie \textit{sehr groß, groß, mittel und klein}), müssen diese bei einer Verwendung des \gls{knn}-Algorithmus in numerische Werte transformiert werden (Werte zwischen 0 und 1). Hierbei würde sich folgende Codierung für das Attribut \textit{Körpergröße} anbieten:\seFootcite{Vgl.}{S. 210}{Cleve.2014}

\begin{itemize}
\item \textit{sehr groß} $\rightarrow$ 1
\item \textit{groß} $\rightarrow$ 0,66
\item \textit{mittel} $\rightarrow$ 0,33
\item \textit{klein} $\rightarrow$ 0
\end{itemize}

Die Ordnungsrelation, hier \textit{sehr groß > groß > ...}, darf dabei jedoch nicht verloren gehen. In Abhängigkeit zu dem jeweiligen Verfahren, müssen Daten, so wie dies bei Maßeinheiten immer wieder der Fall ist, oftmals kodiert werden.\seFootcite{Vgl.}{S. 211}{Cleve.2014}

\paragraph{Normalisierung und Skalierung}
Unterschiedliche Maßeinheiten -- wie \textit{Körpergröße} und \textit{Körpergewicht} -- können die Datenanalyse negativ beeinflussen und müssen daher in eine einheitliche Skalierung transformiert werden, um eine gleiche Gewichtung aller Attribute zu erreichen. Man bedient sich hierbei in der Regel an der \textit{Min-Max-Normalisierung} (siehe \vref{minmax}) oder der \textit{Z-Transformation}, um numerische Werte auf ein [0,1] Intervall zu normieren.\seFootcite{Vgl.}{S. 114}{Han.2012}\seFootcite{Vgl.}{S. 212}{Cleve.2014}

\begin{figure}[H]
\begin{equation}
x_{neu} = \frac{x - min(x_i)}{max(x_i) - min(x_i)}
\end{equation}
\caption{Min-Max-Normalisierung}
\label{minmax}
\end{figure}

\paragraph{Datenaggregation}
Nicht nur aus Sicht der Datenkompression (vgl. \vref{datenkompression}) ist die Datenaggregation erforderlich. Vielmehr \glqq kann die Aggregation aus inhaltlichen Gründen sinnvoll sein.\grqq\seFootcite{}{S. 214}{Cleve.2014} Wenn Daten auf einer zu detaillierten Ebene vorliegen -- wie beispielsweise Einwohnerzahlen von Stadtteilen -- müssen diese für einen Städtevergleich erst summiert werden, um bundesweite Aussagen treffen zu können. Je nach Kontext können verschiedene Aggregationsmethoden (wie z.B. Summenbildung, Durchschnitt, usw.) für die Transformation zu einem einzigen Wert angewendet werden.\seFootcite{Vgl.}{S. 112}{Han.2012}

\paragraph{Datenglättung}
Die bereits in \vref{outlierchapter} vorgestellten Techniken zur Bereinigung von verrauschten Daten und Ausreißern, finden auch bei der Transformation ihre Verwendung. Die Datenglättung strebt nach einer reduzierten Datenmenge, in welcher jeder numerische Wert durch einen idealisierten Wert, wie beispielsweise der \textit{Regression}, ersetzt wird.\seFootcite{Vgl.}{S. 214-215}{Cleve.2014}
\section{Data Mining}
\subsection{Funktionsmodellierung}
\subsection{Regressionsanalyse}
\subsection{MatLab}
\section{Interpretation und Evaluation der Modelle}
Die Interpretation sowie die Evaluation der Resultate des Data Minings stehen am Ende jedes KDD-Prozesses. Um eine fundierte Entscheidung für ein Modell treffen zu können, werden im Folgenden die vier vorgestellten Modellvarianten interpretiert und miteinander verglichen (siehe \vref{inter}), sowie die ausgewählte Funktion anschließend in \vref{eva} evaluiert.

\subsection{Interpretation}
\label{inter}
Anhand der in \vref{int} vorgestellten Kriterien der \textit{Validität}, der \textit{Neuartigkeit}, der \textit{Nützlichkeit}, sowie der \textit{Verständlichkeit} des Modelles, werden die vier vorgestellten und realisierten Regressionsmodelle aus \vref{umsetzung} miteinander verglichen. Für eine vereinfachte Darstellung der Erfüllung der Anforderungen werden den Modellen folgende Abkürzungen zu geordnet:

\begin{itemize}
\item Multiple Regression der Winkel- und Distanzbetrachtung $\rightarrow$ \textbf{A}
\item Nichtparam. Regression der Winkel- und Distanzbetrachtung $\rightarrow$ \textbf{B}
\item Multiple Regression der Koordinatenbetrachtung $\rightarrow$ \textbf{C}
\item Nichtparam. Regression der Koordinatenbetrachtung $\rightarrow$ \textbf{D}
\end{itemize}

Die \vref{tab:verg} bietet in Form einer Matrix einen übersichtlichen Vergleich, inwieweit jedes Modell die obengenannten Kriterien erfüllen konnte. Die \textit{Validität} setzt sich dabei einerseits aus der Bewertung der objektiven Bestimmtheitsmaßen der einzelnen Modellierungen zusammen, andererseits auch aus subjektiven Wahrnehmungen, wie die vorgestellte Problematik der Wahrscheinlichkeitsbestimmung von Schussversuchen aus spitzem Winkel (vgl. \vref{heatmap}). Dabei überzeugt einzig allein das Modell \textbf{D} mit Bestimmtheitsmaßen um die \textsf{80}\% und einer korrekten Modellierung der Schussversuche aus spitzem Winkel. Die anderen drei Modelle haben mit ca. \textsf{50}\% eine zu geringe Anpassung der Funktion an die Daten und werden demzufolge auch für zukünftige Daten nicht valide sein. Das Kriterium der \textit{Neuartigkeit} erfüllen alle Modelle, da bislang keine Modellierungen von Funktionen zur Berechnung der Wahrscheinlichkeit eines Torerfolges im Fußball existieren. Die \textit{Nützlichkeit} ergibt sich größtenteils aus der \textit{Validität}, da nur ein Modell für einen Anwender nützlich sein kann, wenn valide Werte zurückgegeben werden. Modell \textbf{B} und \textbf{D} können diese Anforderung teilweise erfüllen und ermöglichen trotz abstrakter Modellierung erste Ansätze zur Mustererkennung. Letztlich beschreibt die \textit{Verständlichkeit} inwiefern die Ergebnisse des Modells von einem anderen Anwender erfasst werden. 

%%%%%%%%%%%%%%%%%% Vergleich %%%%%%%%%%%%%%%%%%
\tablefirsthead{\hline\multicolumn{5}{|c|}{\textbf{Vergleich der Modelle}}\\\hline\hline \textbf{Kriterium} & \textbf{A} & \textbf{B} & \textbf{C} & \textbf{D}\\\hline}
\tablehead{}
\tabletail{}
\tablelasttail { \multicolumn {5}{| c |}
{\textsl { \textcolor{green}{\ding{52}} erfüllt~~~~~~~\textcolor{orange}{\ding{108}} teilweise erfüllt~~~~~~~ \textcolor{red}{\ding{54}} nicht erfüllt  }}\\\hline }
\bottomcaption{Vergleich der Modelle\label{tab:verg}}
\begin{center}%
\begin{supertabular}{ | P{4cm} | P{2cm}  | P{2cm} | P{2cm}  |  P{2cm}  |}
&&&&\\
\textit{Validität}	& \textcolor{red}{\ding{54}}	 & \textcolor{orange}{\ding{108}} & \textcolor{orange}{\ding{108}}	&\textcolor{green}{\ding{52}}	\\
&&&&\\
\hline
&&&&\\
\textit{Neuartigkeit}	& \textcolor{green}{\ding{52}}	 & \textcolor{green}{\ding{52}}	& \textcolor{green}{\ding{52}}	& \textcolor{green}{\ding{52}}	\\
&&&&\\
\hline
&&&&\\
\textit{Nützlichkeit}	&\textcolor{red}{\ding{54}}	 & \textcolor{orange}{\ding{108}} & \textcolor{orange}{\ding{108}}	& \textcolor{green}{\ding{52}}	\\
&&&&\\
\hline
&&&&\\
\textit{Verständlichkeit}	&\textcolor{red}{\ding{54}}	 & \textcolor{red}{\ding{54}}	& \textcolor{orange}{\ding{108}}	& \textcolor{green}{\ding{52}}	\\
&&&&\\
\hline
\end{supertabular}
\end{center}

\subsection{Evaluation}
\label{eva}

\chapter{Zusammenfassung}

\section{Fazit}
Im Rahmen der vorliegenden Arbeit konnte gezeigt werden, wie Methoden des Data Minings innerhalb eines strukturierten Prozesses im Bereich der Datenanalyse im Fußball eingesetzt werden können, um neues Wissen zu schaffen. Durch den systematischen Aufbau des Knowledge Discovery in Data Prozesses und den darin beinhaltenden Methoden zur Verbesserung der Datenqualität konnte aus der rohen Datenmenge, gemäß den Anforderungen, eine valide und fundierte Funktion modelliert werden, die die Wahrscheinlichkeit für einen Torerfolg eines Schusses aus jeder möglichen Position berechnet. 

Innerhalb der vorbereitenden Schritte der Datenselektion, -vorverarbeitung und -transformation wurde in der Umsetzung deutlich, wie essentiell diese Phasen für die Qualität der Resultate des Data Minings sind. So konnten die Eigentore, die zuvor kein Bestandteil der Anforderungen waren, nachträglich ausgeschlossen und Ausreißer in der Datenmenge identifiziert oder das Problem der weißen Linien aufgedeckt werden. Aus der vergleichenden Analyse der modellierten Funktionen resultiert, dass die Verwendung des \textit{nichtparametrischen} Regressionsmodelles in Form sogenannter \textit{Splines} in Kombination der Betrachtung der Koordinaten des Schusses besonders geeignet ist und zu einer genauen Anpassung des Modelles an die Daten führt. In Bezug auf die Betrachtung des Winkels und der Distanz des Schussversuches zum Tor erwies sich weder das \textit{multiple} noch das \textit{nichtparametrische} Regressionsmodell als valide. Die eindrucksvollen Ergebnisse der Evaluation beweisen, dass mit lediglich zwei Parametern, den Koordinaten des Schusses, eine gute Balance zwischen einem Under- bzw. Overfitting des Modelles gefunden werden konnte. Das Software-Tool MATLAB bestätigte des Weiteren die weltweit erstklassige Reputation mit den bereitgestellten Bibliotheken zur Regressionsanalyse, sowie deren intuitiven und benutzerfreundlichen Bedienung. \enlargethispage{2\baselineskip} 

Die erstmalige Modellierung der \textit{Expected Goals} durch eine Funktion bietet die Basis für ein datenbasiertes und objektiv bewertetes Scouting, sowie weitere nützliche Anwendungsfälle, die im nachfolgenden Ausblick in \vref{ausblick} behandelt werden.



\section{Ausblick}
\label{ausblick}

Die vorliegende erarbeitete Funktion wurde anhand der Daten aus der deutschen Bundesliga modelliert und ist somit auch zunächst lediglich für diese Fußballliga gültig. Das Modell könnte dahingehend erweitert werden, dass für jede europäische Topliga ein eigenes Modell erstellt wird, um Muster zwischen diesen Ländern erkennen zu können. In Bezug auf einen Spieler ergeben sich weitere alternative Anwendungsmöglichkeiten. Werden beispielsweise alle Schüsse eines Spielers in die Funktion eingesetzt, ergibt sich daraus die erwartete Anzahl an Toren. Dieser Wert wird mit den tatsächlich erzielten Toren verglichen, um statistisch belegte Aussagen über die Erfolgsquote des Spielers treffen zu können, welche für oder gegen dessen Qualität sprechen. Des Weiteren bietet dieses Modell die Grundlage auch Defensivaktionen, wie zum Beispiel gewonnene Zweikämpfe, in einen neuen Maßstab zu setzen, da durch den Ballbesitzwechsel die gegnerische Mannschaft aus der vorherigen Position kein Tor mehr schießen kann. Beispielsweise konnte ein Abwehrspieler die siebzig-prozentige Wahrscheinlichkeit auf einen Torerfolg aus fünf Metern Distanz mit einem gewonnenen Zweikampf verhindern, wodurch seine Kennzahl steigt. Im Gegensatz dazu würde ein verlorener Zweikampf in dieser Situation seinen KPI negativ belasten.

In Bezug auf die vorgestellte Anwendung innerhalb des Scoutingbereiches im Fußball, ergibt sich die Frage, inwieweit ein solches Modell einen Scout ersetzen beziehungsweise dessen Aufgaben automatisiert ausführen kann. Diese Frage lässt sich teilweise mit ja beantworten. Möchte ein Verein einen neuen Stürmer verpflichten und verfügt dabei über ein limitiertes Budget, so kann mit Hilfe dieses Modells eine automatisierte Vorselektion passender Spieler durchgeführt werden, wodurch die Anzahl der dafür benötigten Scouts reduziert werden kann. Eine subjektive Einschätzung des Spielers muss letztlich immer erfolgen, sodass eine vollständige Ersetzung der Funktion des Scouts auch in Zukunft ausgeschlossen werden kann. Zu viele nicht objektiv messbare Indikatoren, wie taktisches Spielverständnis oder Teamfähigkeit, spielen eine entscheidende Rolle bei der Verpflichtung eines neuen Spielers. Dieses Modell soll nicht das fußballerische Expertenwissen des Scouts ersetzen, sondern subjektive Wahrnehmungen, die jedoch durch solch eine Funktion objektiv messbar sind, validieren und alltäglich anstehende Aufgaben erleichtern. Zusätzlich können Algorithmen entworfen werden, die automatisiert auf dem Transfermarkt nach sogenannten \glqq Schnäppchen\grqq~oder verborgenen Talenten suchen, die für einen günstigen Preis dadurch verpflichtet werden können. 

\enlargethispage{\baselineskip}Nicht zuletzt dient diese Modellierung auch als Erfahrungsgrundlage für weitere zukünftige Modelle, die aus der großen rohen Datenmenge, die durch neueste Technologien im Fußball erhoben werden, die Grundlage zur Ableitung neuen Wissens schaffen.


% Anhang der Arbeit
% 
%
\seAppendix{}

\newcommand{\waInputStyles}{\texttt{se-wa-input-styles-v097.tex}}

\chapter{Opta}

\begin{figure}[H]
\centering
\includegraphics[scale=0.52]{se-wa-jpg/daten}
\caption{Auszug aus der XML-Datei mit den Events}
\label{xmldaten}
\end{figure}

Die \vref{xmldaten} zeigt einen Ausschnitt aus der XML-Datei in der alle Events eines Spiels aufzeichnet sind. Im oberen Teil sind die Mannschaftsaufstellungen zu sehen, wobei jeder Spieler eine eigene ID besitzt. Darunter folgt gelb markiert zum Anpfiff der Partie der erste Pass mit dem \textit{Outcome} 1, welcher einen erfolgreichen Pass identifiziert. 

\begin{sidewaysfigure}
\centering
\includegraphics[scale=0.3]{se-wa-jpg/lines}
\caption[Problem der weißen Linien]{Problem der weißen Linien}
\label{lines}
\end{sidewaysfigure}

\chapter{MatLab}

\begin{figure}[H]
\centering
\includegraphics[scale=0.34]{se-wa-jpg/inter}
\caption{Ausschluss der Interpolation}
\label{inter}
\end{figure}

\begin{figure}[H]
\centering
\includegraphics[scale=0.34]{se-wa-jpg/splinewdTM}
\caption{Overfitting bei einem Span-Wert von 1\%}
\label{splinewdTM}
\end{figure}

\begin{figure}[H]
\centering
\includegraphics[scale=0.34]{se-wa-jpg/splinewdTL}
\caption{Underfitting bei einem Span-Wert von 50\%}
\label{splinewdTL}
\end{figure}

\chapter{Evaluation}

Hier werden die Ergebnisse der Evaluation des nichtparametrischen Regressionsmodelles unter der Betrachtung der Position des Schusses dokumentiert. 

\begin{sidewaysfigure}
\centering
\includegraphics[scale=0.3]{se-wa-jpg/cGoals_correlation_14_15}
\caption{Evaluation der Gegentore der Saison 2014/15}
\label{lines}
\end{sidewaysfigure}

\begin{sidewaysfigure}
\centering
\includegraphics[scale=0.3]{se-wa-jpg/goals_correlation_14_15}
\caption{Evaluation der Tore der Saison 2014/15}
\label{lines}
\end{sidewaysfigure}

\begin{sidewaysfigure}
\centering
\includegraphics[scale=0.3]{se-wa-jpg/points_correlation_14_15}
\caption{Evaluation der Punkte der Saison 2014/15}
\label{lines}
\end{sidewaysfigure}

\begin{sidewaysfigure}
\centering
\includegraphics[scale=0.3]{se-wa-jpg/cGoals_correlation_15_16}
\caption{Evaluation der Gegentore der Saison 2015/16}
\label{lines}
\end{sidewaysfigure}

\begin{sidewaysfigure}
\centering
\includegraphics[scale=0.3]{se-wa-jpg/goals_correlation_15_16}
\caption{Evaluation der Tore der Saison 2015/16}
\label{lines}
\end{sidewaysfigure}

\begin{sidewaysfigure}
\centering
\includegraphics[scale=0.3]{se-wa-jpg/points_correlation_15_16}
\caption{Evaluation der Punkte der Saison 2015/16}
\label{lines}
\end{sidewaysfigure}

\begin{sidewaysfigure}
\centering
\includegraphics[scale=0.3]{se-wa-jpg/cGoals_correlation_16_17}
\caption{Evaluation der Gegentore der Saison 2016/17}
\label{lines}
\end{sidewaysfigure}

\begin{sidewaysfigure}
\centering
\includegraphics[scale=0.3]{se-wa-jpg/goals_correlation_16_17}
\caption{Evaluation der Tore der Saison 2016/17}
\label{lines}
\end{sidewaysfigure}

\begin{sidewaysfigure}
\centering
\includegraphics[scale=0.3]{se-wa-jpg/points_correlation_16_17}
\caption{Evaluation der Punkte der Saison 2016/17}
\label{lines}
\end{sidewaysfigure}





%
%  Erzeugung eines Glossars
%
% Achtung: Das Glossar wird nur ausgegeben, wenn mindestens ein Eintrag in der Arbeit 
%                definiert wurde
%
%
\newpage
\sePrintGlossary{}


%
% Literaturverzeichnisses
%
%\newpage
\sePrintBibliography{}

%\input{\seWaPathText/se-test-literaturverzeichnis}


%
% Festlegung des grundlegenden Formatierungsstils des Literaturverzeichnis
%
\bibliographystyle{jurabib}

% Eigentliche Ausgabe der in der Arbeit verwendeten Quellen
%
%
% Angabe der bib-Dateien, in denen die Quellen beschrieben sind;
% die Angabe geht davon aus, dass eine wa.bib-Datei in demselben 
% Verzeichnis liegt, wie se-ba-vorlage.tex
%

% 2016-04-01
%
% Umbenennung von Quellen- in Literaturverzeichnis (nicht empfohlen, da sich 
% die 
% 
\renewcommand*{\bibname}{Literaturverzeichnis}
\seBibliography{wa}


%
% Erzeugung der ehrenw\"ortlichen Erkl\"arung
%
% Der optionale Parameter kann verwendet werden, um f\"ur das Thema der Arbeit eine 
% andere Formatierung vorzunehmen; das sollte in der Regel nicht erforderlich sein;
% ausserdem besteht die Gefahr inkonsistenter Titel auf dem Titelblatt und in der 
% ehrenw\"ortlichen Erkl\"arung
%
\seEhrenwoertlicheErklaerung{} % dieses Kommando sollte standardm\"assig verwendet werden
%\seEhrenwoertlicheErklaerung[\LaTeX-Vorlage zur Anfertigung \seThemaWaArbeit{} (Version \version{})]


\end{document}











