% 2012-03-22 Verwendung des optionalen Parameters f\"ur die Pluralform einer Abk\"urzung
%
% 2012-02-06 Umstellung auf die neuen Kommandos
%
%
%
%  J\"org Baumgart
%  Definition einiger Abk\"urzungen
%  


% Definition von Abk\"urzungen
%
% 1. Parameter: Schluessel (key) der Abkuerzung
% 2. Parameter: Abkuerzung
% 3. Parameter: Vollform
% 4. Parameter: Vollform im Plural (optional; falls nicht definiert, wird der Wert des dritten Parameters verwendet)
%

\seNewAcronymEntry{dhbw}{DHBW}{Duale Hochschule Baden-W\"urttemberg}{}{}

\seNewAcronymEntry{usb}{USB}{Universal Serial Bus}{}

\seNewAcronymEntry{ctan}{CTAN}{Comprehensive \TeX{} Archive Network}{}

\seNewAcronymEntry{sap}{SAP}{eigenständiger Markenname - früher: \textit{Systeme, Anwendungen und Produkte in der Datenverarbeitung}}{}

\seNewAcronymEntry{mvc}{MVC}{Model-View-Controller}{}

\seNewAcronymEntry{vba}{VBA}{Visual Basic for Applications}{}

\seNewAcronymEntry{erp}{ERP}{Enterprise Resource Planing}{}

\seNewAcronymEntry{ui}{UI}{User Interface}{}

\seNewAcronymEntry{html}{HTML}{Hypertext Markup Language}{}

\seNewAcronymEntry{svg}{SVG}{Scalable Vector Graphics}{}

\seNewAcronymEntry{dom}{DOM}{Document Object Model}{}

\seNewAcronymEntry{xml}{XML}{Extensible Markup Language}{}

\seNewAcronymEntry{css}{CSS}{Cascading Style Sheets}{}

\seNewAcronymEntry{jpg}{JPEG}{Joint Photographic Expters Group}{}

\seNewAcronymEntry{png}{PNG}{Portable Network Graphics}{}

\seNewAcronymEntry{w3c}{W3C}{World Wide Web Consortium}{}

\seNewAcronymEntry{d3}{D3}{Data Driven Document}{}

\seNewAcronymEntry{kpi}{KPI}{Key Perfomance Indicator}{Key Perfomance Indicators}

\seNewAcronymEntry{oss}{OSS}{Open Source Software}{}

\seNewAcronymEntry{osi}{OSI}{Open Source Initiative}{}

\seNewAcronymEntry{c4a}{C4A}{Cloud for Analytics}{}

\seNewAcronymEntry{mit}{MIT}{Massachusetts Institute of Technology}{}

\seNewAcronymEntry{wtfpl}{WTFPL}{WHAT THE FUCK PUBLIC LICENSE}{}

\seNewAcronymEntry{json}{JSON}{JavaScript Object Notation}{}

\seNewAcronymEntry{gsn}{GSN}{Global Soccer Network}{}

%\seNewAcronymEntry{kdd}{KDD}{Knowledge Discovery in Data}{}

%\seNewAcronymEntry{dm}{DM}{Data Mining}{}

%\seNewAcronymEntry{crisp-dm}{CRISP-DM}{Cross Industry Standard Process for Data Mining}{}

%\seNewAcronymEntry{iot}{IoT}{Internet of Things}{}


%\seNewAcronymEntry{olap}{OLAP}{Online Analytical Processing}{}

%\seNewAcronymEntry{sql}{SQL}{Structured Query Language}{}

%\seNewAcronymEntry{knn}{KNN}{K-Nearest Neighbours}{}
\seNewAcronymEntry{matlab}{MATLAB}{MATrix LABoratory}{}

\seNewAcronymEntry{rss}{RSS}{Residual Sum of Squares}{}


\seNewAcronymEntry{tss}{TSS}{Total Sum of Squares}{}

\seNewAcronymEntry{cft}{CFT}{Curve Fitting Tool}{}

\seNewAcronymEntry{lowess}{Lowess}{Locally Weighted Scatter Plot Smooth}{}



\seNewAcronymEntry{dfl}{DFL}{Deutsche Fußball Liga}{}



\seNewAcronymGlossaryEntry{ml}{ML}{Machine Learning}%
{}%
{Das \textit{Lernen} umfasst viele komplexe Aspekte, welche sich nicht direkt auf Computer nachbilden lassen. Maschinelles Lernen (\textit{Machine Learning}) verfolgt durch den Entwurf von Algorithmen rechnergestützte und automatisierte Lernverfahren verfügbar zu machen, um aus Daten neues Wissen zu generieren.\seFootcite{Vgl.}{S. 13}{Cleve.2014}}

\seNewAcronymGlossaryEntry{olap}{OLAP}{Online Analytical Processing}%
{}%
{Die Prinzip wird innerhalb von Business Warehouses verwendet, um innerhalb eines multidimensionalen Datenmodells, in Form eines Datenwürfels, durch gezielte Operationen (\textit{wie Slicing und Dicing, Roll-up, Drill-Down, usw.}) Daten abzufragen.\seFootcite{Vgl.}{S. 1067}{Elmasri.2011}}

\seNewAcronymGlossaryEntry{sql}{SQL}{Structured Query Language}%
{}%
{Innerhalb relationaler Datenbanken wird die deskriptive Abfragesprache SQL verwendet. Die umfassende Funktionalität vereinigt sowohl die Erstellung und Definition von Datenbankschemata (\textit{Data Definition Language}), wie auch die Manipulierung der eigentlichen Daten (\textit{Data Manipulation Language}).\seFootcite{Vgl.}{S. 87-88}{Elmasri.2011}}

\seNewAcronymEntry{knn}{KNN}{K-Nearest Neighbours}{}

\seNewAcronymGlossaryEntry{dm}{DM}{Data Mining}%
{}%
{Es beschreibt einen Prozess des \glqq Sammelns, Säuberns, Verarbeitens und Analysierens von Daten, zur Gewinnung von nützlichen Information.\grqq\seFootcite{}{S.1}{Aggarwal.2015} Mit Hilfe von bereitgestellten Methoden und Werkzeugen, welche auf die Daten angewandt werden, können Muster aus großen Datenmenge erkannt und somit neues Wissen generiert werden.}

\seNewAcronymGlossaryEntry{iot}{IoT}{Internet of Things}%
{}%
{Eines der jüngsten und wichtigsten Forschungsgebiete der Informatik ist das Internet der Dinge (\textit{Internet of Thing}). Es vernetzt alltäglich Gegenstände über das Internet, sodass diese intelligent miteinander kommunizieren können, um den Menschen bei seinen Tätigkeiten zu unterstützen.\seFootcite{Vgl.}{S. 1}{Chaouchi.2013}}

\seNewAcronymGlossaryEntry{crisp-dm}{CRISP-DM}{Cross Industry Standard Process for Data Mining}%
{}%
{Hierbei handelt es sich um eine industrieorientierte Ausgestaltung des Data Mining Prozesses, welcher in die folgenden sechs Phasen eingeteilt ist: Verstehen der Aufgabe, Verständnis der Daten, Datenvorverarbeitung, Modellbildung, Evaluation und Einsatz im Unternehmen.\seFootcite{Vgl.}{S. 6-8}{Cleve.2014}}

\seNewAcronymGlossaryEntry{kdd}{KDD}{Knowledge Discovery in Data}%
{}%
{Hierbei handelt es sich um eine mögliche Gestaltung des Data Mining Prozesses. Dabei sind folgende fünf Phasen vorhergesehen: Datenselektion, Datenvorverarbeitung, Datentransformation, Data Mining, sowie die Interpretation und Evaluation.\seFootcite{Vgl.}{S. 3}{Runkler.2015}}

\seNewAcronymEntry{svm}{SVM}{Suppport Vector Machine}{}

\seNewAcronymEntry{nn}{NN}{Neuronale Netze}{}

\seNewAcronymGlossaryEntry{mdkq}{MDKQ}{Methode der kleinsten Quadrate}%
{}%
{Das Ziel der Regression ist die Bestimmung der Parameter der Funktion mit Hilfe der Methode der kleinsten Quadrate. Diese versucht die quadratischen Abstände vom Funktionsgraphen $\hat{f}$ zu den Daten zu minimieren ($\min RSS = \sum\forall i (y_i - \hat{y}(x_i))^2$. Im einfachsten Falle liegt eine \textit{lineare} Regressionsfunktion der Form $\hat{f}(x) = \hat{\alpha} \cdot x + \hat{\beta}$ vor, wobei die Parameter $\hat{\alpha}$ und $\hat{\beta}$ wie folgt bestimmt werden können:\seFootcite{Vgl.}{S. 36 ff}{Studenmund.2014}\begin{equation}
	\hat{\alpha} = \frac{\sum\forall i x_i y_i - m \overline{x} \overline{y}}{\sum\forall i x^2_i - m \overline{x}^2}
\end{equation} \begin{equation}
	\hat{\beta} = \overline{y} - \hat{\alpha} \overline{x}^2
\end{equation}   }

\seNewGlossaryEntry{bigdata}{Big Data}{}{Hierbei handelt sich um Datenmenge, welche die Leistungsfähigkeit herkömmlicher Analysewerkzeugen übersteigt. Die Definition des Begriffes erfolgt häufig anhand der \glqq vier V\grqq.\seFootcite{Vgl.}{S. 10-12}{Freiknecht.2014} \begin{itemize}\item \textbf{Volume:} Es wird gerechnet, dass der weltweit verfügbare Datenbestand sich alle zwei Jahre verdoppelt.
\item \textbf{Velocity:} Eine große Datenmenge entsteht innerhalb kurzer Zeit und muss demzufolge schnell verarbeitet und analysiert werden.
\item \textbf{Variety:} Daten sind sich ähnlich, besitzen jedoch keine einheitliche Struktur und müssen aufgrund geringer Qualität aufbereitet werden.
\item \textbf{Veracity:} In der heutigen Informationsgesellschaft muss die Vertrauenswürdigkeit der Daten immer wieder überprüft werden.\end{itemize}}

\seNewGlossaryEntry{clustering}{Clustering}{}{Das Clustering ist eines der Grundbausteine des Data Mining, welche das Ziel der Aufteilung der Daten in verschieden Gruppen verfolgt, wobei diese Gruppen, auch \textit{Cluster} genannt, zuvor nicht bekannt sind. Die Elemente eines solchen Clusters sollten nach innen möglichst homogen sein, sich jedoch nach außen von anderen Clustern deutlich unterscheiden.\seFootcite{Vgl.}{S. 3}{Anderberg.2014}}

\seNewGlossaryEntry{regression}{Regression}{}{
Um Daten durch eine Funktion zu approximieren, wird die Regression verwendet. Im Allgemeinen beinhaltet diese die Analyse einer abhängigen Variablen von einer oder mehreren unabhängigen Variablen, die durch eine Regressionsgleichung ausgedrückt werden. Die quadratischen Abstände zur Funktion werden dabei durch die \textit{Methode der kleinsten Quadrate} minimiert, sodass eine möglichst genaue Anpassung der Funktion an die Daten resultiert. In der Praxis haben sich eine Vielzahl verschiedener Regressionsmodelle etabliert, die je nach Anwendungsfall ihre Verwendung finden: \textit{lineare} Regression, \textit{nichtlineare} Regression, \textit{multiple} Regression, sowie die \textit{nichtparametrische} Regression.\seFootcite{Vgl.}{S. 68-94}{Gunther.2014}}

\seNewGlossaryEntry{outlier}{Outlier Detection}{}{Definition folgt}

\seNewAcronymEntry{ki}{KI}{Künstliche Intelligenz}{Künstlichen Intelligenz}



% 2012-03-24
% \"Uber den optionalen Parameter in eckigen Klammern wird die Pluralform f\"ur das erste 
% Auftreten der Abk\"urzung definiert

\seNewAcronymEntry[URLs]{url}{URL}{Uniform Resource Locator}%
{Uniform Resource Locators}


% Definition von Symbolen
%
% 1. Parameter: Schluessel (key) des Symbols
% 2. Parameter: Symbol
% 3. Parameter: Text, der die Sortierreihenfolge festlegt (optional; falls nicht definiert, wird der Wert des zweiten 
%                        Parameters verwendet)
% 4. Parameter: Beschreibung des Symbols
%

\seNewSymbolEntry{ND}{ND}{a}{Nutzungsdauer einer Maschine}

\seNewSymbolEntry{pi}{$\pi$}{b}{Die Kreiszahl}




% Definition von Glossareintraegen
%
% 1. Parameter: Schluessel (key) des Glossareintrags
% 2. Parameter: Begriff, der im Glossar definiert wird
% 3. Parameter: Pluralform des Begriffes (optional; falls nicht definiert, wird der Wert des zweiten Parameters verwendet)
%                        Achtung: Pluralform gilt nur fuer das erste Auftreten des Begriffes im Text
% 4. Parameter: Beschreibung des Glossareintrags
%
%
%

\seNewGlossaryEntry{glos:AD}{Active Directory}{Active Directories}
{Active Directory ist in einem Windows 2000/Windows
Server 2003-Netzwerk der Verzeichnisdienst, der die zentrale
Organisation und Verwaltung aller Netzwerkressourcen erlaubt. Es
erm\"oglicht den Benutzern \"uber eine einzige zentrale Anmeldung den
Zugriff auf alle Ressourcen und den Administratoren die zentral
organisierte Verwaltung, transparent von der Netzwerktopologie und
den eingesetzten Netzwerkprotokollen. Das daf\"ur ben\"otigte
Betriebssystem ist entweder Windows 2000 Server oder
Windows Server 2003, welches auf dem zentralen
Dom\"anencontroller installiert wird. Dieser h\"alt alle Daten des
Active Directory vor, wie z.\,B. Benutzernamen und
Kennw\"orter.\protect\footnote{Bedauerlicherweise wei{\ss} der Autor dieses Dokumentes nicht mehr, woher diese Information stammt -- das 
geht in einer richtigen wissenschaftlichen Arbeit nat\"urlich \"uberhaupt nicht!!!}}
%\protect\seFootcite{Vgl.}{S. 200}{Dud09}}


\seNewGlossaryEntry{glos:bs}{Betriebssystem}{Betriebssysteme}{Die Begriffsdefinition sollten Sie eigentlich kennen!}



% Definition von Glossareintraegen, die gleichzeitig im Abk�rzungsverzeichnis auftreten
%
% 1. Parameter: Schluessel (key) des Glossareintrags
% 2. Parameter: Abk\"urzung
% 3. Parameter: Vollform
% 4. Parameter: Vollform im Plural (optional; falls nicht definiert, wird der Wert des dritten Parameters verwendet)
% 5. Parameter: Beschreibung des Glossareintrags

\seNewAcronymGlossaryEntry{glos:ma}{MA}{Mobile Applikation}{Mobile Applikationen}
{Eine Applikation, die auf einem mobilen Endger\"at ausgef\"uhrt wird.}

% 2012-03-24
% \"Uber den optionalen Parameter in eckigen Klammern wird die Pluralform f\"ur die Abk\"urzung definiert

\seNewAcronymGlossaryEntry[TAen]{glos:ta}{TA}{Transaktion}%
{Transaktionen}%
{Was eine Transaktion ist, sollten Sie ebenfalls bereits wissen!}





% Alternative Definition von Abk\"urzungen; diese sollten nicht verwendet werden!!!
%
%\newacronym{dhbw}{DHBW}{Duale Hochschule Baden-W\"urttemberg}
%\newacronym{usb}{USB}{Universal Serial Bus}


% Alternative Definition von Symbolen
%
% Achtung: ohne sort wird nach Name sortiert
%\newglossaryentry{pi}{
%name=$\pi$,
%description={Die Kreiszahl},
%type=symbolslist,
%sort=b
%}
%
%\newglossaryentry{ND}{
%name=$\mbox{\textsl{ND}}$,
%description={Nutzungsdauer einer Maschine},
%type=symbolslist,%
%sort=a
%}



% Alternative Definition von Glossareintr\"agen
%
%\newglossaryentry{glos:AD}{
%first=Active Directory\textsuperscript{GL},
%name=Active Directory,
%description={Active Directory ist in einem Windows 2000/Windows
%Server 2003-Netzwerk der Verzeichnisdienst, der die zentrale
%Organisation und Verwaltung aller Netzwerkressourcen erlaubt. Es
%erm\"oglicht den Benutzern \"uber eine einzige zentrale Anmeldung den
%Zugriff auf alle Ressourcen und den Administratoren die zentral
%organisierte Verwaltung, transparent von der Netzwerktopologie und
%den eingesetzten Netzwerkprotokollen. Das daf\"ur ben\"otigte
%Betriebssystem ist entweder Windows 2000 Server oder
%Windows Server 2003, welches auf dem zentralen
%Dom\"anencontroller installiert wird. Dieser h\"alt alle Daten des
%Active Directory vor, wie z.\,B. Benutzernamen und
%Kennw\"orter.\protect\seFootcite{Vgl.}{S. 200}{Dud09}}
%}














