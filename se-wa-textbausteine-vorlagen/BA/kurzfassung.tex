% Erzeugung der Kurzfassung; Verfasser, Firma und Thema werden automatisch \"ubernommen
%
% Der optionale Parameter kann verwendet werden, um f\"ur das Thema der Arbeit eine 
% andere Formatierung vorzunehmen; das sollte in der Regel nicht erforderlich sein;
% ausserdem besteht die Gefahr inkonsistenter Titel auf dem Titelblatt und in der 
% Kurzfassung
%
\seKurzfassung{} % dieses Kommando sollte standardm\"assig verwendet werden

%\seKurzfassung[\LaTeX-Vorlage zur Anfertigung \seThemaWaArbeit{} (Version \version{})]

Die heutige technisierte Massenerhebung an Daten im Fußballsport bietet neueste Möglichkeiten der Datenanalyse, wobei die Leistungsfähigkeit herkömmlicher Analysewerkzeuge bei Weitem überstiegen wird. Das Ziel dieser Bachelorarbeit ist es, eines der momentan angesagtesten Statistikmodelle, das \textit{Expected-Goals-Modell}, durch eine Funktion zur Berechnung der Wahrscheinlichkeit eines Torerfolges im Fußball aus jeder möglichen Position des Spielfeldes zu modellieren. Durch die Extraktion dieses neuen und fundierten Wissens können Trainer, Spielanalysten und Scouts profitieren.

Im Rahmen des systematisch angewendeten \textit{Knowledge Discovery in Data} Prozesses werden dem Leser die Grundlagen wie auch die Methoden des Data Minings näher gebracht sowie die einzelnen Prozessschritte und deren beinhaltenden Methoden zur Datenaufbereitung vorgestellt. Anhand dieser Methodik werden die zugrundeliegenden Daten auf Basis der aufgestellten Anforderungen selektiert, vorverarbeitet und transformiert, um diese durch die Verwendung unterschiedlicher Regressionsmodelle zu einer Funktion zu modellieren. Diese Resultate werden anschließend sowohl interpretiert als auch evaluiert, wobei sich die \textit{nichtparametrische} Regression unter der Betrachtung der Koordinaten des Schusses als am geeignetsten herausstellt hat. Abschließend wird ein Ausblick über mögliche Anwendungsfälle des mathematischen Modelles im Bereich des Datenscoutings im Fußballsport gegeben.