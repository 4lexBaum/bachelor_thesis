\section{Interpretation und Evaluation der Modelle}
\label{iue}
Die Interpretation sowie die Evaluation der Resultate des Data Minings stehen am Ende jedes KDD-Prozesses. Um eine fundierte Entscheidung für ein Modell treffen zu können, werden im Folgenden die vier vorgestellten Modellvarianten interpretiert und miteinander verglichen (siehe \vref{inter}), sowie die ausgewählte Funktion anschließend in \vref{eva} evaluiert.

\subsection{Interpretation}
\label{inter}
Anhand der in \vref{int} vorgestellten Kriterien der \textit{Validität}, der \textit{Neuartigkeit}, der \textit{Nützlichkeit}, sowie der \textit{Verständlichkeit} des Modelles, werden die vier vorgestellten und realisierten Regressionsmodelle aus \vref{umsetzung} miteinander verglichen. Für eine vereinfachte Darstellung der Erfüllung der Anforderungen werden den Modellen folgende Abkürzungen zu geordnet:

\begin{itemize}
\item Multiple Regression der Winkel- und Distanzbetrachtung $\rightarrow$ \textbf{A}
\item Nichtparam. Regression der Winkel- und Distanzbetrachtung $\rightarrow$ \textbf{B}
\item Multiple Regression der Koordinatenbetrachtung $\rightarrow$ \textbf{C}
\item Nichtparam. Regression der Koordinatenbetrachtung $\rightarrow$ \textbf{D}
\end{itemize}

Die \vref{tab:verg} bietet in Form einer Matrix einen übersichtlichen Vergleich, inwieweit jedes Modell den obengenannten Kriterien genügt. Die \textit{Validität} setzt sich dabei einerseits aus der Bewertung der objektiven Bestimmtheitsmaßen der einzelnen Modellierungen zusammen, andererseits auch aus subjektiven Wahrnehmungen, wie die vorgestellte Problematik der Wahrscheinlichkeitsbestimmung von Schussversuchen aus spitzem Winkel (vgl. \vref{heatmap}). Dabei überzeugt einzig allein das Modell \textbf{D} mit Bestimmtheitsmaßen um die \textsf{80}\% und einer korrekten Modellierung der Schussversuche aus spitzem Winkel. Die anderen drei Modelle haben mit ca. \textsf{50}\% eine zu geringe Anpassung der Funktion an die Daten und werden demzufolge auch für zukünftige Daten nicht valide sein. Das Kriterium der \textit{Neuartigkeit} erfüllen alle Modelle, da bislang keine Modellierungen von Funktionen zur Berechnung der Wahrscheinlichkeit eines Torerfolges im Fußball existieren. Die \textit{Nützlichkeit} ergibt sich größtenteils aus der \textit{Validität}, da nur ein Modell für einen Anwender nützlich sein kann, wenn valide Werte zurückgegeben werden. Modell \textbf{B} und \textbf{D} können diese Anforderung teilweise erfüllen und ermöglichen trotz abstrakter Modellierung erste Ansätze zur Mustererkennung. Letztlich beschreibt das Kriterium der \textit{Verständlichkeit} inwiefern die Ergebnisse des Modells von einem anderen Anwender nachvollzogen werden können. Die Visualisierung als Werkzeug spielt dabei eine entscheidende Rolle, um die Resultate in einem anschaulichen und nachvollziehbaren Format für die Entscheidungsträger, die oftmals selbst nicht am Prozess beteiligt waren, zu präsentieren. Der Vorteil bei der Koordinatenbetrachtung liegt einerseits in der simplen Form der Übergabe der Parameter in Form der Koordinatenpunkte des Schusses, ohne vorherige Winkel- oder Distanzberechnung, andererseits auch in der Verwendung des gleichen Koordinatensystems (vgl. \vref{rowK}), wodurch sich die positionsbedingten Wahrscheinlichkeiten direkt erkennen lassen. Im Gegensatz dazu lässt sich nur schwer auf den ersten Blick das Modell der Winkel- und Distanzbetrachtung zurück auf die Position des Schusses übertragen und fordert somit ein höheres Maß an Verständnis.

%%%%%%%%%%%%%%%%%% Vergleich %%%%%%%%%%%%%%%%%%
\tablefirsthead{\hline\multicolumn{5}{|c|}{\textbf{Vergleich der Modelle}}\\\hline\hline \textbf{Kriterium} & \textbf{A} & \textbf{B} & \textbf{C} & \textbf{D}\\\hline}
\tablehead{}
\tabletail{}
\tablelasttail { \multicolumn {5}{| c |}
{\textsl { \textcolor{green}{\ding{52}} erfüllt~~~~~~~\textcolor{orange}{\ding{108}} teilweise erfüllt~~~~~~~ \textcolor{red}{\ding{54}} nicht erfüllt  }}\\\hline }
\bottomcaption{Vergleich der Modelle\label{tab:verg}}
\begin{center}%
\begin{supertabular}{ | P{4cm} | P{2cm}  | P{2cm} | P{2cm}  |  P{2cm}  |}
&&&&\\
\textit{Validität}	& \textcolor{red}{\ding{54}}	 & \textcolor{orange}{\ding{108}} & \textcolor{orange}{\ding{108}}	&\textcolor{green}{\ding{52}}	\\
&&&&\\
\hline
&&&&\\
\textit{Neuartigkeit}	& \textcolor{green}{\ding{52}}	 & \textcolor{green}{\ding{52}}	& \textcolor{green}{\ding{52}}	& \textcolor{green}{\ding{52}}	\\
&&&&\\
\hline
&&&&\\
\textit{Nützlichkeit}	&\textcolor{red}{\ding{54}}	 & \textcolor{orange}{\ding{108}} & \textcolor{orange}{\ding{108}}	& \textcolor{green}{\ding{52}}	\\
&&&&\\
\hline
&&&&\\
\textit{Verständlichkeit}	&\textcolor{red}{\ding{54}}	 & \textcolor{red}{\ding{54}}	& \textcolor{orange}{\ding{108}}	& \textcolor{green}{\ding{52}}	\\
&&&&\\
\hline
\end{supertabular}
\end{center}

Aus \vref{tab:verg} lässt sich entnehmen, dass einzig Modell \textbf{D} alle Kriterien erfüllt, um neues Wissen zu repräsentieren. Die Validität der Funktion wird im Folgenden anhand einiger Testverfahren evaluiert.

\subsection{Evaluation}
\label{eva}
Um die Ergebnisse der Modellierung evaluieren zu können, wird die Validität der Funktion nachfolgend geprüft. Ein aussagekräftiger Indikator ist dabei der Vergleich der erwarteten Tore einer Mannschaft gegenüber den tatsächlich erzielten Tore. Dazu werden zunächst alle Schussversuche -- die den Anforderungen entsprechen (vgl. \vref{datar}) -- einer Mannschaft aus einer Bundesliga-Saison selektiert, anschließend für jeden einzelnen Schussversuch die Koordinaten der Funktion übergeben und letztlich die Wahrscheinlichkeiten für einen Torerfolg aufsummiert, woraus sich ein Erwartungswert ergibt, die sogenannten \textit{Expected Goals}. In \vref{bspeva} ist dazu exemplarisch der Vergleich der Tore eines Teams gegenüber den erwartenden Tore aus der Bundesliga-Saison 2014/15 dargestellt.

\begin{table}[H]
\centering
\caption{Beispiel der Evaluation}
\label{bspeva}
%\rowcolors[\hline]{1}{lightgray}{white}
\begin{tabular}{| P{2.5cm}| P{3.5cm}| P{3.5cm}| P{2.5cm}|}
\hline
\textbf{Team} & \textbf{Tore} & \textbf{Erwartete Tore} & \textbf{Differenz}        \\ \hline
Bayern        & 80            & 66                      & 14 \\ \hline
BVB           & 47            & 54                      & 7                         \\ \hline
Wolfsburg     & 72            & 65                      & 7                         \\ \hline
Bayer04       & 62            & 51                      & 11 \\ \hline
Mainz         & 45            & 49                      & 4                         \\ \hline
S04           & 42            & 43                      & 1  \\ \hline
H96           & 40            & 46                      & 6                         \\ \hline
Frankfurt     & 56            & 49                      & 7                         \\ \hline
Augsburg      & 43            & 48                      & 5                         \\ \hline
Hoffenheim    & 49            & 48                      & 1  \\ \hline
Gladbach      & 53            & 44                      & 9                         \\ \hline
Hertha        & 36            & 38                      & 2  \\ \hline
Stuttgart     & 42            & 43                      & 1  \\ \hline
Bremen        & 50            & 52                      & 1  \\ \hline
Freiburg      & 36            & 37                      & 1  \\ \hline
Paderborn     & 31            & 38                      & 7                         \\ \hline
Koeln         & 34            & 36                      & 2  \\ \hline
HSV           & 25            & 39                      & 14 \\ \hline
\end{tabular}
\end{table}

Werden diese Werte mit Hilfe der Korrelation verglichen, ergibt sich ein Korrelationskoeffizient von \textsf{0.9}, welcher einen starken Zusammenhang zwischen tatsächlich erzielten Tore und den erwarteten Tore aufzeigt. \vref{g1415} im Anhang stellt diesen Zusammenhang noch einmal graphisch dar, wobei die blaue Linie die \glqq Ideallinie\grqq~repräsentiert, auf der die erwartete Anzahl der Tore den real erzielten Toren entspricht. Dieser Zusammenhang kann gleichermaßen für die Anzahl der erwarteten Gegentore gegenüber den tatsächlich erhaltenen Gegentore aufgestellt werden, wie in \vref{cg1415} abgebildet. Werden innerhalb einer Spielpaarung die erwarteten Tore für beide Mannschaften berechnet, so kann ein ganzes Spielergebnis (Sieg, Niederlage oder Unentschieden) daraus bestimmt werden. Wenn dieser Vorgang für alle Spielpaarungen einer ganzen Bundesliga-Saison durchlaufen wird, ergibt sich ein Vergleich zwischen den erwarteten Punkten einer Mannschaft gegenüber den real erzielten Punkte, welcher in \vref{p1415} dargestellt ist. In \vref{anheva} befinden sich diese Zusammenhänge ebenfalls für die Bundesliga-Saisons 2015/16 und 2016/17.

Aus den obengenannten Vergleichen ergeben sich weitere Indikatoren, die die Validität der ausgewählten Funktion fundieren:

\begin{itemize}
\item Für \textsf{15\%} der Spiele kann das Spielergebnis exakt bestimmt werden (genauer Spielstand – bsp.  \textsf{0:0}).
\item Für  \textsf{50\%} der Spiele kann das Spielergebnis bestimmt werden (unabhängig von der Anzahl der Tore, das bedeutet ein Ergebnis von \textsf{1:0 $\hat{=}$ 2:1 $\hat{=}$ 3:1} usw.) .
\item Für \textsf{75\%} der Spiele weicht die erwartete Anzahl gegenüber der tatsächlichen Anzahl der geschossenen Tore um maximal ein Tor ab.
\item Die Abweichung der erwarteten Tore aller verfügbaren Spiele gegenüber den tatsächlich erzielten Tore liegt lediglich zwischen \textsf{1} bis \textsf{2\%}.
\end{itemize}

Diese Indikatoren, sowie die sehr guten Korrelationswerte aller evaluierten Bundesliga-Saisons, belegen eine genaue Anpassung der Funktion an die Daten, wodurch ein fundiertes Modell für die Berechnung der Wahrscheinlichkeit eines Torerfolges im Fußball resultiert.