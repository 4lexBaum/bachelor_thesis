\section{Ziel}

\begin{quote} 
\glqq Expected Goals: Das angesagteste Statistikmodell im Profifußball.\grqq\seFootcite{}{}{NilsNordmann.2016}
\end{quote}

Im heutigen kommerziellen Fußballsport verwenden Experten immer wieder die Worte \glqq Geld schießt Tore\grqq, jedoch können das Daten auch. Modewörter aus der Informatik wie \gls{dm}, \gls{bigdata} oder \gls{ml} haben sich inzwischen auch im Bereich des Fußballs etabliert. Durch die Massenerhebung von Spieldaten, wie die Erfassung der Positionsdaten aller Spieler und des Balles oder die Aufzeichnung aller Aktionen eines Spiels wie Pässe, Schüsse oder Fouls, entsteht eine rohe Datenflut an Informationen, die die Leistungsfähigkeit herkömmlicher Analysewerkzeuge bei Weitem übersteigt. Der Mensch ist in seiner Wahrnehmung limitiert, ein Computer jedoch nicht. Mit den richtigen Methoden und Werkzeugen lässt sich Wissen aus dieser überdimensionalen Datenmenge extrahieren, wodurch die Analyse im Fußball revolutioniert wird. Spitzenclubs wie Manchester City oder der FC Bayern München leisten sich ganze Teams von Datenspezialisten, welche die Daten interpretieren und auf
das Spielgeschehen übertragen. Die Datenanalyse kann und soll dabei nicht das fußballerische Expertenwissen der Trainer oder Scouts ersetzen, sondern viel mehr die subjektiven Wahrnehmungen validieren. Konkret behandelt diese Arbeit eines der momentan angesagtesten Stastikmodelle im Fußball, die \textit{Expected Goals}, ein Modell das die Wahrscheinlichkeit eines Schusses für einen Torerfolg aus jeder möglichen Position ermittelt. Dazu existieren bereits zahlreiche unterschiedliche Ansätze, jedoch wurde bislang keine Funktion zur Berechnung der Wahrscheinlichkeit modelliert. An dieser Stelle greift die Intention dieser Arbeit, die das Ziel verfolgt mit der Modellierung einer Funktion zur Berechnung der Wahrscheinlichkeit eines Torerfolges im Fußball, neues und fundiertes Wissen zu schaffen, von dem Trainer, Spielanalysten und Scouts profitieren. Daraus ergeben sich wissenschaftliche Teilfragen, von der Selektion der relevanten Daten bis hin zur 
	