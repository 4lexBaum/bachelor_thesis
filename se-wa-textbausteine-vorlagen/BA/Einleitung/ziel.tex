\section{Ziel}
Hintergrund:
-	Begriff Expected Goals wird als einer der neuen Schlüsselindikatoren im Fußball angesehen
-	Frage nach der Wahrscheinlichkeit von Punkt X,Y einen Torerfolg zu erzielen
-	Zugrunde liegen die Spieldaten der Bundesligasaisons 2014/15, 2015/16, sowie die aktuellen Spiele der Saison 2016/16
-	Expected Goals gibt es in zahlreichen Varianten, doch wurde noch keine Funktion dafür modelliert (Ziel der Arbeit = neues Wissen schaffen)
-	Trainer, Spielanalysten und Scouts würden von einem fundierten und wissenschaftlich begründeten KPI profitieren

- Beantwortung der Fragen:
1.	Welche Daten liegen vor?
2.	Wie sollen die für die Funktion relevanten Daten selektiert werden?
3.	Müssen Daten bereinigt bzw. aufbereitet werden?
4.	Wie kann eine Funktion aus Daten modelliert werden?
5.	Welche Arten der Regressionsanalyse gibt es?
6.	Welche Tools/welche Software kann für die Berechnung genutzt werden?
7.	Welche Annahmen werden für das Modell getroffen und warum?
8.	Wie kann der Erfolg der resultierenden Funktion gemessen werden?
