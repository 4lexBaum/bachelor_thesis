\section{Umgebung}
\paragraph{Unternehmen}
Die SAP\footnote{\gls{sap}} wurde 1972 von fünf ehemaligen IBM Mitarbeitern gegründet und ist seit mehr als 40 Jahren, hinsichtlich des Marktanteils mit über 282.000 Kunden, das weltweit führende Unternehmen für Anwendung- und Analysesoftware. Der im baden-württembergischen Walldorf gegründete Aktienkonzern bietet mit dem bis heute bekanntesten Produkt \textit{SAP ERP} eine Softwarelösung zur Abbildung aller Geschäfts- und Produktionsprozesse in einem Unternehmen von Personal- und Rechnungswesen bis hin zur Logistik. Mit dem heutigen Stand der Entwicklung setzt die SAP ihren Fokus verstärkt auf die Bereiche Cloud, Mobile und Internet of Things, um mit den anderen Unternehmen konkurrieren zu können und den Anschluss an den Trend der Zeit nicht zu verlieren. Die SAP beschäftigt in über 180 Ländern mehr als 77.00 Mitarbeiter und erzielte im Jahr 2015 einen Umsatz von 20,8 Mrd Milliarden Euro, sowie ein Betriebsergebnis von 6,3 Milliarden Euro.\footnote{Zahlen vor Abzug der Steuern\\ Weitere Information zum Geschäftsbericht der SAP SE aus dem Jahr 2015 unter: \\ http://www.sap.com/docs/download/investors/2015/sap-2015-geschaeftsbericht.pdf [10.01.2017]}

\paragraph{Abteilung}
Die Praxisphase erfolgte in der Abteilung \textit{Sports \& Enternainment}, die sich von den klassischen SAP Geschäftsbereichen isoliert hat und alles rund um den Sport betreut. Im Bereich des Fußballs liegt der Fokus einerseits auf der Organisation des gesamten Vereins inklusive Umfeld, sprich Management, Marketing, Mannschaft, Jugend oder auch Fans, andererseits auch auf der Spielanalyse mit Hilfe von erhobenen Daten. Dazu steht die Abteilung in regelmäßigen Kontakt mit dem Bundesligaverein der TSG 1809 Hoffenheim sowie der deutschen Nationalmannschaft, um ständig neue Anwendungsfälle zu gewinnen. Alle Funktionalitäten sollen in einem Produkt, dem sogenannten \textit{Sports One} vereint werden, welches aus verschiedenen Rollen, wie Spieler, Trainer oder auch Mannschaftsarzt verwendet werden kann. Im Bereich der Spielanalyse und der Leistungsdiagnostik werden Unmengen an Daten gesammelt, die es für den späteren Anwender zu visualisieren gilt. Hier findet sich der in dieser Arbeit beschriebene Anwendungsfall wider, mit dessen unterstützender Funktion eine Funktion für die Berechnung der Wahrscheinlichkeit eines Torerfolges modelliert werden soll.