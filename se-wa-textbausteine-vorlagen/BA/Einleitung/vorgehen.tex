\section{Vorgehen}

Neben dem allgemeinen Ziel der Modellierung der Funktion zur Berechnung der Wahrscheinlichkeit eines Torerfolges im Fußball, ergeben sich weitere wissenschaftliche Teilfragen, die es innerhalb dieser Arbeit zu beantworten gilt. Zunächst muss festgestellt werden, welche Datenmenge zugrunde liegt, welche Daten daraus selektiert werden sollen und ob diese gegebenenfalls aufbereitet oder bereinigt werden müssen. Anschließend muss ein Analysewerkzeug gewählt werden, welches durch die Unterstützung eines Softwaretools eine Funktion aus den Daten modelliert. Letztlich gilt es den Erfolg der resultierenden Modellierung zu messen, um die Validität sicherzustellen. Als grundlegende Methodik wird dazu der allgemeingültige Knowledge Discovery in Data Prozess für Data-Mining-Projekte verwendet, welcher die genannten wissenschaftlichen Teilfragen dieser Arbeit systematisch beantwortet. Zunächst werden dazu die theoretischen Bestandteile des Data Minings studiert (siehe \vref{dm}), die einzelnen Phasen und Methoden des Knowledge Discovery in Data Prozesses eruiert (siehe \vref{kdd}), sowie die mathematischen Grundlagen der Funktionsmodellierung mittels der Regressionsanalyse und des Softwaretools MATLAB erläutert (siehe \vref{fm}). Des Weiteren wird der heutige Stand und die Bedeutung der \textit{Expected Goals} näher beleuchtet (siehe \vref{goals}) und die vorhandenen Spieldaten analysiert, um daraus die Schussversuche selektieren zu können (siehe \vref{opta}). Eine wirtschaftliche Betrachtung des Modelles aus Sicht der SAP als auch aus Sicht einen Vereines runden die Analysephase ab (siehe \vref{wa}). Die im theoretischen Teil vorgestellten Methoden der Prozessphasen werden innerhalb der Umsetzung angewendet (siehe \vref{umsetzung}), um eine fundierte Modellierung der Funktion zu gewährleisten, die abschließend anhand einiger Verfahren evaluiert wird (siehe \vref{iue}). So kann der Leser die Arbeit systematisch nachvollziehen und sich entlang des Prozesses hangeln.

Die Funktion wird dabei einerseits unter der Betrachtung der Koordinaten des Schussversuches, als auch in Bezug der Distanz und des Winkels des Schussversuches zum Tor modelliert. Dabei wurden von den internen Fachexperten der Abteilung folgende Anforderung für die Funktion (siehe \vref{tab:anf}) aufgestellt, die es für die Modellierung zu berücksichtigen gilt. Anforderung \textsf{1} und \textsf{2} beschreiben die In- und Output-Variablen der Funktion. Die Parameter, die die Wahrscheinlichkeit zwischen 0 und 1 bestimmen, sind lediglich die Koordinaten des Schussversuches in Form eines $x$- und $y$-Wertes. Wichtig ist, dass lediglich Schussversuche berücksichtigt werden dürfen, die während des \glqq laufenden\grqq~Spiels abgegeben wurden, da die Wahrscheinlichkeit für einen Torerfolg aus einer Standardsituation, wie die eines Elfmeter oder direkten Freistoßes, separat betrachtet werden muss (Anforderung \textsf{3}).\footnote{Bei einem Elfmeter gibt es beispielsweise keine Einwirkung von gegnerischen Spielern.} Die Anforderung \textsf{4} schließt Eigentore von der Modellierung aus, da diese die Wahrscheinlichkeit eines Schussversuches nicht beeinflussen und damit die Modellierung verzerren würden. Diese Anforderung war jedoch nicht von vornherein vorgegeben und konnte erst innerhalb des iterativen Prozesses als Problematik erkannt werden. Anlässlich der Tatsache, dass keine konkrete Aussage getroffen werden kann, ob ein geblockter Schuss sonst in einem Torerfolg resultiert hätte, bleiben solche Schussversuche ebenfalls unberücksichtigt (Anforderung \textsf{5}). Durch die Anforderung \textsf{6} und \textsf{7} wird die Form der Funktion bestimmt, um eine eventuelle zufällige Verteilung der Schussversuche auszugleichen und die Berechnung des Winkels und der Distanz des Schusses zum Tor zu vereinfachen.

%%%%%%%%%%%%%%%%%% Anforderungen %%%%%%%%%%%%%%%%%%
\tablefirsthead{\hline\multicolumn{3}{|c|}{\textbf{Anforderungen}}\\\hline\hline \textbf{Nr.} & \textbf{Beschreibung} & \textbf{vorgegeben}\\\hline}
\tablelasttail{}
\bottomcaption{Auflistung der Anforderungen\label{tab:anf}}
\begin{center}%
\begin{supertabular}{ | P{1.5cm} | P{8cm} | P{2.5cm} |}
\vspace*{1mm}\textbf{1.} 	& \vspace*{1mm}Input-Variablen der Funktion sind die Koordinaten eines Schusses& \vspace*{1mm}	\textit{ja}\\
\hline
\vspace*{1mm}\textbf{2.}	& \vspace*{1mm}Output der Funktion ist die Wahrscheinlichkeit zwischen 0 und 1	& \vspace*{1mm}\textit{ja} 	\\
\hline
\vspace*{1mm}\textbf{3.}	& \vspace*{1mm}Es dürfen nur Schüsse berücksichtigt werden, die während des \glqq laufenden\grqq~Spiels abgegeben wurden  	& \vspace*{1mm}\textit{ja}  	\\
\hline
\vspace*{1mm}\textbf{4.}	& \vspace*{1mm}Eigentore müssen ausgeschlossen werden  	& \vspace*{1mm}\textit{nein}  	\\
\hline
\vspace*{1mm}\textbf{5.}	& \vspace*{1mm}Geblockte Schüsse müssen ausgeschlossen werden\	& \vspace*{1mm}\textit{ja}  	\\
\hline
\vspace*{1mm}\textbf{6.}	& \vspace*{1mm}Der Ursprung der Funktion liegt in der Mitte der gegnerischen Torlinie & \vspace*{1mm}\textit{ja}  	\\
\hline
\vspace*{1mm}\textbf{7.}	& \vspace*{1mm}Die Funktion muss symmetrisch zu beiden Spielhälften sein (Spiegelung an der gedachten Linie zwischen der generischen und der eigenen Tormitte) & \vspace*{1mm}\textit{ja}  	\\
\hline
\end{supertabular}
\end{center}
