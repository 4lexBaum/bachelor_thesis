\section{Vorgehen}
Methodik:
Als grundlegende Methodik wird der allgemeingültige Knowledge Discovery Process verwendet. 
Der Fokus liegt dabei vor allem im Schritt des Data Minings, in dem auch die Funktion letztendlich modelliert wird. Die vorherigen Schritte zeigen die Datenaufbereitung als auch die –transformation, um den ganzen Kontext besser verstehen zu können. In den einzelnen Schritten gibt es wiederum wissenschaftliche Methoden, die im theoretischen Teil kurz vorgestellt und in der Umsetzung dann angewendet werden. Beispielsweise findet sich unter dem Punkt Data Mining die mathematische Methode der Regressionsanalyse. So kann der Leser die Arbeit systematisch nachvollziehen und sich entlang des roten Pfadens hangeln. 

Erwartete Ergebnisse:
-	verschiedene Funktionen bei unterschiedlicher Betrachtung:
o	der Auswahl der Daten (Schüsse aus dem Spiel, Standards, …)
o	des Winkels zum Tor
o	der Distanz zum Tor
-	unterschiedliche Flächen der Funktion im dreidimensionalen Raum
o	Kegel
o	Teil eines Ellipsoids
