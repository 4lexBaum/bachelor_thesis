\section{Expected Goals}
\label{goals}
Dieser Abschnitt soll dem Leser den aktuellen Forschungsstand der \textit{Expected Goals} vermitteln, deren Bedeutsamkeit für den Fußballsport dabei explizit aufzeigen, sowie den Einfluss von Data-Mining-Methoden hinsichtlich der Wissensgewinnung darstellen.

\begin{quote}
\textit{\glqq Expected Goals - Das angesagteste Statistikmodell im Profifußball\grqq}
\end{quote}

So betitelt Nils Nordmann seinen Online-Artikel im Interview mit Dustin Böttger, Geschäftsführer von \gls{gsn}, einem der gefragtesten Datenanalysten aus Deutschland, der mit mehreren Bundesligavereinen in Kooperation steht.\seFootcite{}{}{NilsNordmann.2016} Statistische Analysen sind im Bereich des Fußballs keine Neuheit mehr, jedoch liegt der Ursprung der sportlichen Datenanalyse in einer anderen Sportart. Der amerikanische Historiker und Statistiker Bill James veröffentlichte 1977 erste Analysen zwischen geschlagenen und gefangenen Bällen im Baseball, um eine objektive Bewertung der Gesamtleistung eines Spielers aufstellen zu können. Schumaker, Solieman und Chen bezeichnen diese Entwicklung als eine Art \glqq Revolution\grqq-- einen Wandel von traditionellen Statistiken hin zum Wissensmanagement.\seFootcite{Vgl.}{S.36}{Schumaker.2010} Diese löste eine Welle der Erstellung neuer Maßzahlen aus, wovon einige im Jahr 2002 von der amerikanischen Baseball Profimannschaft \textit{Oakland A’s Billy Bean} als Grundlage zur Zusammenstellung eines neuen Teams dienten. Die \textit{Boston Red Sox} ließen sich von dieser Idee inspirieren und  gewannen anschließend sogar 2004 und 2007 die Meisterschaft.\seFootcite{Vgl.}{S.36}{Schumaker.2010} Auch aus anderen Sportarten gibt es vergleichbare Beispiele, wie die digitale Revolution im Basketball im Jahr 1980 durch den Statistiker Dean Oliver, der neue Messwerte zur Beurteilung von Spielern veröffentlichte.\footnote{Dean Oliver beriet 2005 die \textit{Seatlle Supersonics} und verhalf zur amerikanischen Meisterschaft.}

Waren im Fußball in der Vergangenheit noch rein quantitative \glspl{kpi} wie der Ballbesitz, die Passquote oder die Anzahl der Torschüsse von Bedeutung, wird das Spiel heutzutage bis in das kleinste Detail (z.B. die Anzahl der vertikal \glqq überspielten\grqq~Gegenspieler durch einen Pass) analysiert. Durch den Fortschritt der Videotechnik können alle Aktionen eines Spieles aufgezeichnet werden, wodurch sich neue stichhaltige Bewertungsmethoden herauskristallisiert haben. Sumpter, Anderson und weitere Fachexperten untersuchen mit Hilfe von Mathematik und Statistik das Spiel und stellen in ihren Ausführungen einige Thesen und Modelle auf.\seFootcite{Vgl.}{}{Sumpter.2016}\seFootcite{Vgl.}{}{Anderson.2014}\seFootcite{Vgl.}{}{Heuer.2010} Eines der momentan angesagten Modelle ist das der \glqq \textbf{Expected Goals}\grqq (\textit{dt. die zu erwartenden Tore}), welches die Qualität von Torschüssen vielseitig, objektiv und plausibel misst.\seFootcite{}{}{NilsNordmann.2016} Dazu wird jedem Schuss, unter der Berücksichtigung von Parametern (wie beispielsweise der Position oder des Körperteils mit dem geschossen wurde), eine bestimmte Erfolgswahrscheinlichkeit zugewiesen. Die Bestimmung der Wahrscheinlichkeit, die Auswahl der einbezogenen Schüsse wie auch Parameter und teilweise das gesamte Modell wird öffentlich von den Analytikern (meist aus Unternehmen der Sportanalyse/-beratung) nur kurz ausgeführt oder gar komplett geheimgehalten. Einblicke in ihre Modellierungen bieten unter anderem Opta Sports\seFootcite{Vgl.}{}{PhilippObloch.2015}, der TV-Sender Sky Sports,\seFootcite{Vgl.}{}{PhilippErtl.2016} oder Experten, wie Michael Caley, in ihren Internetpublikationen.\seFootcite{Vgl.}{}{MichaelCaley.2017} Ein \textit{Expected-Goals-Modell} offeriert eine statistisch belegte und damit objektive Bewertung von Schüssen und bildet einen neuen \gls{kpi} bezüglich der Qualität einer Torchance. Anhand dieser Grundlage ist es möglich, weitere Bewertungsmethoden für Spieler und Mannschaften zu ermitteln, die vor allem im Scouting-Bereich ihre Anwendung finden. Durch die qualitative Bewertung der Schüsse eines Stürmers mittels des Expected-Goals-Modells, kann eine objektive Aussage über dessen Erfolgsquote getroffen werden (beispielsweise ob diese über den erwarteten Toren liegt), welche dann zur Spielersuche herangezogen werden kann. Eine Gefahr in der Modellierung der Expected Goals stellt die \textit{Überparametrisierung} (vgl. \vref{bhm}) dar. Werden zu viele Parameter, z.B. welcher Spieler geschossen hat und ob mit seinem starken oder schwachen Fuß geschossen wurde, seine Tagesform, die Leistung des generischen Torhüters, usw. bei der Modellierung herangezogen, verliert das Modell durch zu viele Details seine Abstraktion und folglich seine allgemeine Aussagekraft (für alle Schüsse). Die Kunst liegt in der \vref{bhm} beschriebenen Balance von \textit{Underfitting} und \textit{Overfitting} des Modells.

Durch die Technisierung der Datenaufnahme im Fußball werden stetig mehr Daten während eines Spieles erhoben\footnote{Beispielsweise durch Videobildverarbeitung oder Sensordaten.}, woraus im Laufe einer Saison eine Datenmenge resultiert, die die Leistungsfähigkeit herkömmlicher Analysewerkzeuge übersteigt. Um wertvolle Informationen aus den umfangreichen Daten zu extrahieren, greifen auch Datenanalysten im Bereich des Fußball auf die Prozesse und Methoden des Data Minings zurück. Ausführliche Einblicke in die Komplexität der Datenanalyse im Sport stellen unter anderem Schumaker et al. in ihrer Ausarbeitung vor.\seFootcite{Vgl.}{}{Schumaker.2010} Data-Mining-Methoden, wie das \gls{clustering} zur Einteilung von Spielertypen, die Regressionanalyse zur Ermittlung von Erfolgsfaktoren einer Saison, Entscheidungsbäume zur Bestimmung des perfekten Ein- und Auswechslungszeitpunktes, als auch Neuronale Netze zur Prognose von Spielausgängen, werden hierbei zur Wissensgewinnung verwendet.\seFootcite{Vgl.}{}{GunjanKumar.2013} Darüber hinaus werden einige dieser Techniken zur komplexen Erkennung von Taktiken und Spielphilosophien eingesetzt, welche in der Ausführung von Rein konkretisiert werden.\seFootcite{Vgl.}{}{Rein.2016}



