\section{Fazit}
Im Rahmen der vorliegenden Arbeit konnte gezeigt werden, wie Methoden des Data Minings innerhalb eines strukturierten Prozesses im Bereich der Datenanalyse im Fußball eingesetzt werden können, um neues Wissen zu schaffen. Durch den systematischen Aufbau des Knowledge Discovery in Data Prozesses und den darin beinhaltenden Methoden zur Verbesserung der Datenqualität konnte aus der rohen Datenmenge, gemäß den Anforderungen, eine valide und fundierte Funktion modelliert werden, die die Wahrscheinlichkeit für einen Torerfolg eines Schusses aus jeder möglichen Position auf dem Spielfeld berechnet. 

Im Zuge der vorbereitenden Schritte der Datenselektion, -vorverarbeitung und -transformation wurde in der Umsetzung deutlich, wie essentiell diese Phasen für die Qualität der Resultate des Data Minings sind. So konnten die Eigentore, die zuvor kein Bestandteil der Anforderungen waren, nachträglich ausgeschlossen und Ausreißer in der Datenmenge identifiziert oder das Problem der weißen Linien aufgedeckt werden. Aus der vergleichenden Analyse der modellierten Funktionen resultiert, dass die Verwendung des \textit{nichtparametrischen} Regressionsmodells in Form sogenannter \textit{Splines} in Kombination mit der Betrachtung der Koordinaten des Schusses besonders geeignet ist und zu einer genauen Anpassung des Modells an die Daten führt. In Bezug auf die Betrachtung des Winkels und der Distanz des Schussversuches zum Tor erwies sich weder das \textit{multiple} noch das \textit{nichtparametrische} Regressionsmodell als valide. Die Ergebnisse der Evaluation beweisen, dass mit lediglich zwei Parametern, den Koordinaten des Schusses, eine gute Balance zwischen einem Under- bzw. Overfitting des Modells gefunden werden konnte. Das Software-Tool MATLAB bestätigte des Weiteren die weltweit erstklassige Reputation mit den bereitgestellten Bibliotheken zur Regressionsanalyse, sowie deren intuitiven und benutzerfreundlichen Bedienung. \enlargethispage{2\baselineskip} 

Die erstmalige Modellierung der \textit{Expected Goals} durch eine Funktion bietet die Basis für ein datenbasiertes und objektiv bewertetes Scouting, sowie weitere nützliche Anwendungsfälle, die im nachfolgenden Ausblick in \vref{ausblick} behandelt werden.

