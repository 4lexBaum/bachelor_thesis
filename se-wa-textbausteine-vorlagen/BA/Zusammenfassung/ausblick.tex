\section{Ausblick}
\label{ausblick}

Die vorliegende erarbeitete Funktion wurde anhand der Daten aus der deutschen Bundesliga modelliert und ist somit auch zunächst lediglich für diese Fußballliga gültig. Das Modell könnte dahingehend erweitert werden, dass für jede europäische Topliga ein eigenes Modell erstellt wird, um Muster zwischen diesen Ländern erkennen zu können. In Bezug auf einen Spieler ergeben sich weitere alternative Anwendungsmöglichkeiten. Werden beispielsweise alle Schüsse eines Spielers in die Funktion eingesetzt, ergibt sich daraus die erwartete Anzahl an Toren. Dieser Wert wird mit den tatsächlich erzielten Toren verglichen, um statistisch belegte Aussagen über die Erfolgsquote des Spielers treffen zu können, welche für oder gegen dessen Qualität sprechen. Des Weiteren bietet dieses Modell die Grundlage auch Defensivaktionen, wie zum Beispiel gewonnene Zweikämpfe, in einen neuen Maßstab zu setzen, da durch den Ballbesitzwechsel die gegnerische Mannschaft aus der vorherigen Position kein Tor mehr schießen kann. Beispielsweise konnte ein Abwehrspieler die siebzig-prozentige Wahrscheinlichkeit auf einen Torerfolg aus fünf Metern Distanz mit einem gewonnenen Zweikampf verhindern, wodurch seine Kennzahl steigt. Im Gegensatz dazu würde ein verlorener Zweikampf in dieser Situation seinen KPI negativ belasten.

In Bezug auf die vorgestellte Anwendung innerhalb des Scoutingbereiches im Fußball, ergibt sich die Frage, inwieweit ein solches Modell einen Scout ersetzen beziehungsweise dessen Aufgaben automatisiert ausführen kann. Diese Frage lässt sich teilweise mit ja beantworten. Möchte ein Verein einen neuen Stürmer verpflichten und verfügt dabei über ein limitiertes Budget, so kann mit Hilfe dieses Modelles eine automatisierte Vorselektion passender Spieler durchgeführt werden, wodurch die Anzahl der dafür benötigten Scouts reduziert werden kann. Eine subjektive Einschätzung des Spielers muss letztlich immer erfolgen, sodass eine vollständige Ersetzung der Funktion des Scouts auch in Zukunft ausgeschlossen werden kann. Zu viele nicht objektiv messbare Indikatoren, wie taktisches Spielverständnis oder Teamfähigkeit, spielen eine entscheidende Rolle bei der Verpflichtung eines neuen Spielers. Dieses Modell soll nicht das fußballerische Expertenwissen des Scouts ersetzen, sondern subjektive Wahrnehmungen, die jedoch durch solch eine Funktion objektiv messbar sind, validieren und alltäglich anstehende Aufgaben erleichtern. Zusätzlich können Algorithmen entworfen werden, die automatisiert auf dem Transfermarkt nach sogenannten \glqq Schnäppchen\grqq~oder verborgenen Talenten suchen, die für einen günstigen Preis dadurch verpflichtet werden können. 

Nicht zuletzt dient diese Modellierung auch als Erfahrungsgrundlage für weitere zukünftige Modelle, die aus der großen rohen Datenmenge, die durch neueste Technologien im Fußball erhoben werden, die Grundlage zur Ableitung neuen Wissens schaffen.