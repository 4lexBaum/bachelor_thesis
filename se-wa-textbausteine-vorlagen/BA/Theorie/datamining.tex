\subsection{Data-Mining-Methoden}
\label{dmmethoden}

Nachdem die Daten in geeigneter Form vorliegen, kommt das eigentliche Herzstück des \gls{kdd}-Prozesses -- das \textit{Data Mining} -- zum Tragen. In diesem Schritt wird zu nächst überprüft, welche grundlegende Data-Mining-Aufgabe es zu lösen gilt, um anschließend ein passendes Analyseverfahren zur Identifizierung von Mustern und Zusammenhängen auswählen zu können.\seFootcite{Vgl.}{S. 10}{Cleve.2014} Die interdisziplinäre Wissenschaft des Data Minings umfasst bewährte Techniken aus mehreren Forschungsgebieten, welche auf verschiedenste Problemfälle der Realität, wie Zeitreihenanalysen, Funktionsmodellierungen, Klassifikation uvm., angewendet werden können. Grundsätzlich basieren fast alle Analyseverfahren auf der Mathematik, insbesondere der Statistik.\seFootcite{Vgl.}{S. 12}{Cleve.2014} Im allgemeinen unterteilt man die Data-Mining-Methoden in zweit Kategorien: \textit{Prognose} und \textit{Beschreibung}. Hierzu gibt \vref{dmmethods} einen Überblick über die Einteilung der etablierten Methoden, welche im Folgenden kurz aufgeführt werden.

\paragraph{Prognose:} In dem Bereich der Prognose unterscheidet man zwischen zwei Gruppen: \textit{statistische Methoden} und \textit{symbolische Methoden}. Letztere versuchen das Wissen durch Symbolik und Verknüpfung, auf einer leichter interpretierbaren Ebene für Menschen, zu vermitteln. Im Gegensatz dazu, repräsentieren statistische Methoden das Wissen mit Hilfe der Erstellung von mathematischen Modellen.\seFootcite{Vgl.}{S. 3}{Garcia.2015} Die am häufigst angewendeten statistischen Methoden sind die:\seFootcite{Vgl.}{S. 3-5}{Garcia.2015}\seFootcite{Vgl.}{S. 23-24}{Han.2012}

\begin{itemize}
\item \textit{Regressionsanalyse}
\\ Die älteste \gls{dm}-Methode dient zur Funktionsmodellierung von einer abhängigen oder mehreren unabhängigen Variablen. Die Form der Funktion wird dabei durch das ausgewählte Verfahren, beispielsweise \textit{lineare oder quadratische Regression}, bestimmt und kann anhand bestimmter Parameter validiert werden, wie \glqq gut\grqq~diese zu den eingebrachten Daten passt.\seFootcite{Vgl.}{S. 3}{Garcia.2015}

\item \textit{(Künstliche)} \gls{nn}
\\ In diesem Teilbereich der künstlichen Intelligenz wird versucht, einen Wissensspeicher zu kreieren, der ähnlich unserem leistungsfähigen Gehirn funktioniert. Hierbei werden die biologischen Elemente und Vorgehensweise des Gehirns, in Form von \textit{Neuronen}, in die Welt des Computers übertragen. Durch gerichtete und gewichtete Verbindungen sind diese Neuronen untereinander verknüpft und bilden so ein gemeinsames Netz für die Informationsverarbeitung.\seFootcite{Vgl.}{S. 47}{Cleve.2014}

\item \gls{svm}
\\ Die auf \gls{ml} basierende Methode versucht Objekte zu klassifizieren. Dabei werden alle Objekte als Vektoren in einem Raum dargestellt und durch sogenannte \textit{Hyperebenen} (fungieren als Trennflächen) geteilt, um eine möglichst zuverlässige Zuordnung der Daten in vordefinierte Klassen zu erreichen.\seFootcite{Vgl.}{S. 313}{Aggarwal.2015}
\end{itemize}

Im Bereich der symbolischen Methoden hat sich die Technik des \textit{Entscheidungsbaumes} etabliert. Sie dient ebenfalls der Klassifizierung von Objekten, indem innerhalb jedes Iterationsschrittes das am \textit{besten} zu klassifizierende Attribut gefunden wird, um die Daten daran aufzusplitten. Durch dieses Verfahren entsteht ein Entscheidungsbaum, von dem Regeln, wie \textit{If-Else-Zweige}, abgeleitet werden können.\seFootcite{Vgl.}{S. 5}{Garcia.2015}

\paragraph{Beschreibung:}

\begin{itemize}
\item \textit{Clustering}
\\ Im Gegensatz zur Klassifizierung sind bei der Methode des Clustering zuvor keine Klassen bzw. Gruppen definiert. Dieses weitverbreitete Werkzeug im Bereich des Data Minings versucht Daten in sogenannte \textit{Cluster} zu unterteilen, wobei die Elemente dieser Gruppen sich möglichst ähnlich (\textit{homogen}), jedoch auch gleichzeitig von den anderen Clustern deutlich zu unterscheiden sein sollten (\textit{heterogen}).\seFootcite{Vgl.}{S. 3}{Anderberg.2014}
\enlargethispage{\baselineskip} 
\item \textit{Assoziationsanalyse}
\\ Diese Methode versucht Wissen durch assoziative Beziehungen zwischen den Daten herzuleiten. Das einfachste Beispiel hierfür wäre im Einzelhandelsbereich: \glqq Wenn ein Kunde Produkt A kauft, würde dieser auch Produkt B kaufen.\grqq~Durch diese extrahierten Muster, können wiederum Regeln abgeleitet werden.
\end{itemize}


\paragraph{Visualisierung als Werkzeug:}
Nicht zuletzt ist die Visualisierung unerlässlich für den Erfolg eines Data-Mining-Projektes. Die Resultate werden oftmals zur Entscheidungsfindung herangezogen, wobei die Entscheidungsträger nicht immer direkt am Prozess beteiligt sind. Die Ergebnisse müssen folglich in einer anschaulichen und nachvollziehbaren Form dargestellt werden, um Vertrauen und Akzeptanz in die Resultate zugewinnen.\seFootcite{Vgl.}{S. 14}{Cleve.2014} Weiterhin kann die Visualisierung auch schon in der Datenvorverarbeitung genutzt werden oder als eigenständige Methode innerhalb des Data Minings, da sich häufig erst Zusammenhänge zwischen den Attributen durch die Darstellung der Daten erkennen lassen.\seFootcite{Vgl.}{S. 14}{Cleve.2014}

\paragraph{Auswahl der Methode:}
Für die Modellierung einer Funktion zur Berechnung der Wahrscheinlichkeit eines Torerfolges im Fußball (auch bekannt unter dem Begriff \textit{Expected Goals}), kann die passende Data-Mining-Methode aus der \vref{dmmethods} ausgewählt werden. Der zu erwartende Torerfolg soll folglich prognostiziert und durch ein mathematisches Modell repräsentiert werden. Unter den statistischen Methoden eignet sich für die Modellierung einer Funktion am besten die Regressionsanalyse, da ein Torerfolg von mehreren Faktoren abhängig ist. Dementsprechend wird dieses Verfahren als Data-Mining-Methode für die Beantwortung der vorliegenden Problemstellung ausgewählt und dessen Bestandteile zunächst in \vref{fm} betrachtet, um diese Technik in der späteren Umsetzung anwenden zu können.	

\begin{sidewaysfigure}
\centering
\includegraphics[scale=1.6]{se-wa-jpg/dmmethods}
\caption[Übersicht: Data-Mining-Methoden]{Übersicht: Data-Mining-Methoden\protect\footnotemark}
\label{dmmethods}	
\footnotetext{Vgl. Abbildung \textit{García} et al., Data preprocessing in data mining, 2015, S. 4.}
\end{sidewaysfigure}