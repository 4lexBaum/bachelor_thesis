\section{Knowledge Discovery in Data}
\label{kdd}

Das folgende Kapitel beschreibt den \textit{Knowledge-Discovery-in-Data}-Prozess, der im vorherigen Kapitel (vgl. \vref{dmkdd}) als grundlegende Methodik der Arbeit ausgewählt wurde. Hierzu werden die einzelnen Prozessschritte der Datenselektion, der Datenvorverarbeitung, der Datentransformation, der Data-Mining-Methoden, sowie der Interpretation der Ergebnisse konkretisiert, um diese in der späteren Umsetzung der wissenschaftlichen Aufgabe anwenden zu können.

\glqq Experten [...] haben realisiert, dass eine große Anzahl an Datenquellen der Schlüssel zu bedeutsamen Wissen sein kann und das dieses Wissen in dem Entscheidungsfindungsprozess genutzt werden sollten. Eine einfache \gls{sql}-Abfrage oder \gls{olap} reichen für eine komplexe Datenanalyse oft nicht aus.\grqq\seFootcite{Vgl.}{S. 1}{Adhikari.2015} Hier greift der in \vref{kddpic} dargestellte \gls{kdd}-Prozess, ein multiples iteratives Modell, indem die einzelnen Schritte solange wiederholt und aufeinander abgestimmt werden müssen, bis aus den zugrundeliegenden Daten, Wissen abgeleitet werden kann.\seFootcite{Vgl.}{S. 7}{Mariscal.2010} Das Data Mining selbst kommt erst nach ausführlicher Datenvorbereitung zum Einsatz und kann so zu einer automatischen und explorativen Anpassung eines Modells -- wie der Funktionsmodellierung (vgl. \vref{fm}) -- an riesige Datenmengen genutzt werden.\seFootcite{Vgl.}{S. 1}{Adhikari.2015}\seFootcite{Vgl.}{S. 7}{Mariscal.2010}

In der Literatur existieren unterschiedliche Vorstellungen der einzelnen Prozessschritte, wodurch es oftmals zu Überschneidungen zwischen den einzelnen Gebieten kommt. So findet sich die Methode der \textit{Data Integration} einerseits in der Datenselektion wieder, andererseits auch in der Datenvorverarbeitung.\seFootcite{Vgl.}{S. 1}{Garcia.2015}\seFootcite{Vgl.}{S. 198}{Cleve.2014} Im Folgenden wird versucht, diese Schritte klar von einander abzutrennen. Hierbei wird sich größtenteils an den Ausarbeitungen von Han et al. und Cleve et al. orientiert.


