\subsection{Datentransformation}
Nachdem die (Roh-)Daten selektiert wurden, bereinigt und auf eine relevante Zielmenge reduziert wurden, müssen diese nur noch in eine adaptierte Form für die Algorithmen des Data Minings transformiert werden.\seFootcite{Vgl.}{S. 112}{Han.2012} Oftmals müssen sogar neue Attribute aus einem Datensatz kreiert werden, da diese nicht in geeigneter Struktur für das Data-Mining-Verfahren vorliegen.\seFootcite{Vgl.}{S. 48}{Garcia.2015} Dazu gibt eine Reihe an unterschiedlichsten Transformationsmöglichkeiten, wobei in dieser Arbeit ein Auszug der relevanten Methoden vorgestellt wird:

\paragraph{Codierung}
Liegen beispielsweise Attribute mit einer ordinalen Ausprägung vor (wie \textit{sehr groß, groß, mittel, klein}, müssen diese bei einer Verwendung des \gls{knn}-Algorithmus in numerische Werte umgewandelt werden -- Werte zwischen 0 und 1. Hierbei würde sich folgende Codierung für das Attribut \textit{Körpergröße} anbieten:\seFootcite{Vgl.}{S. 210}{Cleve.2014}

\begin{itemize}
\item \textit{sehr groß} $\rightarrow$ 1
\item \textit{groß} $\rightarrow$ 0,66
\item \textit{mittel} $\rightarrow$ 0,33
\item \textit{klein} $\rightarrow$ 0
\end{itemize}

Die Ordnungsrelation, sprich \textit{sehr groß > groß > ...}, darf dabei jedoch nicht verloren gehen. In Abhängigkeit des jeweiligen Verfahrens, müssen Daten oftmals codiert werden, sowie dies bei Maßeinheiten immer wieder der Fall ist.\seFootcite{Vgl.}{S. 211}{Cleve.2014}

\paragraph{Normalisierung und Skalierung}
Unterschiedliche Maßeinheiten -- wie \textit{Körpergröße} und \textit{Körpergewicht} -- können die Datenanalyse negativ beeinflussen und müssen daher in eine einheitliche Skalierung transformiert werden, um eine gleiche Gewichtung aller Attribute zu erreichen. Man bedient sich hierbei in aller Regel an der \textit{Min-Max-Normalisierung} (siehe \vref{minmax}) oder der \textit{Z-Transformation}, um numerische Werte auf ein [0,1] Intervall zu normieren zu können.\seFootcite{Vgl.}{S. 114}{Han.2012}\seFootcite{Vgl.}{S. 212}{Cleve.2014}

\begin{figure}[H]
\begin{equation}
x_{neu} = \frac{x - min(x_i)}{max(x_i) - min(x_i)}
\end{equation}
\caption{Min-Max-Normalisierung}
\label{minmax}
\end{figure}

\paragraph{Datenaggregation}
Nicht nur aus Sicht der Datenkompression (vgl. \vref{datenkompression}) ist die Datenaggregation erforderlich, vielmehr \glqq kann die Aggregation aus inhaltlichen Gründen sinnvoll sein.\grqq\seFootcite{}{S. 212}{Cleve.2014} Wenn Daten in einer zu detaillierten Ebene vorliegen -- wie beispielsweise Einwohnerzahlen von Stadtteilen -- müssen diese für einen Städtevergleich erst summiert werden, um Aussagen treffen zu können. Je nach Kontext können verschiedene Aggregationsmethoden (wie z.B. Summenbildung, Durchschnitt, usw.) für die Transformation zu einem einzigen Wert angewendet werden.\seFootcite{Vgl.}{S. 112}{Han.2012}

\paragraph{Datenglättung}
