\subsection{Datenselektion}
Die Datenselektion befasst sich hauptsächlich mit der Auswahl der geeigneten Datenmengen -- der \textit{Zieldaten} -- auf Basis derer die spätere Erforschung ausgeübt wird.\seFootcite{Vgl.}{S. 42}{Fayyad.1996} Der Datenanalyst befasst sich in dieser Phase mit der Bestimmung der für die Analyse geeigneten Daten und des Exports dieser Datenauswahl beispielsweise in eine Datenbank. Die selektierten Daten können zum Beispiel technischen oder rechtlichen Restriktionen unterliegen, wie zum Beispiel Zugriffs- oder Kapazitätsbeschränkungen. Hierbei sollte auf eine repräsentative Teilmenge des Datenbestandes zurückgegriffen werden.\seFootcite{Vgl.}{S. 9}{Cleve.2014}