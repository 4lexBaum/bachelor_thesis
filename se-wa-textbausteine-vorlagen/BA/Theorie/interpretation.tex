\newpage
\subsection{Interpretation}
\label{int}
Am Ende jedes \gls{kdd}-Prozesses stehen die Interpretation sowie die Evaluation der entdeckten Muster und Beziehungen aus dem Data Mining. Oftmals können Unternehmen keinen Nutzen aus den Analyseverfahren ziehen, da diese häufig irrelevante, triviale, bedeutungslose oder sogar bereits bekannte Daten generieren. Die gewonnenen Muster sollten den folgenden vier Kriterien genügen, um neues Wissen zu repräsentieren:\seFootcite{Vgl.}{S. 11-12}{Cleve.2014}

\begin{enumerate}
\item \textbf{Validität:} Hierbei wird die Gültigkeit des Musters für das gefundene Modell, als auch in Bezug auf neue Daten, mit Hilfe eines objektiven Maßstabs bewertet.
\item \textbf{Neuartigkeit:} Das Kriterium beantwortet die Frage, inwiefern das neu erworbene Wissen zu den bisherigen Forschungen steht. Einerseits kann es den Wissensstand ergänzen oder im Widerspruch zu diesem stehen.
\item \textbf{Nützlichkeit:} Beschreibt den Nutzen, welcher für den Anwender durch die Resultate erzielt wird.
\item \textbf{Verständlichkeit:} Die Ergebnisse des Modells sollten von einem anderen Anwender verstanden werden.
\end{enumerate}

Anhand dieser Anforderungen sollen die späteren Resultate der modellierten Funktionen gemessen werden. Um dabei eine aussagekräftige Interpretation der Ergebnisse treffen zu können, erfordert es ein hohes Maß an Verständnis für die vorliegende Problemstellung. Dazu bietet sich ein Team von Experten an, welche die Resultate validiert, sodass eine korrekte Bewertung erzielt werden kann. Für die Interpretationsphase ist die Verwendung von Werkzeugen, wie das der Visualisierung, sinnvoll, um schnellen Aufschluss über die gewonnenen Muster und Zusammenhänge zu erlangen. Innerhalb des iterativen \gls{kdd}-Prozesses (siehe \vref{kddpic}) ist ein Rücksprung in die vorherigen Phasen typisch.\seFootcite{Vgl.}{S. 11}{Cleve.2014} Meist müssen Daten erneut nachbereitet, eine andere Data-Mining-Methode ausgewählt oder sogar Daten neu selektiert werden, wenn das  angestrebte Ergebnis mit der verwendeten Datenbasis nicht realisierbar ist.\seFootcite{Vgl.}{S. 11}{Cleve.2014}