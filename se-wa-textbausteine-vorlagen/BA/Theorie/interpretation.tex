\subsection{Interpretation}
Am Ende des \gls{kdd}-Prozesses steht die Interpretation sowie die Evolution der entdeckten Muster und Beziehungen aus dem Data Mining. Oftmals können Unternehmen keinen Nutzen aus den Analyseverfahren erzielen, da diese häufig irrelevante, triviale, bedeutungslose oder sogar bereits bekannte Daten generieren. Die gewonnenen Muster sollten den folgenden vier Kriterien genügen, um neues Wissen zu repräsentieren:\seFootcite{Vgl.}{S. 11-12}{Cleve.2014}

\begin{enumerate}
\item \textbf{Validität:} Hierbei wird die Gültigkeit des Muster für das gefundene Modell, als auch in Bezug auf neue Daten, in einem objektives Maßstab beschrieben.
\item \textbf{Neuartigkeit:} Das Kriterium beantwortet die Frage, inwiefern das neu erworbene Wissen zu den bisherigen Forschungen steht. Einerseits es kann den Wissensstand ergänzen oder im Widerspruch dazu stehen.
\item \textbf{Nützlichkeit:} Beschreibt das Nutzen, welches für den Anwender durch die Resultate erzielt wurde.
\item \textbf{Verständlichkeit:} Die Ergebnisse des Modells sollten von einem anderen Anwender verstanden werden.
\end{enumerate}

Anhand dieser Anforderungen sollen die späteren Ergebnisse der modellierten Funktionen gemessen werden. Um dabei eine aussagekräftige Interpretation der Ergebnisse treffen zu können, erfordert es ein hohes Maß an Verständnis der vorliegenden Problemstellung. Zu bietet sich ein Team von Experten an, welche die Resultate validieren, sodass eine korrekte Bewertung erzielt wird. Für die Interpretationsphase eignet sich die Verwendung von Werkzeugen, wie der Visualisierung, um schnellen Aufschluss über die gewonnenen Muster und Zusammenhänge zu erlangen. Innerhalb des iterativen \gls{kdd}-Prozesses (siehe \vref{kddpic}) ist ein Rücksprung die vorherigen Phasen typisch.\seFootcite{Vgl.}{S. 11}{Cleve.2014} Meist müssen Daten nochmal nachbereitet, eine andere Data-Mining-Methode ausgewählt oder sogar Daten neu selektiert werden, wenn das gewünschte Ergebnis sich mit der verwendeten Datenbasis nicht erreichen lässt.\seFootcite{Vgl.}{S. 11}{Cleve.2014}